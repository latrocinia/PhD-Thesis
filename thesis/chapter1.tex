\startcomponent chapter1

\product thesis 

\environment layout


\Chapter{Introduction}


\Section{Structural biology in the Omics-age}

Since the start of modern-day Western science, Man is on a mission to
thoroughly study Nature in order to understand, manipulate, and overcome her
\cite[Nietzsche1891]. Above all, a fundamental insight into life is a hallmark
in the whole scientific enterprise, biologically represented in its irreducable
form by the cell. The cell is a highly complex system that is regarded as the
building block of life and is able to reproduce itself independently. Even
though DNA holds a full blueprint of an organism, studied by the field of
genomics, cells themselves are mainly organized by proteins, giving rise to the
field of proteomics, and their interactions, the field of interactomics
\cite[Braun2012]. Recent technological and methodological advances have enabled
the inquiry of the interaction networks that are formed by proteins, and showed
that the set of all interacting protein-complexes, the interactome, is several
orders of magnitude larger than the total number of proteins that the genome
encodes for, the proteome \cite[Stein2011]. In addition, inhibiters of
protein-protein interactions are an upcoming class of molecules with a profound
impact on drug-development \cite[Wells2007]. 

The field of structural biology tries to understand the workings of the
molecules of life by studying their structure, preferable up to atomic
resolution, as this provides a functional and mechanical description of the
system \cite[Campbell2002] and a basis for rational drug design
\cite[Bienstock2012, Sable2015]. The latter can be achieved by high-resolution
methods, mainly X-ray crystallography and NMR spectroscopy. Unfortunately, both
methods are hampered by several limitations. X-ray crystallography is mainly
limited by the production of high-quality crystals, an undertaking that becomes
more difficult with increasing structure size and flexibility of the
macromolecules; for NMR spectroscopy it is mainly the size of proteins that is
limiting structure determination, as spectra become heavily congested for
larger complexes, making peak assignment infeasible. Furthermore, neither
method is amenable for high-throughput investigations, a necessary requirement
for the structural elucidation of the interactome.

In order to close the structure knowledge gap, computational methods have been
devised to aid in this quest. Homology modeling is a successful approach to
predict the structure of a complex with high-sequence identity to another
already known structure, and heavily extends the structural knowledge of the
proteome \cite[Marti-Renom2000]. Macromolecular docking is the field that
occupies itself with predicting the structure of a complex starting from their
individual components \cite[Moreira2010], and can be divided in two main
approaches: template based docking, similar to homology modeling, and free
docking. It has been shown recently that templates are available for most
complexes of structurally characterized proteins \cite[Kundrotas2012]. However,
this approach is only amenable for complexes for which co-crystallized
templates are available \cite[Vakser2013]. The free docking approach can be
further subdivided in ab initio docking and data-driven docking. The former
solely uses shape matching and physico-chemical principles to predict the
structure of complexes with a limited success rate \cite[Huang2015]; the
data-driven approach tries to increase the success rate by including additional
information from biophysical and biochemical methods during the docking
\cite[Karaca2013, Rodrigues2014].  Data-driven docking is also more popularly
known as hybrid or integrative modeling of biomolecular complexes.


\Section{Integrative modeling of biomolecular complexes}

Integrative modeling is a procedure in which data from diverse sources are
combined to accurately predict a model of a biomolecular complex
\cite[Alber2007, Ward2013]. The procedure can be abstracted in four stages
\cite[Schneidman-Duhovny2014]:

\startitemize[n]

\item Gathering information: collect information in the form of e.g.
experimental data, bioinformatics predictions, statistical inference, just
about anything that can be of use during the modeling.

\item Model representation and evaluation: the degrees of freedom of the model
should be chosen, such as an all-atom model or a more coarse-grained
representation, depending on the information content the data are providing.
The data can be used to actively search for proper models and its information
content should be transformed accordingly.

\item Sampling and optimization: the sampling and optimization protocols should
be chosen depending on the degrees of freedom of the system. For a 6
dimensional system, corresponding to the relative placement of two 3D rigid
bodies, an exhaustive search might be performed, while for higher-dimensional
systems Monte Carlo and simulated annealing approaches might be more efficient.

\item Scoring and analysis: the resulting models need to be scored, ranked and
clustered based based on their congruency with the data to acertain model
precision and accuracy.

\stopitemize

In the remainder of this section we will mainly describe sources of data to use
during the modeling, and describe software packages that are geared towards
integrative modeling.


\Subsection{Sources of information}

In addition to the high-resolution techniques, many other experimental methods
have been devised to extract structural or low-resolution information from
biomolecular complexes. NMR spectroscopy is also capable of pinpointing
interface residues through the use of chemical shift perturbations (CSPs)
\cite[Case2013], and the relative orientation of subunits to each other by
residual dipolar couplings (RDCs) \cite[Chen2012], among several other methods.
Small angle X-ray scattering (SAXS) experiments result in a 1D scattering
curve, from which a diverse set of parameters can be determined with structural
interpretation, e.g. radius of gyration, and even complete shapes, though with
low-resolution \cite[Putnam2007, Schneidman-Duhovny2012]. Biochemical methods
such as mutagenesis and radical footprinting provide information on the binding
interface. Bioinformatics prediction methods can also deliver this information
by analyzing sequences and extract conserved interface residues through
co-evolution \cite[Hopf2014]. Two other experimental approaches that provide
shape data and distance restraints are cryo-electron microscopy (cryo-EM) and
chemical cross-linking coupled with mass-spectrometry and we will discuss them
more in-depth.


\Subsubsection{Cryo-electron microscopy}

Cryo-EM is a set of various transmission electron-microscopy techniques, namely
cryo-electron tomography (cryo-ET), electron crystallography, and
single-particle cryo-EM, that all ultimately results in a three-dimensional
density of the sample \cite[Milne2013]. In cryo-ET whole cell slices are
studied by systematically tilting the sample and imaging projections from
different tilt angles; electron crystallography is mainly aimed at
investigating membrane proteins that can form two-dimensional crystals or
helices; single-particle cryo-EM is used to study individual macromolecular
assemblages by imaging many projections of random orientations of the assembly.

However, all three approaches are are limited by the same phenomenon: the
prolonged irradiation of the specimen with electrons results in extensive
damage, reminiscent of the impact of a nuclear bomb \cite[Glaeser1978].  To
diminish this effect, the sample is typically plunge-freezed in liquid ethanol
to instantly vitrify it, resulting in a near-native hydrated state.  However,
the allowed electron dose is still severely limited, resulting in very noisy
projections, well below atomic resolution. Electron crystallography tries to
improve on this by combining the well-defined phases of the images with the
high-resolution electron diffraction pattern to attain atomic resolution.
Cryo-ET can signicantly increase the resolution of particular assemblages by
subtomogram averaging: a process where similar particles are aligned and
averaged, resulting in an increased signal-to-noise ratio. Singe-particle
cryo-EM in turn images many particles on a grid, each with a random
orientation. By aligning similarly oriented projections, class averages can be
obtained with a highly improved signal-to-noise ratio. If enough class averages
are available, the three-dimensional density can be reconstructed through
several iterative approaches. 

Thanks to recent dramatic advances in direct electron detectors and improved
particle processing software, the resolution of cryo-EM has impressively
increased and sky-rocketed the cryo-EM field from blob-ology \cite[Smith2014]
to the rising star in structural biology (subtitle of the cryo-EM Gordon
Research Conference 2014), to revolutionizing structural biology
\cite[Bai2015]. Although electron diffraction resolution has remained the same
at around 2\Angstrom\ \cite[Gonen2005], cryo-ET's subtomogram averaging now
attains sub-nanometer resolution \cite[Schur2013], and the single-particle
cryo-EM resolution record for a now stands at 2.2\Angstrom\
\cite[Bartesaghi2015].

Still, despite all the advances, the resolution of cryo-EM densities are
typically too low for ab initio structural modeling. The information content of
cryo-EM data is highly dependent on the resolution, with individual domains
becoming visible at 10\Angstrom, secondary structure elements at 9\Angstrom,
the separation of beta-sheets and bulky side-chains at around 4\Angstrom\
\cite[Baker2010]. Thus, for typical cryo-EM data of 7\Angstrom\ resolution and
lower, additional data need to be incorporated in an integrative approach to
attain an atomic model of the macromolecular assembly.


\Subsubsection{Chemical cross-linking coupled with mass spectrometry}

A very different method from cryo-EM is chemical cross-linking coupled with
mass spectrometry (CXMS). Here, proteins complexes are covalently linked with
chemical cross-links to determine spatial proximity between components. A
standard CXMS experiment consists of six stages \cite[Tran2015]. First the
cross-links are added to the sample after optimizing the reaction parameters,
and the cross-linked proteins are isolated to reduce the number of
false-positives. The third step is to digest the cross-linked proteins into
peptides using trypsine or other chemical agents, after which the peptides are
enriched using physico-chemical methods. The two final steps are MS
optimization for peptide detection and data-processing to detect cross-linked
residues. 

Even though the procedure is straightforward, each step is marked by
optimization and many parameters need to be chosen, such as which linker to
use, and how to enrich the cross-linked peptides \cite[Merkley2013]. However,
the major bottleneck is the final data analysis as millions to billions of
fragments can be produced and need to be considered \cite[Tran2015]. After a
successful analysis, the cross-linked peptides can be mapped back on the
proteins and distance restraints between components can be derived, where the
length of the linker is an idicator for the acceptable range of the restraint.
The shorter the linker the more information the restraint provides, though at
the price of a reduced number of formed links. So again, the inclusion of the
low-resolution long-range distance restraints provided by CXMS require an
integrative approach to accurately and precisely model the protein assemblies.


\Subsection{Software packages and platforms}

Performing integrative modeling requires dedicated high-end software packages
with powerful minimization, optimization and sampling algorithms. Currently,
there are two main software packages and platforms available that can handle
data from a substantial number of experimental methods. One is the Integrative
Modeling Platfrom (IMP) developed by the Sali lab \cite[Russel2012], and the
other is our in-house data-driven docking software HADDOCK (High Ambiguity
Driven DOCKing) \cite[Dominguez2003, deVries2007]. 


\Subsubsection{IMP}

The IMP software package is consists of several layers, each given more control
to the user \cite[Russel2012, Webb2011]. The base layer is written in C++ for
speed, where each class is encapsulated for use in Python. This provides a
scripting interface to setup an integrative modeling approach with data derived
from diverse sources translated to restraints. A level higher are direct user
applications, such as MultiFit for cryo-EM \cite[Lasker2009, Lasker2010] and
FoXS for the calculation of SAXS curves \cite[Schneidman-Duhovny2010]. In
addition, the IMP package is also integrated into the molecular graphics
visualization program UCSF Chimera \cite[Yang2012]. IMP is Free Software,
licenced under the LGPL and GPL.


\Subsubsection{HADDOCK}

The HADDOCK software's inception was in 2003, starting out as a binary protein
docking program originally capable of incorporating CSP data and bioinformatics
predictions on the interface during in the form of ambiguous interaction
restraints (AIRs) \cite[Dominguez2003]. Since then HADDOCK's capabilities have
steadily increased, and now also supports the use of RDCs \cite[vanDijk2005],
protein-DNA docking \cite[vanDijk2006], solvated docking using explicit water
\cite[vanDijk2006a, Kastritis2012], docking up to 6 components
\cite[Karaca2010], NMR pseudocontact shifts \cite[Schmitz2011], SAXS and
collision cross sections derived from MS \cite[Karaca2013], and protein-peptide
docking \cite[Trellet2013]. The HADDOCK webserver was introduced in 2010
\cite[deVries2010] for easy usage to the science community. The HADDOCK
software is free to use for academic purposes and ships with its source code. 


\Section{Explorative modeling}


The goal of integrative modeling ultimately is to produce representing
biomolecular assemblies that are consistent with the data, thus putting the
emphasis on the model. We can also turn this around, and instead put the
emphasis on the data and aim at quantifying the information content. This
different paradigm I call explorative modeling. A hallmark of this approach is
to systamatically sample a decent representative portion of the degrees of
freedom of the system under investigation, and calculating for each sampled
point the fit with the data, ultimately given a distribution of fits. The
method is inherently computationally demanding as the number of points to
sample is sizeable by itself and increases exponentially with the number of
degrees of system being investigated. However, for two-body systems,
corresponding to 6 degrees of freedom, the approach can still be applied. The
goal of explorative modeling is thus to provide the information content of the
data, and preferably visualize this to the structural biologist, to aid in
appreciation the impact of the restraints, to guide future work, and give
insight into model uncertainty.


\Section{Overview of thesis}

This thesis primarely describes new computational methods to handle cryo-EM and
distance restraints data for integrative and explorative modeling. In
\inchapter[chapter:powerfit] we introduce a high-performance cross-correlation
based rigid-body fitting software package to automatically fit high-resolution
structures in low-resolution cryo-EM density maps, called PowerFit. In addition
to algorithm optimizations, it provides a new, more sensitive scoring function
to further extend the applicable resolution range.  In
\inchapter[chapter:image-pyramids] we explore the resolution limits of
rigid-body fitting in cryo-EM data and leverage this information to heavily
accelerate the procedure through the use of multi-scale image pyramids.
\inchapter[chapter:haddock-em] describes the incorporation of cryo-EM data in
the HADDOCK software. The approach can be fully combined with all other
available sources of information in HADDOCK, resulting in a truly integrative
approach. Next, in \inchapter[chapter:haddock2.2-webserver], I present the
HADDOCK2.2 webserver, an upgrade of the HADDOCK webserver, for user-friendly
integrative modeling of biomolecular complexes. \inchapter[chapter:disvis]
deals with quantifying and visualizing the information content of distance
restraints in general, and cross-link data in particular. It introduces the
concept of the accessible interaction space, the set of all possible solutions
of a complex, and sets out a way to exhaustively enumerate this, implemented in
the another software package DisVis. We extend the approach further in
\inchapter[chapter:inferring-interface-residues], where interface residues are
inferred from the distance restraints. The inferred residues can subsequently
be used in the HADDOCK software to complement the docking. In the final
Chapter, we present a perspective on the field of integrative modeling and
propose further lines of research.

\stopcomponent
