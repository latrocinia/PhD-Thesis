\startcomponent chapter1

\product thesis 

\environment layout


\startChapter[title=Introduction, reference=chapter:introduction]


\startSection[title={Structural biology in the Omics-age}]

Since the start of modern-day Western science, Man is on a mission to
thoroughly study Nature in order to understand, manipulate, and overcome her
\cite[Nietzsche1891]. Above all, a fundamental insight into life is a hallmark
in the whole scientific enterprise, where life is biologically represented in
its irreducible form by the cell. The cell is a highly complex system that is
regarded as the building block of life and is able to reproduce itself
independently. Even though DNA holds a full blueprint of an organism, studied
by the field of genomics, it is mainly the proteins that orchestrate the
organization and functioning of cells, which has given rise to the field of
proteomics, and the field of interactomics to characterize their interactions
\cite[Braun2012]. Recent technological and methodological advances have enabled
the inquiry of the interaction networks that are formed by proteins, and showed
that the set of all interacting protein complexes, the interactome, is 1 to 2
orders of magnitude larger than the total number of proteins that the genome
encodes for, the proteome \cite[Stein2011]. Inhibitors of these protein-protein
interactions are an upcoming class of molecules with a profound impact on
drug-development \cite[Wells2007]. 

The field of structural biology tries to understand the workings of the
molecules of life by studying their structure, preferable up to atomic
resolution, as this provides a functional and mechanical description of the
system \cite[Campbell2002] and a basis for rational drug design
\cite[Bienstock2012, Sable2015]. Three dimensional, atomic-resolution
structural information can be obtained by high-resolution methods, mainly X-ray
crystallography and NMR spectroscopy. Unfortunately, both methods are hampered
by several limitations. X-ray crystallography is mainly limited by the
production of high-quality crystals, an undertaking that becomes more difficult
with increasing structure size, flexibility of the macromolecules, and
transient complexes; for NMR spectroscopy it is mainly the size of proteins
that is limiting structure determination, as spectra become heavily congested
for larger complexes, making peak assignment infeasible. Furthermore, neither
method is amenable to high-throughput investigations of complexes and large
assemblies, a necessary requirement for the structural elucidation of the
interactome.

In order to close the structure knowledge gap, computational methods have been
devised to aid in this quest. Homology modeling is a successful approach to
predict the structure of a protein with high-sequence identity to another
already known structure, and heavily extends the structural knowledge of the
proteome \cite[Marti-Renom2000]. Macromolecular docking is the field that
occupies itself with predicting the structure of a complex starting from their
individual components \cite[Moreira2010], and can be divided in two main
approaches: template based docking, similar to homology modeling, and “free”
docking. It has been shown recently that templates are available for most
complexes of structurally characterized proteins \cite[Kundrotas2012]. However,
this approach is only amenable to complexes for which co-crystallized
templates are available \cite[Vakser2013]. The “free” docking approach can be
further subdivided into ab initio docking and data-driven docking. The former
solely uses shape matching and physico-chemical principles to predict the
structure of complexes with a limited success rate \cite[Huang2015]; the
data-driven approach tries to increase the success rate by including additional
information from biophysical and biochemical methods during the docking
\cite[Karaca2013a, Rodrigues2014]. Data-driven docking is also more popularly
known as hybrid or integrative modeling of biomolecular complexes.

\stopSection

\startSection[
title={Integrative modeling of biomolecular complexes},
]

Integrative modeling is a procedure in which data from diverse sources are
combined to accurately predict a model of a biomolecular complex
\cite[Alber2007a, Ward2013]. The procedure can be abstracted in four stages
\cite[Schneidman-Duhovny2014]:

\startitemize[n]

\item \emph{Gathering information:} collect information in the form of
experimental data, bioinformatics predictions, statistical inference, or just
about anything that can be of use during the modeling.

\item \emph{Model representation and evaluation:} the degrees of freedom of the
model should be chosen, i.e. using an all-atom model or a more coarse-grained
representation, depending on how much information the data provide. In
addition, scoring functions for each data-type need to be determined to indicate
consistency between the models and the data.

\item \emph{Sampling and optimization:} the sampling and optimization protocols
should be chosen depending on the degrees of freedom of the system. For a 6
dimensional system, corresponding to the relative placement of two three-dimensional rigid
bodies, an exhaustive search can be performed, while for higher-dimensional
systems Monte Carlo and simulated annealing approaches would be more efficient.

\item \emph{Scoring and analysis:} the resulting models need to be scored, ranked and
clustered based on their congruency with the data to ascertain model precision
and accuracy. 

\stopitemize

In the remainder of this section we will mainly describe sources of data to use
during the modeling, and describe software packages that are geared towards
integrative modeling.


\startSection[title={Sources of information}]

In addition to the high-resolution structural techniques, many other
experimental methods have been devised to extract structural or low-resolution
information. NMR spectroscopy is also capable of pinpointing interface residues
through the use of chemical shift perturbations (CSPs) \cite[Case2013], and the
relative orientation of subunits to each other by residual dipolar couplings
(RDCs) \cite[Chen2012], among several other methods \cite[vanIngen2014]. Small
angle X-ray scattering (SAXS) experiments result in a 1D scattering curve, from
which a diverse set of parameters can be determined with structural
interpretation, e.g. radius of gyration, and even complete (low-resolution)
shapes \cite[Putnam2007, Schneidman-Duhovny2012, Blanchet2013]. Biochemical
methods such as mutagenesis and radical footprinting provide information on the
binding interface.  Bioinformatics prediction methods can also deliver this
information by analyzing sequences and extract conserved interface residues
through co-evolution \cite[Hopf2014]. Two other experimental approaches that
provide shape data and distance restraints are cryo-electron microscopy
(cryo-EM) and chemical cross-linking coupled with mass-spectrometry (CXMS),
which we will discuss more in-depth in the following.


\startSubsection[title={Cryo-electron microscopy}]

Cryo-EM is a set of various transmission electron-microscopy techniques, namely
cryo-electron tomography (cryo-ET), electron crystallography, and
single-particle cryo-EM, that all ultimately results in a three dimensional
density of the sample \cite[Milne2013]. In cryo-ET whole cell slices are
studied by systematically tilting and imaging projections of the sample;
electron crystallography is mainly aimed at investigating membrane proteins
that can form two-dimensional crystals; single-particle cryo-EM is used to
study individual macromolecular assemblages by imaging many projections of
random orientations of the assembly.

However, all three approaches are limited by the same phenomenon: the prolonged
irradiation of the specimen with electrons results in extensive damage,
reminiscent of the impact of a nuclear bomb \cite[Glaeser1978].  To diminish
this effect, the sample is typically plunge-frozen in liquid ethane to
instantly vitrify and fixate it, resulting in a near-native hydrated state.
However, the allowed electron dose is still severely limited, resulting in very
noisy projections, well below atomic resolution. Electron crystallography tries
to improve on this by using the high-resolution electron diffraction pattern to
attain atomic resolution. Cryo-ET can significantly increase the resolution of
particular assemblages by subtomogram averaging: a process where similar
particles are aligned and averaged, resulting in an increased signal-to-noise
ratio. Singe-particle cryo-EM in turn images many particles on a grid, each
with a random orientation. By aligning similarly oriented projections, class
averages can be obtained with a highly improved signal-to-noise ratio. If
enough class averages are available, the three-dimensional density can be
reconstructed through several iterative approaches. 

Thanks to recent dramatic advances in direct electron detectors and improved
particle processing software, the resolution of cryo-EM has impressively
increased and sky-rocketed the cryo-EM field from blob-ology \cite[Smith2014]
to the rising star in structural biology (subtitle of the cryo-EM Gordon
Research Conference 2014), to revolutionizing structural biology \cite[Bai2015,
Nogales2015]. Although electron diffraction resolution has remained the same at
around 2Å \cite[Gonen2005], cryo-ET's subtomogram averaging now attains
sub-nanometer resolution \cite[Schur2013], and the single-particle cryo-EM
resolution record for now stands at 2.2Å \cite[Bartesaghi2015].

Still, despite all these advances, the resolution of cryo-EM densities are in
most cases typically too low for ab initio structural modeling. The information
content of cryo-EM data is highly dependent on the resolution, with individual
domains becoming visible at 15Å, secondary structure elements at 10Å for
helices and 7Å for \m{\beta}-strands, and the separation of beta-sheets and bulky
side-chains at around 4Å \cite[Baker2010]. Thus, for typical cryo-EM data of
7Å resolution and lower, additional data need to be incorporated in an
integrative approach to attain an atomic model of the macromolecular assembly.

\stopSubsection

\startSubsection[title={Chemical cross-linking coupled with mass spectrometry}]

A very different method from cryo-EM is chemical cross-linking coupled with
mass spectrometry. Here, protein complexes are covalently linked with chemical
cross-links to determine spatial proximity between components. A standard CXMS
experiment consists of six stages \cite[Tran2015]: 1) the cross-links are added
to the (purified) sample after optimizing the reaction conditions; 2) the
cross-linked proteins are isolated to reduce the number of false-positives; 3)
the cross-linked proteins are subsequently digested using trypsine or other
proteases; 4) the resulting peptides are enriched using physico-chemical
methods, such as size exclusion, affinity, and strong cation exchange
chromatography. The final two steps are 5) MS optimization for peptide
detection and 6) data-processing to detect cross-linked residues. 

Even though the procedure is straightforward, each step is marked by
optimization and many parameters need to be chosen, such as which linker to
use, and how to enrich the cross-linked peptides \cite[Leitner2010,
Merkley2013]. However, the major bottleneck is the final data analysis as
millions to billions of fragments can be produced and need to be considered
\cite[Tran2015]. After a successful analysis, the cross-linked peptides can be
mapped back on the proteins and distance restraints between components can be
derived, where the length and flexibility of the linker are used to define an
acceptable range for the distance restraint. The shorter the linker the more
information the restraint provides, though at the price of a reduced number of
formed cross-links. So again, the inclusion of the low-resolution long-range
distance restraints provided by CXMS require an integrative approach to
accurately and precisely model the protein assemblies. A few recent examples
where CXMS data were used are the INO80 complex of Saccharomyces cerevisiae
\cite[Tosi2013], the Polycomb Repressive Complex 2 \cite[Ciferri2012], and the
30S-elF1-elF3 translation initiation complex \cite[Erzberger2014]. 

\stopSubsection

\startSubsection[title={Software packages and platforms}]

Performing integrative modeling requires dedicated high-end software packages
with powerful minimization, optimization and sampling algorithms. Currently,
there are several software packages and platforms available that can handle
data from a substantial number of experimental methods, but I will focus on
three. One is the Rosetta software from the Baker lab \cite[Leaver-Fay2011],
the second is the Integrative Modeling Platform (IMP) developed by the Sali
lab \cite[Russel2012], and the third is our in-house data-driven docking
software HADDOCK (High Ambiguity Driven DOCKing) \cite[Dominguez2003,
deVries2007]. 

\startSubsubsection[title=Rosetta]

Rosetta is at its core a structure prediction software package, and is well
known for its elaborate and accurate scoring function and conformational sampling
techniques \cite[Rohl2004]. Although originally a de novo protein prediction
program \cite[Simons1999], it has ventured into a more integrative approach
and can now also perform X-ray crystallography refinement (MR-Rosetta)
\cite[DiMaio2011], use NMR data (CS-Rosetta), CXMS data \cite[Kahraman2013],
and recently also cryo-EM data \cite[Demers2014, DiMaio2015], resulting in the
current prediction software package juggernaut that it is today
\cite[Adams2013]. The Rosetta source code was recently rewritten with the
release of Rosetta3 \cite[Leaver-Fay2011]. Rosetta is free to use for academic
purposes.  

\stopSubsubsection

\startSubsubsection[title=IMP]

The IMP software package was from the outset designed as an integrative
modeling platform, and is well-known for its use in the development of a model
of the Nuclear Pore Complex \cite[Alber2007, Alber2007a]. The IMP software
consists of several user interface layers, each giving more control to the
user \cite[Russel2012, Webb2011]. The base layer is written in C++ for speed,
where each class is encapsulated for use in Python. This provides a scripting
interface to setup an integrative modeling approach with data derived from
diverse sources translated to restraints. One level higher are the direct user
applications, such as MultiFit for cryo-EM \cite[Lasker2009, Lasker2010] and
FoXS for the calculation of SAXS curves \cite[Schneidman-Duhovny2010]. In
addition, the IMP package is also integrated into the molecular graphics
visualization program UCSF Chimera \cite[Yang2012]. IMP is Free Software,
licensed under the LGPL and GPL.

\stopSubsubsection

\startSubsubsection[title=HADDOCK]

The first version of HADDOCK was created in 2003, starting out as a binary
protein docking program originally capable of incorporating CSP data and
bioinformatics predictions \cite[Dominguez2003]. Since then, HADDOCK's
capabilities have steadily increased, and now also supports the use of RDCs
\cite[vanDijk2005], relaxation anisotropy \cite[vanDijk2006b], protein-DNA
docking \cite[vanDijk2006a], solvated docking using explicit water
\cite[vanDijk2006, Kastritis2012], docking up to 6 components
\cite[Karaca2010], NMR pseudocontact shifts \cite[Schmitz2011], SAXS and
collision cross sections derived from MS \cite[Karaca2013], and
protein-peptide docking \cite[Trellet2013]. The HADDOCK web server was
introduced in 2010 \cite[deVries2010] to provide a user-friendly interface to
the science community. The HADDOCK software is free to use for academic
purposes and ships with its source code, but does require CNS (Crystallography
and NMR System) for its computational back end \cite[Brunger2007]. 

\stopSubsubsection

\stopSubsection

\stopSection


\startSection[title={Explorative modeling}]

The goal of integrative modeling ultimately is to produce representative models
of biomolecular assemblies that are consistent with the acquired data, thus
putting the emphasis on the structural models. However, this does not
necessarily provide insight into the information content of the restraints and
certainty in the models. We can also turn this around, and instead put the
emphasis on the data and aim at quantifying the information content by counting
all accessible states that are either consistent or inconsistent with the data.
I am referring to this different paradigm and associated field as
\emph{explorative modeling}. A hallmark of this approach is to systematically
sample a decent representative portion of the degrees of freedom of the system
under investigation, and calculating for each sampled point the fit with the
data, ultimately resulting in a distribution of states satisfying the input
data.  The method is inherently computationally demanding as the number of
points to sample is sizable by itself and increases exponentially with the
number of degrees of system being investigated. However, for two-body systems,
corresponding to 6 degrees of freedom assuming rigid entities, the approach is
manageable. The goal of explorative modeling is thus to provide the information
content of the data, and preferably visualize this to the structural biologist,
to aid in appreciating the impact of the data in restraining the accessible
conformational/interaction space, to give insight into model uncertainty, and
guide future work.

\stopSection


\startSection[{title=Overview of thesis}]

This thesis primarily describes new computational methods to handle cryo-EM
and distance restraints data for integrative and explorative modeling. In
\inchapter[chapter:powerfit] I introduce a high-performance cross-correlation
based rigid-body fitting software package called PowerFit to automatically fit
high-resolution structures in low-resolution cryo-EM density maps. In addition
to algorithm optimizations, it provides a novel and more sensitive scoring
function to further extend the applicable resolution range. In
\inchapter[chapter:image-pyramids] I explore the resolution limits of
rigid-body fitting in cryo-EM data and leverage this information to heavily
accelerate the procedure through the use of multi-scale image pyramids.
\inchapter[chapter:haddock-em] describes the incorporation of cryo-EM data in
the HADDOCK software. The approach can be fully combined with all other
available sources of information in HADDOCK, resulting in a truly integrative
approach. Next, in \inchapter[chapter:haddock2.2] I present the
HADDOCK2.2 web server, an upgrade of the HADDOCK web server, for user-friendly
integrative modeling of biomolecular complexes. \inchapter[chapter:disvis]
deals with quantifying and visualizing the information content of distance
restraints in general, and CXMS data in particular. It introduces the
concept of the accessible interaction space, the set of all possible solutions
of a complex, and defines a way to exhaustively enumerate the accessible
space. This is implemented in another software package called DisVis, and
represents a first step into explorative modeling. I extend the approach
further in \inchapter[chapter:inferring-interface-residues], where interface
residues are inferred from the accessible space defined by the distance
restraints. The inferred residues can subsequently be used in the HADDOCK
software to complement the docking process. In the final
\inchapter[chapter:summary-and-perspective] , I present a summary of the
thesis and provide a personal perspective on the field of integrative modeling,
proposing further lines of research.

\stopSection

\stopChapter

\stopcomponent
