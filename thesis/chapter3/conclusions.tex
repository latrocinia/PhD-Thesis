\Section{Conclusions}

Here we have explored the resolution limits of rigid body fitting in
high-resolution cryo-EM densities, ranging between 5.5 and 6.9\Angstrom\ resolution,
using 5 different ribosome cases. We have shown that also for experimental data
the L-CW-LCC score is the most sensitive of the 4 correlation-based scores
tested and that it can unambiguously fit most chains objectively at the correct
location. In addition, we quantified the success rate of fitting subunits based
on their size represented by their number of residues. As expected, larger
subunits require a lower level of detail to be unambiguously fitted into the
density. This phenomenon can be leveraged by building an image pyramid, i.e.
representing the data at different resolutions, and subsequently fitting a
subunit in the smaller, lower-resolution density dataset. The resulting speed
increase can be up to 30-fold for CPUs and 40-fold for GPUs with virtually no
loss in the success rate of fitting. We have implemented the use of
image-pyramids in PowerFit for fast objective fitting of high-resolution
structures in lower-resolution density maps. 

