\Section{Introduction}

A structural understanding of large macromolecular complexes is of fundamental
importance to explain and manipulate cellular processes. Cryo-electron
microscopy (cryo-EM) is quickly becoming the method of choice for studying
these macromolecular machines, as recent advances have enabled unprecedented
levels of detail to be visualized. Consequently, sub-nanometer resolution maps
are no exception anymore, though the level of detail is usually still too low
for de novo building of atomic structures. When possible, cryo-EM data are
therefore combined with high-resolution atomic models of subunits for a proper
structural understanding of the data. Typically, the first step in the modeling
is placing the subunits in the density as a rigid body, after which the models
can be refined using flexible fitting.

Many tools and software exist to help users in the rigid body fitting, both for
manual and automatic placement. Though manual placement is frequently
performed, most notably using UCSF Chimera, it is subjective and can lead to
over-interpretation of the data as there is no objective target function to be
optimized. The problem is exacerbated when flexible fitting is applied
afterwards, as it requires an initial local cross-correlation minimum between
the model and the density, else the model would drift away from its fitted
location. 

An automatic and objective method to determine the placement of the subunits is
by performing a full-exhaustive systematic cross-correlation search of the
three translational and three rotational degrees of freedom of the model in the
density. Many advances have been made in the sensitivity and speed of
cross-correlation based rigid body fitting. Recently, we introduced the
core-weighted local cross-correlation scores, based on work from Wu et al, in
our rigid body fitting package PowerFit. However, no thorough investigation
into the limits of rigid body fitting has been performed so far, nor has the
resolution requirements to fit a subunit of a certain size in the density been
quantified. In addition, the size of cryo-EM data has been steadily increasing
as a result of the higher information content, which unfortunately slows down
the exhaustive search considerably, as the search is usually performed using
FFT-techniques for fast translational scans.

Here we report on a comprehensive exploration of cross-correlation based rigid
body fitting in cryo-EM densities, using 6 high-resolution ribosome maps in the
range of 5.5 to 8.9\Angstrom\ for which high-resolution models have been fitted in the
density. We analyzed the success rate of fitting all XXX subunits in these maps
as a function of resolution by progressively lowering the resolution of the
initial data down to 30\Angstrom\ using four different scoring functions. Furthermore,
we show how the size of the subunit influences the success rate of fitting and
how over-interpreted regions of the map can be identified. Ultimately, we
leverage this information by using the concept of multi-scale image pyramids,
well known in the field of image analysis, to significantly reduce the
computational resources and time required to perform a fit up to two orders of
magnitude. We have implemented this in our PowerFit package, which can be
freely downloaded, for fast rigid body fitting in cryo-EM data.

