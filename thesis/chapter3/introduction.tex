\Section{Introduction}

A structural understanding of large macromolecular complexes is of fundamental
importance to rationalize and manipulate cellular processes. Cryo-electron
microscopy (cryo-EM) is quickly becoming the method of choice for studying
these macromolecular machines as recent advances are enabling unprecedented
levels of detail to be visualized \cite[Bai2015]. Sub-nanometer resolution maps
are no exception anymore, although the level of detail is usually still too low
for de novo building of atomic structures. When possible, cryo-EM data are
therefore combined with high-resolution atomic models of subunits for a proper
structural understanding of the data. Typically, the first step in the modeling
process is placing the subunits in the density as rigid bodies, after which the
models can be refined using some flexible fitting procedure
\cite[Esquivel-Rodriguez2013].

A variety of tools and software have been developed to help users in the rigid
body fitting, both for manual and automatic placement. Though manual placement
is frequently performed, most notably using UCSF Chimera \cite[Pettersen2004],
it is subjective and can lead to over-interpretation of the data, as there is
no objective target function to be optimized. The available local
cross-correlation function in UCSF Chimera is limited, as it samples only the
current orientation. The problem of manual fitting is exacerbated when flexible
fitting is applied afterwards, as it requires an initial local
cross-correlation minimum between the model and the density, else the model
would drift away from its fitted location. 

An automatic and objective method to determine the placement of the subunits is
to perform a full-exhaustive systematic cross-correlation search of the three
translational and three rotational degrees of freedom of the model in the
density. Many advances have been made in both sensitivity and speed of
cross-correlation based rigid body fitting \cite[Volkmann1999, Roseman2000,
Chacon2002, Wu2003, Garzon2007, Volkmann2009, Hrabe2012, Hoang2013,
Derevyanko2014].  In \inchapter[chapter:powerfit] we introduced the
core-weighted local cross-correlation scores in our rigid-body fitting package
PowerFit.  However, to our knowledge, no thorough investigation into the limits
of rigid body fitting has been performed so far, nor has the resolution
requirements to fit a subunit of a certain size in the density been quantified.
In addition, as the size of cryo-EM data has been steadily increasing as a
result of the higher information content, the CPU requirements for an
exhaustive search, which is usually performed using Fast Fourier Transform
(FFT)-techniques for fast translational scans, are considerably increasing,
which slows down the entire process.

Here we report on a comprehensive exploration of cross-correlation based
rigid-body fitting into cryo-EM densities, using five high-resolution ribosome
maps in the range of 5.5 to 6.9Å for which high-resolution models are
available. We analyze the success rate of fitting all 379 subunits into these
maps as a function of resolution using four different scoring functions. This
is done by progressively lowering the resolution of the initial data down to
30Å. Furthermore, we show how the size of the subunits influences the success
rate of fitting and how over-interpreted regions of the map can be identified.
Finally, we leverage this information by using the concept of multi-scale image
pyramids \cite[Cyganek2009], well known in the field of image analysis, to
significantly reduce the required computational resources and time to perform a
fit by up to two orders of magnitude. This is implemented in our PowerFit
package for fast rigid body fitting in cryo-EM data, which can be freely
downloaded from \from[url:powerfit].

