\Section{Results and discussion}

\Subsection{Success rate of different scoring functions}

We first determined the best performing cross correlation score for fitting
subunits in the experimental maps. As fine rotational searches are
computationally demanding, we performed at this stage solely a translational
correlation scan with the correct orientation of each chain. If no correct
local cross correlation minimum can be found here, a fine search is futile. The
success rate of fitting a subunit correctly is plotted against the resolution
of the data in \infigure[fig:fit-results]{A}. A fit is considered successful if
the subunit is placed within 2 voxels of the true solution, and we only
considered the best-ranked fit, i.e. the fit with the highest correlation
score. Congruent with an earlier analysis of noise-free simulated data, the
L-CW-LCC score performs the best of the four, followed by the L-LCC, CW-LCC and
LCC (see \inchapter[chapter:powerfit].  Remarkably, the L-CW-LCC is capable of
fitting about 90\% of the 379 subunits unambiguously down to a resolution of
13\Angstrom. Inclusion of the Laplace pre-filter has the biggest impact and
increases the applicable resolution extent by about 5\Angstrom. The
core-weighted approach has a smaller impact and extents the resolutions 1 to
2\Angstrom further. 

\placefigure[top][fig:fit-results] {\getbuffer[cap:fit-results]}
{\externalfigure[fig:fit-results]}


\Subsection{Size dependence of success rate}

To quantify the success rate of the L-CW-LCC further, we performed a fine
rotational search (6.6\Deg\ interval, 27672 orientations) for each subunit.  We
divided the chains in 7 categories based on their size represented by their
number of residues (\intable[tab:size-categories]). The success rate of fitting
each category of subunits is shown in \infigure[fig:fit-results]{B}. We again
only considered unambiguous fits, i.e.  the top ranked fit. As expected, the
smallest chains have the lowest success rate. Even when fitting in 8\Angstrom\
resolution data the success rate is smaller than 50\%. This increases to around
80\% already for subunits with a residue count of 50 to 100. The success rate
stabilizes to 90\% for larger sized chains and is stable down to 12\Angstrom\
resolution data. After the 12\Angstrom\ point, the success rate drops rapidly,
though less strongly for larger chains. For subunits larger than 500 residues,
which also include the rather large rRNA chains, the success rate remains
stable down to 20\Angstrom.  We conclude from this analysis that the bulk of
the subunits can be unambiguously fitted in the density down to 12\Angstrom\
resolution. 

\placetable[top][tab:size-categories] {\getbuffer[cap:size-categories]}
{\getbuffer[tab:size-categories]}

Interestingly the success rate spikes locally at 16\Angstrom\ resolution, and
is more pronounced for larger chains. This can also be observed in
\infigure[fig:fit-results]{A} only for the L-CW-LCC score. The reason for this
is not fully apparent. It might be an artifact of the core-weighted procedure:
the core-indices of subunits consisting of multiple subunits may shift suddenly
and coalesce, locally increasing the sensitivity of the score. Another reason
might be that for those subunits the local resolution is significantly lower,
and that fitting with a high-resolution template of the subunit results in too
much noise entering the correlation score, which is remedied by filtering the
template further down to lower resolutions. Although this has no impact on the
main finding of the fitting analysis, the observation is intriguing. 


\Subsection{Detecting over-interpreted regions of the density}

The advantage of objectively fitting subunits in the density and characterizing
the success rate is that it allows the identification of possibly
over-interpreted regions of the density. For example, in the largest size
category the elF3c chain (543 residues) of EMD-2845 was placed incorrectly in
the density. When inspecting the current fit
(\infigure[fig:overinterpreted-density]), we can see that the global features
of the chain are present in the density, although it is not of sufficient
resolution to identify the secondary structure elements, and that some parts
are sticking outside the density. This was also implicated by the authors, as
the local resolution of the density drops to around 10 to 15\Angstrom\ in that
region \cite[Aylett2015]. Interestingly, the chain is unambiguously fitted when
the resolution of the data is filtered to 21\Angstrom\ resolution, indicating
that the global features of the chains are indeed consistent with the data, but
the high-resolution structure might be over-interpreting the data. 

\placefigure[top][fig:overinterpreted-density]
{\getbuffer[cap:overinterpreted-density]}
{\externalfigure[fig:overinterpreted-density]}


\Subsection{Fast fitting with multi-scale image pyramids}

Now that resolution indicators are defined for reliably fitting a particular
size chain into the density, this knowledge can be leveraged through building a
multi-scale image pyramid to speed up the search. To demonstrate the speedup
that can be achieved, we applied our approach to another ribosome case with a
reported resolution of 5.7\Angstrom\ (EMD-2917, 5AKA)
\cite[vonLoeffelholz2015]. We constructed an image pyramid by filtering and
resampling the original data down to 9, 12, 13 and 20\Angstrom\ resolution
(\infigure[fig:image-pyramid-application]). We fitted chains larger than 50
residues: chains consisting of 50 to 100 residues were fitted in the
9\Angstrom\ resolution density, chains consisting of 100 to 300 residues in the
12\Angstrom\ data, chains consisting of 300 to 500 residues the 13\Angstrom\
map, and for subunits bigger than 500 residues we used the 20\Angstrom\ data.

All 31 chains could be unambiguously fitted considering only the top solution
with the best cross-correlation, with the exception of the 4.5S RNA consisting
of 74 residues. A local cross correlation maximum can be found at the correct
location, with the successful fit placed at rank 17. This is probably due to
the local resolution of the data dropping to around 10 to 12\Angstrom\ in that
region, indicating flexibility of the chain \cite[vonLoeffelholz2015]. The time
required to fit one subunit into the original map (\m{180 \times 180 \times
180} voxels) is approximately 10 hours using a single AMD Opteron 6320 CPU
processor and 40m for an NVIDIA GTX680 GPU. This reduces to 6h and 29m for the
9\Angstrom\ resolution data (\m{160 \times 160 \times 160}), 2h and 7m for the
12\Angstrom\ data (108 \times 100 \times 120), 1.5h and 5m for the 13\Angstrom\
data (\m{96 \times 96 \times 108}), and 20m and 1m for the 20\Angstrom\ data
(\m{64 \times 64 \times 72}), respectively. Thus, the speed increase is up to
30 times for CPU and 40 times for GPU calculations for the larger subunits, at
only a small cost in the success rate of fitting.

\placefigure[top][fig:image-pyramid-application]
{\getbuffer[cap:image-pyramid-application]}
{\externalfigure[fig:image-pyramid-application]}
