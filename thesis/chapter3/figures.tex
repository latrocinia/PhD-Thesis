\useexternalfigure[fig:image-pyramid-example][image-pyramid-example][width=0.75\textwidth]
\startbuffer[cap:image-pyramid-example]
\caption{Example of a multi-scale image pyramid of the D. melanogaster ribosome
(EMD-5591)}. 
{The time required for an exhaustive search increases with
increasing resolution.}
\stopbuffer

\useexternalfigure[fig:fit-results][fit-results][width=1.1\textwidth]
\startbuffer[cap:fit-results]
\caption{Aggregate fitting results of the 5 ribosome cases.}
{(A) Success rate of fitting a subunit unambiguously at the correct
position for four correlation scores as a function of the density map
resolution. (B) Success rate of correctly fitting a subunit consisting of a
given number of residues with the L-CW-LCC score as a function of the density
map resolution. The subunits were divided into seven categories based on their
respective number of residues.}
\stopbuffer

\useexternalfigure[fig:overinterpreted-density][overinterpreted-density]
\startbuffer[cap:overinterpreted-density]
\caption{The elF3c chains as currently placed in the density by manual rigid-body
fitting (EMD-2845, 4UER)}
{The density is shown at an iso-contour level of 0.03 (A) and 0.01 (B).}
\stopbuffer

\useexternalfigure[fig:image-pyramid-application][image-pyramid-application]
\startbuffer[cap:image-pyramid-application]
\caption{Cryo-EM data of E. coli ribosome (EMD-2917) at different resolutions
with the deposited structure (5AKA) fitted into the original map (left).}
{The resolution and the size of the data, the latter expressed in numbers of
voxels, are indicated under each density.}
\stopbuffer
