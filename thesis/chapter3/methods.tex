\Section{Methods}


\Subsection{Exploring the limits of rigid-body fitting}

To explore the resolution limit for successful rigid body fitting, we selected
five high-resolution cryo-EM ribosome maps from the EMDatabank
\cite[Lawson2011], ranging in resolution from 5.5 to 6.9\Angstrom, for which
structural models were deposited in the Protein Databank \cite[Gutmanas2014]
(\intable[tab:ribosome-cases]). The ribosome is an excellent case study as it
contains many chains of various sizes and types. Subsequently, we tried to fit
each separate chain independently in their respective density map with
PowerFit, using four different scoring functions: the local cross-correlation
(LCC), the core-weighted (CW-) LCC, and their Laplace pre-filtered versions the
L-LCC and L-CW-LCC, respectively, given by the master equation

\placetable[top][tab:ribosome-cases]
{\getbuffer[cap:ribosome-cases]}
{\getbuffer[tab:ribosome-cases]}

\placeformula[eq:correlation-equation]
\startformula
\text{CC} = \frac{1}{N} \frac{\sum_i^N \left( w_i \rho_c - \overline{\rho_c}
\right) \cdot \left( w_i \rho_o - \overline{\rho_o} \right)}{\sigma_c^w
\sigma_o^w}
\stopformula

where the summation is over all \m{N} voxels that are within half a resolution
distance of any atom of the search object indexed by \m{i}; \m{w_i} is a weight
factor given to voxel \m{i}; \m{\rho_c} and \m{\rho_o} are the intensities of
the search object and the cryo-EM density at voxel \m{i}, respectively;
\m{\overline{\rho_c^w}} and \m{\overline{\rho_o^w}} are the weighted density
average for the search object and the local cryo-EM data, respectively, given
by \m{\overline{\rho_x^w} = 1/N \sum_i^N w_i \rho_x}. Finally, \m{\sigma_c^w}
and \m{\sigma_o^w} are the weighted density standard deviations for the search
object and EM-data. The LCC-score is defined by setting \m{w_i} to 1, while for
the CW-LCC \m{w_i} is given by the core-index (\citeauthor{Wu2003},
\inchapter[chapter:powerfit]), a measure for how close the voxel is to the core
of the search object. The Laplacian enhanced scoring functions are defined by
mapping \m{\rho \to \nabla^2 \rho_x} \cite[Chacon2002]. 

After each round of fitting, the resolution of the cryo-EM data was lowered by
1\Angstrom\ using the following procedure. Assuming that the density is
described by a collection of atoms with a spherical Gaussian density
distribution, where the width of the Gaussian depends on the resolution of the
data, the density at each point \m{\vec{r}} in space is given by

\placeformula[eq:density]
\startformula
\rho \left( \vec{r}| R \right) = \sum^N_i A_i \exp\left( -\frac{\left| \vec{r}
- \vec{r}_i \right|^2 }{2 \sigma_R^2} \right)
\stopformula

Here the summation is over all \m{N} atoms indexed by \m{i}, where the
amplitude of the density is given by the atom number of the element \m{A_i},
\m{\vec{r}} is the position of atom \m{i}, and the spread \m{\sigma_R} is a
function of the resolution \m{R} given by

\placeformula[eq:sigma-to-resolution]
\startformula
\sigma\left( R \right) = \frac{1}{\sqrt{2} \pi} R
\stopformula

This definition of the resolution ensures that the amplitude of the specified
resolution is at \m{1/e} of the maximum in Fourier space. In order to lower the
resolution of the map to a lower target resolution, the density can simply be
convoluted with a Gaussian kernel as

\placeformula[eq:target-density]
\startformula
\rho_{\text{target}}\left( \vec{r} \right) = G_k \ast \rho_{\text{init}}
\stopformula

where \m{\rho_{\text{init}}} and \m{\rho_{\text{target}}} are the initial and
target density, respectively, and \m{G_k} is the Gaussian kernel with standard
deviation \m{\sigma_k}, and \m{*} is the convolution operator. The convolution
of two Gaussians results in another Gaussian \cite[Weisstein2015] as follows

\placeformula[eq:gaussian-convolution]
\startformula
G_1 \ast G_2 = A \exp\left[ - \frac{\left| \vec{r} - \left( \vec{r}_1 +
\vec{r}_2 \right)\right|^2} {2 \left(\sigma_1^2 + \sigma_2^2 \right)}\right]
\stopformula

where \m{G_1} and \m{G_2} are two Gaussian functions with center \m{\vec{r}_1}
and \m{\vec{r}_2} and width \m{\sigma_1} and \m{\sigma_2}, and \m{A} a
normalization constant of no interest here. Thus, \m{\sigma_k} is then simply

\placeformula[eq:sigma-kernel]
\startformula
\sigma_k = \sqrt{\sigma^2_{\text{target}} - \sigma^2_{\text{init} }}
\stopformula

This procedure gives a handle and tool to lower the resolution of a map to a
specified target resolution. After lowering the resolution, the data were
resampled such that the voxel spacing was 1/4\high{th} of the new resolution using
simple tri-linear interpolation. 


\Subsection{Leveraging the limits using multi-scale image pyramids}

The rapid advancement of the cryo-EM field has resulted in an impressive
increase in the number of high-resolution density maps and corresponding atomic
models. The increase in the level of detail, however, also requires the number
of voxels to represent the data to rise. Consequently, the time required for an
exhaustive search can increase dramatically as fitting algorithms typically use
the FFT for rapid translation correlation scans, which scale with \m{N \log N} where
\m{N} is the number of voxels. 

The actual level of detail present in current high-resolution maps may,
however, be far surpassing the minimal required information to unambiguously
fit a subunit into the density. The superfluous amount of information can be
leveraged by building a multi-scale image pyramid to speed up the search: by
progressively lowering the resolution and subsampling the data, the size of the
density is reduced, which subsequently results in lower computational resources
and time requirements. However, for the image-pyramid concept to work
effectively, the resolution boundaries to perform a successful fitting of a
particular subunit must be established. A natural parameter to investigate is
the size of the subunit, expressed here simply as the number of residues, since
larger chains carry more information and thus require a lower level of detail
to be properly fitted in the density. Once the success rate for correct fitting
of differently sized chains has been established, this information can be used
to extract the required resolution for a specific chain. An image-pyramid can
then be built by creating densities at different scales to fit subunits of
various sizes. An example of such an image pyramid is shown in
\infigure[fig:image-pyramid-example].

\placefigure[top][fig:image-pyramid-example]
{\getbuffer[cap:image-pyramid-example]}
{\externalfigure[fig:image-pyramid-example]}

