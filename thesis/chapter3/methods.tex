\Section{Methods}

\Subsection{Exploring the limits of rigid-body fitting}

To explore the resolution limit of successful rigid-body fitting, we gathered a
benchmark of six ribosome complexes for which high-resolution cryo-EM data were
available, ranging in resolution from 5.5 to 8.9\Angstrom\, and a deposited
atomic structure in the Protein Databank (PDB). Each individual chain was
fitted independently in their respective density map with PowerFit using four
different scoring functions: the local cross-correlation (LCC), the
core-weighted (CW)-LCC, and their Laplace pre-filtered versions, the L-LCC and
L-CW-LCC, respectively, given by

\placeformula[eq:correlation-equation]
\startformula
\text{CC} = \frac{1}{N} 
\stopformula

After each fitting round, the resolution of the cryo-EM density was lowered by
1\Angstrom\ using the following procedure. Assuming that the density is
described by a collection of atoms with a spherical Gaussian density
distribution, where the width of the Gaussian depends on the resolution of the
density, the density at each point in space is given by

\placeformula[eq:density]
\startformula
\rho \left( \vec{r}| R \right) = \sum^N_i A_i \exp\left( -\frac{\left| \vec{r} - \vec{r}_i \right|^2 }{2 \sigma_R^2} \right)
\stopformula

Here the summation is over all \m{N} atoms indexed by \m{i}, where the
amplitude of the density is given by the atom number of the the element
\m{A_i}, and \m{\vec{r}_i} is the position of atom \m{i}, and the spread
\m{\sigma_R} is a function of the resolution \m{R} given by

\placeformula[eq:sigma-to-resolution]
\startformula
\sigma\left( R \right) = \frac{1}{\sqrt{2} \pi} R
\stopformula

This definition of the resolution ensures that the amplitude of the specified
resolution is at \m{1/e} of the maximum. In order to lower the resolution of
the map to the specific resolution, we can simply convolute the density with
another Gaussian kernel thus

\placeformula[eq:target-density]
\startformula
\rho_{\text{target}}\left( \vec{r} \right) = G_k \ast \rho_{\text{init}}
\stopformula

where \m{\rho_{\text{init}}} and \m{\rho_{\text{target}}} are the initial and
target density, respectively, and \m{G_k} is the Gaussian kernel with standard
deviation \m{\sigma_k}, and \m{\ast} is the convolution operator. The
convolution of two Gaussians results in another Gaussian, and is given by

\placeformula[eq:gaussian-convolution]
\startformula
G_1 \ast G_2 = A \exp\left[ - \frac{\left| \vec{r} - \left( \vec{r}_1 +
\vec{r}_2 \right)\right|^2} {2 \left(\sigma_1^2 + \sigma_2^2 \right)}\right]
\stopformula

where \m{G_1} and \m{G_2} are two Gaussian functions with center \m{\vec{r}_1}
and \m{\vec{r}_2} and width \m{\sigma_1} and \m{\sigma_2}, and \m{A} a
normalization constant of no interest here. Thus, \m{\sigma_k} is then simply

\placeformula[eq:sigma-kernel]
\startformula
\sigma_k = \sqrt{\sigma^2_{\text{target}} - \sigma^2_{\text{init} }}
\stopformula



\Subsection{Leveraging the limits using multi-scale image pyramids}

The rapid advancement of the cryo-EM field has resulted in an impressive
increase in the number of high-resolution density maps and corresponding atomic
models. The increase in level of detail, however, also requires the number of
voxels to represent the data to rise. Consequently, the time required for an
exhaustive search can increase dramatically as the fitting algorithms typically
use the Fast Fourier Transform (FFT) for rapid translation correlation scans,
which scale with \m{N \log(N)} with \m{N} the number of voxels. The level of
detail that is present in current high-resolution maps may be far surpassing
the minimal required information to unambiguously fit a subunit in the density.

In addition, current correlation based fitting software acutally take longer to fit bigger subunits, 
This phenomenon can be leveraged by building a multi-scale image pyramid to
speed up the search: by progressively lowering the resolution and subsampling
the data, the size of the density is limited and the computational resources
and time markedly reduced. 

For the image-pyramid concept to work effectively, resolution bounds on fitting
a particular subunits need to be established. A natural parameter to
investigate is the size of the subunit, as bigger chains carry more information
and thus require a lower level of detail to be properly fitted in the density.
Paradoxically, current correlation based fitting software
actually take longer to fit bigger subunits. After determining the success rate for correctly fitting differently
sized chains, this information can be used to extract the average required
resolution for the data for a specific chain. An image-pyramid can then be
setup by creating densities at different scales to fit subunits of various
sizes.

Here we investigated the size influence based on the number of residues of a
subunit, and divided them in 0 -- 50, 50 -- 100, 100 -- 200, 200 -- 300, 300 --
500, and 500+ categories. 


\Subsection{Implementation details}

We extended the PowerFit package by including an additional script to create an
image-pyramid from an initial density together with its resolution. 





