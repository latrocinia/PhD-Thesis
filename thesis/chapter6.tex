\startcomponent chapter6

\environment layout
\product thesis

\setupexternalfigures[factor=broad, directory={chapter6/figures}]
\input chapter6/figures

\define\CA{C\low{\m{\alpha}}}
\define\boldP{{\bf P}}
\define\boldC{{\bf C}}

\startChapter[
title={Inferring interface residues from the accessible interaction space
defined by distance restraints to improve HADDOCKing models},
reference=chapter:inferring-interface-residues,
]


\Section{Introduction}

Uncovering the precise atomic structures of protein complexes is a highly
sought-after enterprise. Experimental techniques that provide atomic
resolution, mainly X-ray crystallography and NMR spectroscopy, have,
unfortunately, so far only revealed a fraction of the whole interactome, the
set of all interacting proteins \cite[Stein2011]. Protein-protein docking aims
to predict the structure of a complex from its individual proteins to close
this knowledge gap \cite[Moreira2010]. However, its success rate using solely
first-principles – the so-called ab initio docking - is generally low
\cite[Huang2015]. Integrating additional information (if reliable) during the
docking process can increase the confidence in the resulting models, especially
when knowledge about the location of the interface is available
\cite[Rodrigues2014].

Mass spectrometry coupled with cross-linking is an upcoming and promising
biochemical method that provides inter-residue distance restraints
\cite[Leitner2010, Rappsilber2011].  Multiple chemistries are being developed,
making the approach more robust and increasing the information content
\cite[Leitner2014].  Several software packages have already been developed for
visualizing crosslinks, and calculating their path length \cite[Kahraman2011,
Holding2013, Kosinski2015].  In the previous chapter we introduced DisVis to
quantify and visualize the information content of distance restraints. However,
interpreting multiple long-range distance restraints between components from a
structural perspective and deducing the interaction surface remains tedious. A
simpler interpretation is gained if interface residues can be deduced from the
data, as these map directly onto the individual chains and offer a
straightforward prediction of the active site. In addition, in the development
of protein-protein inhibitors, mainly the protein interfaces are of importance,
and less so the precise complex's structure \cite[Sable2015]. Such interface
information might be useful in complementing the cross-linking distance
restraints, since the allowed distance ranges for the latter can be relatively
wide (up to 30Å \cite[Merkley2014]).

Inclusion of cross-linker based distance restraints has already been shown to
improve the modeling of both proteins and protein-complexes with the Rosetta
software \cite[Kahraman2013], and to heavily decrease the number of accessible
conformations of a complex (see \inchapter[chapter:disvis]). Our data-driven
docking software HADDOCK is capable of directly incorporating distance
restraints during the docking \cite[Dominguez2003, deVries2007]. Currently,
Mass Spec Studio provides an advanced software platform for integrative
modeling from MS data, such as hydrogen/deuterium exchange and cross-links,
with HADDOCK \cite[Rey2014]. However, no thorough benchmark study has been
performed to measure the impact and effectiveness of incorporating cross-link
based distance restraints in HADDOCK.

Here we introduce a method to infer interface residues when distance restraints
are available in addition to models or structures of the components. The method
is benchmarked on 90 complexes taken from the Protein-Protein Docking Benchmark
4.0 (PPDB4.0) \cite[Hwang2010] for 3, 5 and 7 cross-links, respectively, with
an upper distance restraint of 30Å, comparable to the information content that
is provided by disuccinimidyl suberate (DSS) and and
bis-sulfosuccinimidyl-suberate (BS3) cross-links \cite[Merkley2014]. Finally,
we show how this can be combined with HADDOCK to complement unambiguous
distance restraints by derived interface information, benchmarking it on 24
cases of the PPDB4.0.


\Section{Methods}


\Subsection{Inferring interface residues from distance restraints}

In the previous Chapter, we have introduced the concept of the accessible
interaction space, the set of all possible complexes that are consistent with a
given number of distance restraints. Indeed, the presence of distance
restraints between two interacting macromolecular biomolecules can
significantly reduce their accessible interaction space. To infer residues that
are likely to be at the interface, we assume that these residues are often
found to be interacting in the interaction space consistent with the
restraints. Important residues may be determined by performing a
full-exhaustive 6 dimensional search of the three translational and three
rotational degrees of freedom and counting the number of interactions that each
solvent accessible residue forms in complexes consistent with a given number of
restraints.  We define two residues to be interacting when their \CA\ -- \CA\
distance is smaller than 10Å. We only consider the \CA-atoms of solvent
accessible residues of both chains to make the computations more tractable as
the number of possible interactions scales with \m{A^2} with \m{A} the number
of atoms involved. The average number of interactions per complex (AIC) that a
residue \m{i} forms is given by 

\placeformula[eq:count-interactions] \startformula \overline{N}_i =
\frac{\sum_R^{\bf P} w_R \sum_C^{\boldC_R} I_{C}} {\sum_R^\boldP w_R \boldC_R}
\stopformula

where the first summation is over all rotations \m{\boldP} indexed by \m{R};
\m{w_R} is a weight factor to correctly average over rotation space; the second
summation is over all complexes \m{\boldC_R} that are formed within a
translational scan indexed by \m{C}; and \m{I_{C}} is the number of
interactions that are formed by residue \m{i} in each sampled complex \m{C}.

This approach has been implemented in DisVis (see \inchapter[chapter:disvis],
\from[url:disvis]), which requires for the interaction
analysis an extra input file containing the solvent accessible residue numbers
for the fixed and scanning chain. As a result, DisVis outputs a file containing
the number of interactions that are formed by each residue for complexes
consistent with at least \m{N} restraints.


\Subsection{Benchmarking interface residue extraction}

We benchmarked our approach on 90 complexes taken from the 
PPDB4.0, of which 58 were classified as Easy, 14 as Medium, and 18 as Difficult.
Virtual cross-links were calculated using a local version of the XWalk software
\cite[Kahraman2011] on the bound complex. The virtual crosslinks were chosen
such that the solvent accessible surface (SAS) distance was shorter or equal
than 34Å \cite[Kahraman2013], and the Euclidean distance smaller or equal than
30Å \cite[Merkley2014] and the cross-linked residues should be present in both
the bound and unbound proteins. The cross-links were randomly picked from the
list of all virtual crosslinks using the SAS-distance dependent probability
distribution as was used by \citeauthor{Kahraman2013} to mimic experimental
cross-link data: 0 -- 10Å 9\%; 10 -- 15Å 18\%; 15 -- 20Å 34\%; 20 -- 25Å 22\%;
and 25 -- 34Å 16\%.

The solvent accessible residues were determined by running {\it naccess}
\cite[Hubbard1992] on the two unbound proteins. Residues that had a relative
solvent accessibility of the main or side chain of 50\% or higher were used as
surface residues. DisVis runs were performed for each complex using 3, 5, and 7
random restraints, respectively, with a 5.27° rotational sampling, and default
values for the voxel spacing (1Å), maximum clashing volume (200Å3) and minimum
interaction volume (300Å3). The AIC was only calculated from the complexes
consistent with all restraints. 

Correct interface residues were taken from the experimental structure of the
complex using the above definition of interaction, under the restriction that
the residue was also present in the unbound proteins. To analyze the predictive
capabilities the precision P and recall R were calculated as
\placeformula[eq:precision]
 
\startformula \text{P} = \frac{\text{TP}}{\text{TP} + \text{FP}} \stopformula

\placeformula[eq:recall] \startformula \text{R} = \frac{\text{TP}}{\text{TP} +
\text{FN}} \stopformula

where TP, FP and FN stand for True Positive, False Positive, and False
Negative, respectively.


\Subsection{HADDOCKing with virtual cross-links}

To determine whether the inclusion of DisVis-determined interface residues
aids the docking process of HADDOCK, we benchmarked HADDOCK using 24 complexes
of the PPDB4.0, of which 16 were the same as those used by
\citeauthor{Kahraman2013}. The remaining 8 were randomly picked Easy
complexes, as the previous 16 already consisted of 7 Medium and 9 Difficult
cases. HADDOCK was benchmarked with 4 different protocols: 

\startitemize

\item using the restraints directly as unambiguous distance restraints (unambig) with
a minimal and maximal Euclidean length of 0 and 30Å, respectively; 
\item using the unambiguous restraints in combination with center-of-mass restraints
\cite[Karaca2013]; 
\item using solely DisVis-based ambiguous interaction restraints (AIRs); 
\item and using a combination of the unambig restraints and DisVis-based AIRs.  

\stopitemize

Each protocol was performed with 3, 5 and 7 generated virtual cross-link
restraints, respectively, as described in the previous section. The
DisVis-based AIRs were determined as follows: a DisVis run was performed as
described above using the unbound structures of the complex together with the
virtual cross-links; active residues were chosen such that their AIC consistent
with all restraints had to be larger than 1; passive residues will be chosen
with a to be determined AIC cutoff based following the method described in the
previous section. Per HADDOCK run 1000 it0-structures were written to file,
using 5 trials per file and 180° sampling, resulting in 10000 sampled solutions
(\m{1000 \times 5 \times 2}), of which the 200 best scoring solutions were
subjected to the semi-flexible refinement (it1 and itw) (default settings of
the server). The solutions were analyzed by calculating the ligand-RMSD
(l-RMSD) against the native complex, by first optimally fitting the receptor
chain and afterwards calculating the RMSD of backbone atoms of the ligand chain
using ProFitV3.1 \cite[Martin2009]. Models with an l-RMSD lower than 10Å were
considered acceptable.


\Section{Results and discussion}


\Subsection{Inferring interface residues from distance restraints}

The analysis of the DisVis benchmark is shown in
\infigure[fig:precision-recall] for all complexes and for each difficulty
category, by plotting the precision and recall against the AIC cutoff of a
residue. This shows for example that for all complexes, using 7 cross-links,
for residues that have a AIC ≥ 1.0 the precision is approximately 60\%, i.e.
60\% of all residues with a AIC ≥ 1.0 are true interface residues; the recall
at 1.0 AIC is around 40\%, meaning that 40\% of all true interface residues is
still retained in the set of residues satisfying the AIC cutoff condition.

\placefigure[page][fig:precision-recall] {\getbuffer[cap:precision-recall]}
{\externalfigure[fig:precision-recall]}

As expected, the precision increases with increasing AIC, while the recall rate
drops steadily. Also, both precision and recall rise with the number of
available cross-links, reflecting the higher information content of the
distance restraints. On average for all complexes, the precision starts at 30\%
regardless of the number of available restraints, and rises with increasing
cutoff to 50, 70 and 80\% for 3, 5, and 7 restraints, respectively.  For Easy
complexes this increases even to 60, and 85 and 90\%, while for Medium and
Difficult complexes the precision is significantly lower, dropping down to 70\%
for Difficult complexes in the presence of 7 restraints. We attribute the noisy
behavior of the precision, especially for Medium complexes, to the smaller
number of complexes sampled compared to the number of Easy complexes (14 versus
58).

The recall percentage averaged over all complexes starts around 90\%, a consequence
of how surface residues were chosen: because of small conformational changes
between the bound and unbound chains, residues that are regarded as solvent
accessible in the bound form might not be accessible in the unbound form. This
also explains the lower starting recall rate of Medium and Difficult complexes,
with the latter being lower than 80\%, as these more challenging complexes
typically exhibit greater conformational change between their bound and unbound
form by definition. Not surprisingly, the recall rate decreases steadily with
increasing cutoff, as fewer residues will be satisfying the AIC cutoff
condition. In contrast to the precision, the recall rate does increase
significantly with the inclusion of more restraints for the more challenging
categories (Medium + Difficult). 


\Subsection{HADDOCKing with virtual cross-links}

The HADDOCK benchmark results for the 8 Easy complexes, and the 16 Medium and
Difficult complexes are displayed in \infigure[fig:benchmark-easy] and
\in[fig:benchmark-rosetta], respectively, in terms of the structure with the
lowest l-RMSD after the water-refinement stage, irrespective of its rank. For
the DisVis-based AIRs the cutoff AIC for active residues was set to 1,
corresponding to a precision of 40 to 60\% and a recall of 20 to 40\% for 3 and
7 cross-links; the AIC cutoff for passive residues was set to 0.1, as the
precision increase is steeper approximately until that point, while still
keeping a reasonable recall rate of approximately 80\%. For the Easy complexes
the unambiguous restraints approach was successful in 50, 37.5 and 75\% of the
cases when using 3, 5 and 7 cross-links.  Adding the center-of-mass restraint
this changed to 62.5, 50, and 62.5\%.  Using the DisVis-based AIRs resulted in
a success rate of 25, 62.5, and 75\%, and combined with the unambiguous
restraints this increased to 50, 75 and 75\% success rate.  Interestingly,
increasing the number of cross-links does not necessarily result in better
structures when using only the unambiguous restraints: the success rate with 5
cross-links is markedly lower than using 3. Also, strangely, the unambiguous
approach with center-of-mass restraint is the only method for which no
acceptable solutions are generated for the 1QA9 complex, even with 7 restraints
included. The results of the DisVis-based approaches, however, are improving
with increasing numbers of cross-links.

\placefigure[top][fig:benchmark-easy] {\getbuffer[cap:benchmark-easy]}
{\externalfigure[fig:benchmark-easy]}

\placefigure[top][fig:benchmark-rosetta] {\getbuffer[cap:benchmark-rosetta]}
{\externalfigure[fig:benchmark-rosetta]}

In the 16 Medium and Difficult complexes the unambiguous approach is successful
in 18.75, 62.5 and 50\% of the cases for 3, 5 and 7 cross-links, respectively;
with inclusion of the center-of-mass restraint this becomes 37.5, 56.25, and
75\%. The DisVis-based AIRs result in 12.5, 50 and 50\% successful cases;
including the unambiguous restraints this increases to 12.5, 56.25 and 62.5\%.
As with the Easy complexes, the success rate of the unambiguous restraints
shows no steady improvement with an increased number of cross-links.
Furthermore, the combination of DisVis-based AIRs and unambiguous restraints is
superior in general to using only DisVis-based AIRs. 

The HADDOCK results using 7 cross-links are compared against Rosetta in
\infigure[fig:benchmark-rosetta]{C}. Rosetta was successful in 68.75\% of the
cases, a slightly higher percentage than HADDOCK using DisVis-based AIRs with
unambiguous restraints, but lower than when using unambiguous with
center-of-mass restraints. However, in general the results are comparable. 

Taking all 24 complexes together we conclude that using only the unambiguous
restraints results in a success rate (again defined as generating at least one
native-like model in the set of 200 refined models) of 29, 54 and 58\% for 3, 5
and 7 cross-links, respectively; using ambiguous restraints with center-of-mass
restraints this increases to 46, 54 and 71\%. Using solely DisVis-based AIRs
the success rates are respectively 17, 54 and 58\%; and DisVis-based AIRs
combined with unambiguous restraints 25, 63 and 67\%.  Based on these results
we conclude that if only 3 cross-links are available, the best protocol to use
is to combine unambiguous restraints with center-of-mass restraints. If 5 or
more cross-links are available it is best to either again combine the
unambiguous restraints with the center-of-mass restraints, or combine them with
DisVis-based AIRs. 

\Section{Conclusions}

In this Chapter we have introduced a method to infer interface residues by
enumerating all interactions a residue forms in the interaction space
consistent with all restraints and normalizing it against the number of
accessible complexes. It was shown that residues with a higher AIC are more
likely to be interface residues with precision reaching almost 90\% for rigid
complexes. This information can be used to guide future mutagenesis studies and
map out the interface, irrespective of the quality of the generated models of
the complex. In addition, we benchmarked several protocols within HADDOCK that
incorporated cross-link based distance restraints. Based on the analysis, using
the cross-links directly as unambiguous restraints is sub-optimal, and instead
should be complemented with either center-of-mass restraints or DisVis-based
AIRs. Furthermore, we have shown that HADDOCK and Rosetta are very comparable
in their docking performance when including the distance restraints. However,
it should be noted that we only investigated the quality of the best model.
Further analysis should also address the ranking of generated models, as using
various types of restraints might significantly affect the scoring functions’
ability to  identify near-native models.

\stopcomponent
