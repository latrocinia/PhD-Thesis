\startcomponent chapter6

\environment layout
\product thesis

\setupexternalfigures[factor=broad, directory={chapter6/figures}]
\input chapter6/figures

\define\CA{C\low{\m{\alpha}}}
\define\boldP{{\bf P}}
\define\boldC{{\bf C}}

\Chapter{Inferring interface residues from distance restraints for
high-resolution HADDOCKing}


\Section{Introduction}

Uncovering the precise atomic structures of protein complexes is a highly
soughtafter enterprise. Experimental techniques that provide atomic resolution
, mainly X-ray crystallography and NMR spectroscopy, have, unfortunately, so
far only revealed a fraction of the whole interactome, the set of all
interacting proteins \cite[Stein2011]. Protein-protein docking aims to predict
the structure of the complex from their individual proteins \cite[Moreira2010].
However, the success rate using solely first-principles is generally low
\cite[Huang2015]. Integrating additional information during the docking
increases the confidence in the resulting models, especially with knowledge on
the location of the interface \cite[Rodrigues2014]. Knowledge of the interface
is also of prime importance in the rational development of protein-protein
inhibitors \cite[Sable2015]. 

Mass spectrometry/cross-linking is an upcoming and promising biochemical method
that provides inter-residue distance restraints \cite[Rappsilber2011]. 
Multiple chemistries are being developed, making the approach more robust and
increasing the information content \cite[Leitner2014]. Inclusion of cross-linker
based distance restraints has already shown to improve the modeling of
both proteins and protein-complexes with the Rosetta software
\cite[Kahraman2013], and heavily decreases the number of accessible complexes
(see \inchapter[disvis]). Our data-driven docking software HADDOCK is also
capable of directly incorporating distance restraints during the docking
\cite[Domingues2003, deVries2007], though no benchmark study has been performed
to measure the impact and effectiveness.

In \inchapter[disvis] we developed a method to visualize the accessible
interaction space to aid the structural biologist. Here we will develop the
approach further by introducing a method to infer interface residues when
distance restraints are available in addition to high-resolution atomic models
of the chains. The method is benchmarked on 91 complexes taken from the
protein-protein docking benchmark 4.0 (PPDB4.0) \cite[Hwang2010] for 3, 5 and 7
restraints with an upper distance of 30\Angstrom, comparable to the information
content that is provided by DSS/BS3 mass spectrometry/crosslink data
\cite[Merkley2014]. Finally, we show how the results can be combined with
HADDOCK to complement the docking runs with regular unambiguous distance
restraints by additionally benchmarking it using 24 cases of the PPDB4.0.


\Section{Methods}


\Subsection{Inferring interface residues from distance restraints}

In the previous Chapter, we have introduced the concept of the accessible
interaction space, the set of all possible complexes that are consistent with a
certain number of distance restraints. Indeed, the presence of distance
restraints between two interacting macromolecular biomolecules can
significantly reduce the accessible interaction space. To infer residues that
are likely to be at the interface, we assume that these residues are often
found to be interacting in the whole interaction space. By performing a
full-exhaustive 6 dimensional search of the three translational and three
rotational degrees of freedom and counting the number of interactions that each
solvent accessible residue forms in complexes consistent with a certain number
of restraints, important residues can be determined. We define two residues to
be interacting when the \CA\ --\ \CA\ distance is smaller than 10\Angstrom. We
only consider the \CA-atoms of solvent accessible residues of both chains to
make the computations more tractable as the number of possible interactions
scales with \m{A^2} with \m{A} the number of atoms involved. The average number
of interactions per complex that a residue \m{i} forms is given by

\placeformula[eq:count-interactions] 
\startformula
\overline{N}_i = \frac{\sum_R^{\bf P} w_R \sum_C^{\boldC_R}
I_{C}} {\sum_R^\boldP w_R \boldC_R} 
\stopformula

where the first summation is over all rotations \m{\boldP} indexed by \m{R};
\m{w_R} is a weight factor to unbias certain rotations/orientations; the second
summation is over all complexes \m{\boldC_R} that are formed within a
translational scan indexed by \m{C}; and \m{I_{C}} is the number of
interactions that are formed by residue \m{i} in each sampled complex.

This approach has been implemented in DisVis, which requires for the
interaction analysis an extra input file containing the residue numbers for the
fixed and scanning chain that are solvent accessible. DisVis then outputs a
file containing the number of interactions that are formed by each residue for
complexes consistent with at least \m{N} restraints.


\Subsection{Benchmarking interface residue extraction}

We benchmarked our approach on 91 complexes taken from the protein-protein
docking benchmark 4.0. The complexes were taken such that at least 7 virtual
cross-links were available, determined by the XWalk software \cite[Kahraman2011]. The virtual
crosslinks were picked such that the solvent accessible surface (SAS) distance
was shorter or equal than 34\Angstrom\ \cite[Kahraman2013], and the Euclidean
distance smaller or equal than 30\Angstrom \cite[Merkley2014]. The cross-links
were randomly picked from the list using the probability distribution as was
used by \cite[authoryear][Kahraman2013] to mimic cross-link data: 0 --
10\Angstrom\ 9\%; 10 -- 15\Angstrom\ 18\%; 15 -- 20\Angstrom\ 34\%; 20 --
25\Angstrom\ 22\%; and 25 -- 34\Angstrom\ 16\%.

The solvent accessible residues were determined by running {\it naccess} \cite[Hubbard1992] on the two
unbound proteins. Residues that had a relative solvent accessiblity of the main
or side chain of 50 were used as surface residues. DisVis runs were performed
for each complex using 3, 5, and 7 random restraints with a 5.27\Deg\ 
rotational sampling, and default values for the voxelspacing (1\Angstrom), and
maximum clashing volume (200\Angstrom\high3) and minimum interaction volume
(300\Angstrom\high3). The average interactions per complex were calculated for
complexes consistent with all restraints. 

Correct interface residues were taken from the complex structure using the
above definition of interaction, under the restriction that the residue was
also present in the unbound proteins. To analyse the predictive capabilities the
precision and recall were calculated, given by

\placeformula[eq:precision] 
\startformula \text{P} = \frac{\text{TP}}{\text{TP}
+ \text{FP}} 
\stopformula

\placeformula[eq:recall] 
\startformula \text{R} = \frac{\text{TP}}{\text{TP} +
\text{FN}} 
\stopformula

where TP, FP and FN stand for True Positive, False Positive, and False
Negative, respectively.


\Subsection{HADDOCKing with distance-restraints}






\Section{Results}


\Subsection{Residues with many interactions are more likely to be at the
interface}

\placefigure[top][fig:precision-recall] {\getbuffer[cap:precision-recall]}
{\externalfigure[fig:precision-recall]}


\Subsection{HADDOCKing with distance restraints}


\Section{Conclusions}

\stopcomponent
