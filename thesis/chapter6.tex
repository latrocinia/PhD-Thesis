\startcomponent chapter6

\environment layout
\product thesis

\setupexternalfigures[factor=broad, directory={chapter6/figures}]
\input chapter6/figures

\define\CA{C\low{\m{\alpha}}}
\define\boldP{{\bf P}}
\define\boldC{{\bf C}}

\Chapter[chapter:inferring-interface-residues]
{Inferring interface residues from distance restraints for high-resolution
HADDOCKing}

\emph{G.C.P. van Zundert and A.M.J.J. Bonvin. To be submitted.}

\Section{Introduction}

Uncovering the precise atomic structures of protein complexes is a highly
soughtafter enterprise. Experimental techniques that provide atomic resolution,
mainly X-ray crystallography and NMR spectroscopy, have, unfortunately, so far
only revealed a fraction of the whole interactome, the set of all interacting
proteins \cite[Stein2011]. Protein-protein docking aims to predict the
structure of the complex from their individual proteins to close the knowledge
gap \cite[Moreira2010]. However, the success rate using solely first-principles
is generally low \cite[Huang2015]. Integrating additional information during
the docking increases the confidence in the resulting models, especially with
knowledge on the location of the interface \cite[Rodrigues2014].

Mass spectrometry coupled with cross-linking is an upcoming and promising
biochemical method that provides inter-residue distance restraints
\cite[Leitner2010, Rappsilber2011].  Multiple chemistries are being developed,
making the approach more robust and increasing the information content
\cite[Leitner2014].  Several software packages have already been developed for
visualizing crosslinks, and calculating their path length \cite[Kahraman2011,
Holding2013, Kosinski2015].  In the previous chapter we introduced DisVis to
quantify and visualize the information content of distance restraints. However,
interpreting multiple long-range distance restraints between components from a
structural perspective and deducing the interaction surface remains tedious. A
simpler interpretation is gained if interface residues can be deduced from the
data, as these map directly onto the individual chains and offer a
straightforward prediction of the active site. In addition, in the development
of protein-protein inhibitors mainly the protein interfaces are of importance,
and less so the precise complex's structure \cite[Sable2015]. 

Inclusion of cross-linker based distance restraints has already shown to
improve the modeling of both proteins and protein-complexes with the Rosetta
software \cite[Kahraman2013], and heavily decreases the number of accessible
complexes (see \inchapter[chapter:disvis]). Our data-driven docking software
HADDOCK is also capable of directly incorporating distance restraints during
the docking \cite[Dominguez2003, deVries2007], though no thorough benchmark
study has been performed to measure the impact and effectiveness.

Here we will introduce a method to infer interface residues when distance
restraints are available in addition to models of the components. The method is
benchmarked on 90 complexes taken from the protein-protein docking benchmark
4.0 (PPDB4.0) \cite[Hwang2010] for 3, 5 and 7 restraints with an upper distance
of 30\Angstrom, comparable to the information content that is provided by
DSS/BS3 mass spectrometry/crosslink data \cite[Merkley2014]. Finally, we show
how the results can be combined with HADDOCK to complement the docking runs
with regular unambiguous distance restraints by additionally benchmarking it
on 24 cases of the PPDB4.0.


\Section{Methods}


\Subsection{Inferring interface residues from distance restraints}

In the previous Chapter, we have introduced the concept of the accessible
interaction space, the set of all possible complexes that are consistent with a
certain number of distance restraints. Indeed, the presence of distance
restraints between two interacting macromolecular biomolecules can
significantly reduce the accessible interaction space. To infer residues that
are likely to be at the interface, we assume that these residues are often
found to be interacting in the interaction space consistent with the
restraints. Important residues may be determined by performing a
full-exhaustive 6 dimensional search of the three translational and three
rotational degrees of freedom and counting the number of interactions that each
solvent accessible residue forms in complexes consistent with a certain number
of restraints.  We define two residues to be interacting when the \CA\ -- \CA\
distance is smaller than 10\Angstrom. We only consider the \CA-atoms of solvent
accessible residues of both chains to make the computations more tractable as
the number of possible interactions scales with \m{A^2} with \m{A} the number
of atoms involved. The average number of interactions per complex (AIC) that a
residue \m{i} forms is given by

\placeformula[eq:count-interactions] 
\startformula
\overline{N}_i = \frac{\sum_R^{\bf P} w_R \sum_C^{\boldC_R}
I_{C}} {\sum_R^\boldP w_R \boldC_R} 
\stopformula

where the first summation is over all rotations \m{\boldP} indexed by \m{R};
\m{w_R} is a weight factor to correctly average over rotation space; the second
summation is over all complexes \m{\boldC_R} that are formed within a
translational scan indexed by \m{C}; and \m{I_{C}} is the number of
interactions that are formed by residue \m{i} in each sampled complex \m{C}.

This approach has been implemented in DisVis, which requires for the
interaction analysis an extra input file containing the residue numbers for the
fixed and scanning chain that are solvent accessible. DisVis then outputs a
file containing the number of interactions that are formed by each residue for
complexes consistent with at least \m{N} restraints.


\Subsection{Benchmarking interface residue extraction}

We benchmarked our approach on 90 complexes taken from the protein-protein
PPD4.0, of which 58 were Easy, 14 Medium, and 18 Difficult. Virtual cross-links
were calculated using a local version of the XWalk software \cite[Kahraman2011]
on the bound complex. The virtual crosslinks were chosen such that the solvent
accessible surface (SAS) distance was shorter or equal than 34\Angstrom\
\cite[Kahraman2013], and the Euclidean distance smaller or equal than
30\Angstrom\ \cite[Merkley2014] and the cross-linked residues should be present
in both the bound and unbound proteins. The cross-links were randomly picked
from the list of all virtual crosslinks using the SAS-distance dependent
probability distribution as was used by \citeauthor{Kahraman2013} to mimic
experimental cross-link data: 0 -- 10\Angstrom\ 9\percent; 10 -- 15\Angstrom\
18\percent; 15 -- 20\Angstrom\ 34\percent; 20 -- 25\Angstrom\ 22\percent; and
25 -- 34\Angstrom\ 16\percent.

The solvent accessible residues were determined by running {\it naccess}
\cite[Hubbard1992] on the two unbound proteins. Residues that had a relative
solvent accessibility of the main or side chain of 50 were used as surface
residues. DisVis runs were performed for each complex using 3, 5, and 7 random
restraints with a 5.27\Deg\ rotational sampling, and default values for the
voxel spacing (1\Angstrom), and maximum clashing volume (200\Angstrom\high3) and
minimum interaction volume (300\Angstrom\high3). The AIC was calculated for
complexes consistent with all restraints. 

Correct interface residues were taken from the complex structure using the
above definition of interaction, under the restriction that the residue was
also present in the unbound proteins. To analyse the predictive capabilities
the precision and recall were calculated, given by

\placeformula[eq:precision] 
\startformula 
\text{P} = \frac{\text{TP}}{\text{TP} + \text{FP}} 
\stopformula

\placeformula[eq:recall] 
\startformula 
\text{R} = \frac{\text{TP}}{\text{TP} + \text{FN}} 
\stopformula

where TP, FP and FN stand for True Positive, False Positive, and False
Negative, respectively.


\Subsection{HADDOCKing with virtual cross-links}

To determine whether the inclusion of DisVis-determined interface
residues aids the docking process of HADDOCK, we additionally benchmarked
HADDOCK using 24 complexes of the PPDB4.0 of which 16 were the same as used by
\citeauthor{Kahraman2013}. The remaining 8 were randomly picked Easy
complexes, as the 16 consisted of 7 Medium and 9 Difficult cases. HADDOCK was
benchmarked with 4 different procedures: using the restraints directly as
unambiguous distance restraints (unambig) with a minimal and maximal Euclidean
length of 0 and 30\Angstrom, respectively; using the unambiguous restraints in
combination with center-of-mass restraints \cite[Karaca2013]; using solely
DisVis-based ambiguous interaction restraints (AIRs); and using a combination
of the unambig restraints and DisVis-based AIRs. Each procedure was performed
again with 3, 5 and 7 generated virtual cross-link restraints, as described in
the prvevious section. The DisVis-based AIRs were determined as follows: a
DisVis run was performed as described above using the unbound structures of the
complex together with the virtual cross-links; active residues were chosen such
that its AIC consistent with all
restraints had to be bigger than 1; passive residues were chosen with a to be
determined AIC cutoff based on the analysis in the previous
section. Per HADDOCK run 1000 it0-structures were written to file, using 5
trials per file and 180\Deg\ sampling, resulting in 10000 sampled solutions
(\m{1000 \times 5 \times 2}), of which the 200 best scoring solutions were
refined in it1 and itw. The solutions were analyzed by calculating the
ligand-RMSD (l-RMSD) against the native complex, by first optimally fitting the
receptor chain and afterwards calculating the RMSD of the ligand chain using
ProFitV3.1 \cite[Martin2009]. Models with an l-RMSD lower than 10\Angstrom\
were considered acceptable.


\Section{Results and discussion}


\Subsection{Inferring interface residues from distance restraints}

The analysis of the DisVis benchmark is shown in
\infigure[fig:precision-recall] for all complexes and for each difficulty
category, by plotting the precision and recall against the AIC cutoff of a
residue. This shows for example that for all complexes when using 7
cross-links, on average, for residues that have a higher AIC of 1.0 the
precision is approximately 60\percent, i.e. 60\percent\ of all residues with a
higher AIC of 1.0 are true interface residues; the recall at 1.0 AIC is around
40\percent, meaning that 40\percent\ of all true interface residues is still
retained in the set of residues satisfying the AIC cutoff condition.

\placefigure[top][fig:precision-recall] {\getbuffer[cap:precision-recall]}
{\externalfigure[fig:precision-recall]}

As expected, the precision increases with increasing AIC, while the recall rate
drops steadily. Also, both precision and recall rise with the number of available
cross-links, reflecting the higher information content of the distance
restraints. On average for all complexes the precision starts at 30\percent\ regardless of
the number of available restraints, and rises to 50, 70 and 80\percent\ for 3,
5, and 7 restraints, respectively. For Easy complexes this increases even to
60, and 85 and 90\percent, while for Medium and Difficult complexes the
precision rate is significantly lower, dropping down to 70\percent\ for
Difficult complexes in the presence of 7 restraints. We attribute the noisy
behavior of the precision rate, especially for Medium complexes, to the
smaller number of complexes sampled.

The starting recall percentage averaged over all complexes starts around
90\percent, a consequence of how surface residues were chosen: because of small
conformational changes between the bound and unbound chains, residues that are
regarded as solvent accessible in the bound form might not be accessible in the
unbound form. This also explains the lower starting recall rate of Medium and
Difficult complexes, with the latter being lower than 80\percent, as the more
difficult complexes typically exhibit greater conformational change between
their bound and unbound form by definition. Not surprisingly, the recall rate
decreases steadily with increasing AIC, as less residues will be satisfying the
AIC cutoff condition. However, the recall rate does increase significantly with
the inclusion of more restraints for all difficulty categories. 


\Subsection{HADDOCKing with virtual cross-links}

The HADDOCK benchmark results for the 8 Easy complexes, and the 16 Medium and
Hard complexes are displayed in \infigure[fig:benchmark-easy] and
\in[fig:benchmark-rosetta], respectively, in terms of the structure with the
lowest l-RMSD after the water-refinement stage. For the DisVis-based AIRs the
cutoff AIC for passive residues was set to 0.1, as the precision increase is
steeper approximately until that point, while still keeping a reasonable
recall rate. For the Easy complexes the unambiguous restraints approach was
successful in 50, 37.5 and 75\percent\ of the cases when using 3, 5 and 7
cross-links.  Adding the center-of-mass restraints this changed to 62.5, 50,
and 62.5\percent.  Using the DisVis-based AIRs resulted in a success rate of
25, 62.5, and 75\percent, and combined with the unambiguous restraints this was
50, 75 and 75\percent\ success rate.  Interestingly, increasing the number of
cross-links does not necessarily result in better structures using solely the
unambiguous restraints, as the success rate with 5 cross-links is markedly
lower than using 3. Also, the unambiguous approach with included center-of-mass
restraints is the only method for which no acceptable solutions are generated
for the 1QA9 complex, even with 7 restraints included. The results of the
DisVis-based approaches, however, are improving with more cross-links
available.

\placefigure[top][fig:benchmark-easy]
{\getbuffer[cap:benchmark-easy]}
{\externalfigure[fig:benchmark-easy]}

\placefigure[top][fig:benchmark-rosetta]
{\getbuffer[cap:benchmark-rosetta]}
{\externalfigure[fig:benchmark-rosetta]}

In the 16 Medium and Difficult complexes the unambiguous approach is successful
in 18.75, 62.5 and 50\percent\ of the cases for 3, 5 and 7 cross-links,
respectively; with inclusion of the center-of-mass restraints this is 37.5,
56.25, and 75\percent.  The DisVis-based AIRs result in 12.5, 50 and
50\percent\ successful cases; including the unambiguous restraints this
increases to 12.5, 56.25 and 62.5\percent. As with the Easy complexes, the
success rate of the unambiguous restraints shows no steady improvement with an
increased number of cross-links. Furthermore, the combination of DisVis-based
AIRs and unambiguous restraints is superior in general to using only
DisVis-based AIRs. 

The HADDOCK results when using 7 cross-links are compared against Rosetta in
\infigure[fig:benchmark-rosetta]{C}. Rosetta was successful in 68.75\percent\
of the cases, slightly higher than HADDOCK using DisVis-based AIRs with
unambiguous restraints, but lower than when using unambiguous with
center-of-mass restraints. However, in general the results are comparable.

Taking all 24 complexes together we conclude that using the solely the
unambiguous restraints results in a success rate of 29, 54 and 58\percent;
using ambiguous restraints with center-of-mass restraints this is 46, 54 and
71\percent; using solely DisVis-based AIRs 17, 54 and 58\percent; and
DisVis-based AIRs combined with unambiguous restraints 25, 63 and 67\percent.
Based on these results we conclude that if only 3 cross-links are available,
the best protocol to use is to combine unambiguous restraints with
center-of-mass restraints. If 5 or more cross-links are available it is best to
either again combine the unambiguous restraints with the center-of-mass
restraints, or combine them with DisVis-based AIRs.


\Section{Conclusions}

In this Chapter we have introduced a method to infer interface residues by
enumerating all interactions a residue forms in the interaction space
consistent with all restraints and normalizing it against the number of
accessible complexes. It was shown that residues with a higher AIC are more
likely to be interface residues with precision reaching almost 90\percent\ for
rigid complexes. In addition, we benchmarked several protocols within HADDOCK
that incorporated cross-link based distance restraints. Based on the analysis,
using the cross-links directly as unambiguous restraints is sub-optimal, and
instead should be complemented with either center-of-mass restraints or
DisVis-based AIRs. Furthermore, we have shown that HADDOCK and Rosetta are very
comparable in their docking performance when including the distance restraints.

\stopcomponent
