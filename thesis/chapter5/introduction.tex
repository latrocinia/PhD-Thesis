Structural characterization of protein complexes is of paramount importance for
a fundamental understanding of cellular processes, and with major applications
in rational drug-design. As the quantity of experimentally determined complexes
is only a fraction of their total predicted number, complementary computational
techniques have been developed for predicting the structure of complexes from
their components \cite[Rodrigues2014, Petrey2014]. Additional low-resolution
information in the form of distance restraints can significantly benefit the
modeling, with a variety of experimental methods providing such information,
such as chemical cross-links detected by mass spectrometry
\cite[Rappsilber2011], and distance measurements from electron paramagnetic
resonance (EPR) and FRET \cite[Kalinin2012].

When two biomolecules are known to interact and no high-resolution model is
available, the structure of the complex can naively be any one state where the
molecules are in contact. We define the accessible interaction space of the
complex as the set of all these states. If a distance restraint is imposed on
the complex, the accessible interaction space reduces, depending on the
information content of the restraint. The interaction space is further reduced
if multiple restraints are included. So far, however, no computational method
has been reported that quantifies this reduction or allows to visualize this
accessible interaction space. 

To aid in this task, we have developed DisVis, a GPU-accelerated Python
software package and command line tool (\disvis) for quantifying and visualizing
the accessible interaction space of distance-restrained binary complexes.
Disvis takes as input two atomic structures and a file with distance
restraints, and outputs the sum of complexes complying with a given number of
restraints together with a density showing the maximum number of consistent
restraints at every position in space. This indicates whether all data are
consistent and can be combined without violations, and allows identification of
false positives, quantification of the information content of the restraints
and visualization of interesting regions in the interaction space. The method
is generic and can easily be incorporated into existing Fast Fourier Transform
(FFT)-accelerated docking programs as a distance-dependent energy function,
allowing the ‘marriage made in heaven’ of direct sampling and scoring of
FFT-generated docking poses \cite[Vajda2013] at a small computational cost.

