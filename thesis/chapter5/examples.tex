To illustrate the capabilities of \disvis, we applied it on two systems, using
MS cross-links data (see the Supplementary Information for details). A fine
rotational search (5.27°, 53256 orientations) was performed using default
values. First we investigated the accessible interaction space of two chains of
the RNA polymerase II complex of S. cerevisiae (1WCM, chain A and E) for which
6 BS3 cross-links were available \cite[Chen2010, Kahraman2013].
The allowed distance was set between 0 and 30Å (Cβ – Cβ) for every restraint.
Two false-positive restraints were added with a distance in the crystal
structure of 35.7 (FP1) and 42.2Å (FP2) to test whether these violating
restraintsthey could be identified. Applying disvis shows that none of the
18.9×109 complexes sampled are consistent with all 8 restraints, though a small
number are conforming to 7 cross-links (9716 complexes). For the latter, only
restraint FP2 is violated. The accessible interaction space consistent with at
least 6 restraints is less than 0.03\% of the full interaction space (Figure 1).
The density clearly indicates the position of the E-chain. Interestingly, both
false-positive restraints are violated in 100\% of the complexes consistent with
at least 6 restraints; in contrast, the highest violation percentage of a
correct cross-link is only 0.1\%. Thus, a high-violation percentage is an
indication of a false-positive restraint. 

Secondly, we applied disvis on two proteins of the 26S proteasome of S. pombe,
PRE5 (O14250) and PUP2 (Q9UT97), with 7 cross-links available (Leitner et al.,
2014). The acceptable distances for the adipic acid dihydrazide (ADH) and
zero-length (ZL) cross-links were set to 23 and 26Å (Cα - Cα), respectively, as
95\% of distances found in a benchmark were shorter \cite[Leitner2014]. The
PRE5-PUP2 complex is significantly smaller than the previous example with the
full interaction space consisting of 6.9×109 complexes. Still, the accessible
interaction space consistent with all 7 restraints is heavily reduced to less
than 0.04\% of the full interaction space. The accessible interaction space of
the PUP2 chain with respect to PRE5 is overlapping with its center of mass
deduced from a homology model (Figure S1). 

The computation for those two examples took 74m and 27m on 16 AMD Opteron 6344
processors and 76m and 19m on an NVIDIA GeForce GTX 680 GPU, respectively.
However, by increasing the voxel spacing to 2Å and using a coarser rotational
search (9.72°, 7416 orientations) rather similar results can be obtained in
only 19m and 8m, respectively, on a single processor (cf. Table S2 and S4 for
example). It should further be noted that the bulk of the time is spent on
computing the FFTs and a negligible part on computing the consistent distance
restraint space (Table S11).

