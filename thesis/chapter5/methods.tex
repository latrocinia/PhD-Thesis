% math definitions
\define\Cmax{{\bf C}_{\text{max}}}
\define\Imin{{\bf I}_{\text{min}}}
\define\rvec{\vec{r}}
\define\rvecF{\rvec_{\text{F}}}
\define\rvecS{{\rvec}_{\text{S}}}
\define\rveccomS{{\rvec}_{\text{comS}}}
\define\boldL{{\bf L}}
\define\boldC{{\bf C}}
\define\boldI{{\bf I}}
\define\boldS{{\bf S}}
\define\boldF{{\bf F}}
\define\boldA{{\bf A}}
\define\redais{{\bf A}_{R, \text{red.}}}
\define\dmin{d_{\text{min}}}
\define\dmax{d_{\text{max}}}
\define\boldV{{\bf V}}
\define\boldP{{\bf P}}

\Section{Methods}

\Subsection{Overview}

We discretely sample the accessible interaction space by treating the two
biomolecules as rigid bodies and performing a 6 dimensional search over the
three translational and three rotational degrees of freedom. We use
FFT-techniques to accelerate the translational search using a 1\Angstrom\ grid
spacing (default). These have long been used in the docking field
\cite[Katchalski-Katzir1992].  One chain is fixed in space and considered the
receptor molecule, while translational scans are performed for each rotation of
the ligand molecule. Two atoms \m{i} and \m{j} are considered to be interacting if
the distance, \m{d}, between them is \m{\rvdW < d \leq \rvdW +} 3\Angstrom\ (by default),
where \m{\rvdW} is the combined van der Waals radius of the two atoms \m{\rvdW^i +
\rvdW^j}, and clashing if \m{d \leq \rvdW}. A conformation is deemed a complex if
the volume of interaction is above- and the volume of clashes below threshold
values (300 and 200\Angstrom\high{3} by default, respectively). 

After every translational scan, all conformations that comply with each
restraint are determined. Next, \disvis\ counts the number of accessible
complexes consistent with each number of restraints, as well as which
restraints are violated. This is repeated until the rotational sampling density
reaches a pre-set value (default 9.72\Deg, 7416 orientations). During the
rotational search, \disvis\ stores the maximum number of consistent restraints
found at every scanned position of the ligand’s center of mass, which
ultimately results in a discrete ‘density’ map. The output thus consists of the
sum of accessible interaction statescomplexes complying with each number of
restraints, a percentage of how often each restraint is violated, and a
discrete-valued density map.


\Subsection{Calculating the accessible interaction space of two interaction
macromolecules}

As a first approximation to calculate the accessible interaction space of two
interacting macromolecules and to make the computation more tractable, we treat
the molecules as rigid entities. This results in a 6 dimensional (3
translational and 3 rotational degrees of freedom) space of possible
conformations that need to be considered. To determine within this 6D space
whether the two chains are interacting and forming a complex, we use
FFT-techniques as used originally in \citeauthor{Katchalski-Katzir1992}.
We keep one chain fixed during the search while we perform FFT-accelerated
translational scans with the other chain. The fixed chain is separated into a
core and interaction region. The core region is the space that is occupied by
combining spheres with each center at the atom coordinate and as radius the
elements’ van der Waals radius; the interaction region is determined similarly,
but the radius is extended by 3\Angstrom\ (by default). The 3D shapes are subsequently
projected onto a grid with a voxel spacing of 1\Angstrom\ (by default). The scanning
chain is only represented by its core object. The resulting shape is again
projected onto a grid with equal voxel spacing as the fixed chain to allow for
FFT-accelerated translational scans during the search.

After the creation of the search objects, we identify clashes and interactions
as a function of rotation \m{R} as follows

\placeformula[eq:clashing-volume]
\startformula
\boldC \left( R \right) = \mathcal{F}^{-1} \left[ \mathcal{F} \left( \boldS \left( R
\right)\right)^* \times \mathcal{F} \left( \boldF_{\text{core}} \right) \right]
\stopformula

\placeformula[eq:interaction-volume]
\startformula
\boldI \left( R \right) = \mathcal{F}^{-1} \left[ \mathcal{F} \left( {\bf S} \left( R
\right)\right)^* \times \mathcal{F} \left( {\bf F}_{\text{inter}} \right) \right]
\stopformula

where the cross-correlation theorem has been used to calculate \m{\boldC} and
\m{\boldI}, the spaces that represent the volume of clashes and interactions at
every grid position in \Angstrom\high{3}, respectively; \m{\mathcal{F}} and
\m{\mathcal{F}^{-1}} represent the FFT operator and its inverse, respectively; \high{*}
is the complex conjugate operator, and \m{\times} the elementwise multiplication
operator; \m{\boldS} is the shape of the scanning chain, and \m{{\bf
F}_{\text{core}}} and \m{{\bf F}_{\text{inter}}} are the core and interaction
shapes of the fixed chain, respectively.

To determine whether a conformation is a plausible complex its clashing volume
should not be too large, while the interaction volume should be of reasonable
size. The accessible interaction space per translational space is then given by

\placeformula[eq:ais-trans]
\startformula
{\bf A}_R \left( \rvec \right) = \startmathcases
\NC 1 \MC \text{if}\ \boldC_R \left(\rvec \right) \leq \Cmax\ \text{and}\ \boldI_R\left(\rvec \right) \geq \Imin \NR
\NC \NC \NR
\NC 0 \MC \text{else} \NR
\stopmathcases
\stopformula

where \m{\Cmax} and \m{\Imin} are parameters representing the allowed maximum
volume of clashes (200\Angstrom\high{3} by default) and the minimum volume of
interactions (300\Angstrom\high{3} by default), respectively. Raising \m{\Cmax}
and lowering \m{\Imin} results in a more lenient counting of accessible states,
while lowering \m{\Cmax} and raising \m{\Imin} makes the counting for
accessible states more stringent. The total number of accessible states is
determined by performing an exhaustive search over rotation space and counting
at every rotation all states where \m{{\bf A}_R} equals 1.  Care should be
taken here that rotation space is as evenly and optimally sampled as possible
to minimize redundancy and biasing certain orientations in the counting. To
take this into account, we used the optimal rotation sets developed by
\citeauthor{Karney2007}, which include a weight factor for every rotation
to average out redundancy. The total number of accessible states \m{N_A} is
thus given by

\placeformula[eq:accessible-states]
\startformula
N_A = \sum_{\boldP} w_R \sum_{x, y, z} {\bf A}_R\left( x, y, z \right)
\stopformula

where \m{w_R} is the weight factor for the specific orientation/rotation, the first
summation is over all rotations \m{\boldP}, and the second summation over all grid
coordinates.


\Subsection{Incorporating distance restraints into the search}

If some distances or distance ranges are known between the subunits of the
complex, this can significantly reduce the accessible interaction space as it
puts extra restraints on the requirement for a conformation to be considered a
complex. To combine this information with FFT-accelerated translational scans,
the whole space of conformations that comply with the distance restraint should
be demarcated for every rotation at once. As the distance of the restraint
depends only on the coordinates of two atoms (or points, more generally), the
space consistent with the restraint must be represented by a sphere with a
radius corresponding to the distance restraint. The remaining parameter that
needs to be determined is the position of the center of this sphere
\m{\rvec_c}, which is given by

\placeformula[eq:center-of-sphere]
\startformula
\rvec_c \left( R \right) = \rvecF - \left[ \rvecS \left( R \right) - \rveccomS \left( R \right) \right]
\stopformula

where \m{\rvecF} and \m{\rvecS} are the coordinates of the restrained atoms of
the fixed and scanning chain, respectively, and \m{\rveccomS} is the center of
mass of the scanning chain. The equation can be simplified by initially placing
the center of mass of the scanning chain on the origin, and rotating the
scanning chain around its center of mass. Furthermore, realizing that
\m{\rvecF} is fixed and \m{\rvecS} now only depends on the rotation of the
scanning chain, \ineq[eq:center-of-sphere] reduces to

\placeformula[eq:simple-center-of-sphere]
\startformula
\rvec_c \left( R \right) = \rvecF - R \rvecS
\stopformula

The space of states complying with the distance restraint per translational
scan \m{\boldL_R} is defined then as

\placeformula[eq:distance-restraint-space]
\startformula
\boldL_R \left( \rvec \right) = \startmathcases
\NC 1 \MC \text{if}\ \dmin \leq |\rvec - \rvec_c | \leq \dmax \NR
\NC \NC \NR
\NC 0 \MC \text{else} \NR
\stopmathcases
\stopformula

where \m{\dmin} and \m{\dmax} are the minimum and maximum allowed distance,
respectively.  Note that the function describing \m{\boldL_R} can be freely
chosen, under the restriction that it should be spherical symmetric, which
opens up the use of more complex distance restraints in FFT-docking software.
The reduced accessible interaction space \m{\redais} is then simply given by

\placeformula[eq:reduced-ais]
\startformula
\redais = \boldA_R \times \boldL_R
\stopformula

In the case of multiple available distance restraints, this generalizes to

\placeformula[eq:generalized-reduces-ais]
\startformula
\redais = \boldA_R \times \sum_n^{N_d} \boldL_{R,n}
\stopformula

where the summation is over all distance restraints \m{N_d} and
\m{\boldL_{R,n}} is the space conforming to distant restraint \m{n}. The value
found at a specific coordinate in \m{\redais} represents the number of
conforming distance restraints at that location in space. 


\Subsection{Quantifying and visualizing the accessible interaction space}

To quantify the accessible interaction space consistent with a certain number
of distance restraints, the number of occurrences that \m{\redais} is equal to
the number of compliant restraints is counted. The accessible interaction space
is visualized by outputting the maximum value found during the rotational
search at every grid position, thus given by

\placeformula[eq:vizualization]
\startformula
\boldV \left( x, y, z \right) = \max \left\{ \redais\left(x, y, z \right): R \in \boldP \right\}
\stopformula

The resulting ‘density’ is written to file in MRC format and represents the
position of the center of mass of the scanning chain relative to the fixed
chain. These files can straightforwardly opened with molecular visualization
programs, such as PyMol and UCSF Chimera. With this information, interesting
regions of high- density can then be sampled more thoroughly. Also,
false-positive restraints can be identified if the exhaustive search does not
result in a region where all restraints of the cross-links are obeyed. In
addition, for each complex that is consistent with at least one restraint, all
restraints that are violated are calculated and stored during the search. This
ultimately results in a violation matrix where every row represents the number
of consistent restraints 𝑁 and every column indicates how often a specific
restraint is violated for complexes consistent with at least \m{N} restraints
(e.g.  \intable[tab:RPII-violations-fine]). Lastly, to give the user an
indication which restraints are most likely to be false-positives, the z-score
is calculated for each restraint based on the violation matrix given by

\placeformula[eq:z-score]
\startformula
Z = \frac{v_i - \bar{v}}{\sigma}
\stopformula

where \m{v_i} is the column average of the violation matrix of restraint \m{i},
and \m{\bar{v}} and \m{\sigma} are the average and standard deviation of the
violation matrix. {\emph{Disvis}} reports restraints with a z-score higher
than 1.0 explicitly.


\Subsection{Implementation details}

We implemented DisVis in Python2.7 using the NumPy \cite[Walt2011] and Cython
packages \cite[Behnel2011]. The OpenCL framework \cite[Stone2010] was used to
offload the computations to the GPU. Python bindings were available through the
pyopencl package \cite[Klockner2012]. We used the high-performance clFFT
library (\from[url:clfft]) together with gpyfft for Python bindings
(\from[url:gpyfft]) to calculate the FFTs.  Computations were performed on AMD
Opteron 6344 CPU processors and on an AMD Radeon HD 7730M and NVIDIA GeForce
GTX 680 GPU. DisVis code can be downloaded freely from \from[url:disvis]
together with documentation and examples.


\Subsection{RNA polymerase II example}

The crystal structure of the RNA polymerase II was downloaded from the Protein
Databank (PDB ID: 1WCM). The largest subunit (chain A) and the 27kDa
polypeptide (chain E) were extracted from the PDB-file. Six BS3 cross-links
were available and taken from XLdb \cite[Kahraman2013]. To investigate the
detection of false-positive restraints, two virtual cross-links were added with
a distance of 35.7 and 42.2\Angstrom\ using the Xwalk webserver
\cite[Kahraman2011]. The maximum allowed distance of the BS3 cross-links was
set to 30\Angstrom, based on molecular dynamics trajectory analysis
\cite[Merkley2014]. The restraints used are shown in
\instable[tab:RPII-crosslinks]. The input files are included in the DisVis
source code. 

Two disvis runs were performed using a rotational sampling density of 5.27\Deg\
(53256 orientations) and 9.72\Deg\ (7416 orientations) with a grid spacing of 1
and 2\Angstrom, respectively. All parameters were left to there default values
(interaction radius 3\Angstrom, minimum required volume of interaction
300\Angstrom\high{3} and maximum allowed volume of clashes
200\Angstrom\high{3}). The number of accessible complexes consistent with each
number of cross-links is shown in \instable[tab:RPII-accessible-complexes-fine]
and S\in[tab:RPII-accessible-complexes-coarse], and the relative occurrence of
restraint violations in \instable[tab:RPII-violations-fine] and
S\in[tab:RPII-violations-coarse] for the fine and coarse run, respectively.


\Subsection{26S proteasome PRE5-PUP2 example}

Homology models were downloaded from the SWIS-MODEL Repository
\cite[Kiefer2009] via the Protein Model Portal (\from[url:proteinmodelportal])
using their Uniprot identifiers (O14250 and Q9UT97). Cross-links were taken
from \citeauthor{Leitner2014} Dataset S1 (\instable[tab:PRE5-PUP2-crosslinks]),
which consist of 4 ADH and 3 zero-length ZL cross-links. The maximum ADH- and
ZL-linker length were set to 23 and 26\Angstrom, respectively, since
95\percent\ of all found distances in a benchmark were smaller. All input files
are included in the DisVis source code. 

Again, two \disvis\ runs were performed using a rotational sampling density of
5.27\Deg\ (53256 orientations) and 9.72\Deg\ (7416 orientations) with a grid
spacing of 1 and 2\Angstrom, respectively, with default parameter values. The
sum of accessible complexes consistent with each number of restraints is shown
in \instable[tab:PRE5-PUP2-accessible-complexes-fine] and
\ins[tab:PRE5-PUP2-accessible-complexes-coarse], and their normalized restraint
violation occurrence in \instable[tab:PRE5-PUP2-violations-fine] and
\ins[tab:PRE5-PUP2-violations-coarse]. 


