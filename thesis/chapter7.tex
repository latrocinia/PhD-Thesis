\startcomponent chapter7

\product thesis

\environment layout 


\Chapter{Summary and perspectives}

\Section{Summary}

The previous Chapters have mainly introduced and showcased novel approaches for
explorative and integrative modeling in the presence of cryo-EM data and
distance restraints.  In \inchapter[chapter:powerfit] I presented the PowerFit
software, a Python package for fast cross correlation based rigid body fitting
of high-resolution structures in low-resolution densities. PowerFit comes with
a new more sensitive scoring function, the core-weighted local cross
correlation, in addition to an optimized protocol for fast fitting. In
\inchapter[chapter:image-pyramids] I reported results of an extensive benchmark
of the PowerFit software using 379 subunits of 5 ribosome density maps. The
success rate of unambiguously fitting subunits bigger than 100 residues reached
approximately 90\% up to 12\Angstrom\ resolution, showing that objective
fitting methods have matured to aid in structural modeling. The limits of rigid
body fitting can be leveraged through the use of image pyramids to gain a
speedup of a factor of 30 on CPUs and 40 on GPUs, and it allows the
identification of possible over-interpreted regions of the density on an
objective basis. 

\inchapter[chapter:haddock-em] describes incorporation and benchmarking of
cryo-EM data into the data-driven docking program HADDOCK. The approach is
flexible and can be fully combined with other available sources of data in
HADDOCK. The approach was demonstrated on two ribosome systems, two
virus-antibody systems, and a symmetric pentamer. An update of the HADDOCK
webserver was presented in \inchapter[chapter:haddock2.2], and shows the
extensive usage of the software all over the world. 

\inchapter[chapter:disvis] dealt with explorative modeling using distance
restraints in general, and cross-link data specifically. I introduced the
concept of the accessible interaction space and presented a method to quantify
and visualize it. This directly indicates the information content of distance
restraints and shows whether all data is self-consistent and, if not, it gives
an indication of which restraint is a false-positive. The approach was
implemented in the DisVis software, another Python package, even though the
approach is general and can easily incorporated into FFT-based docking programs
for the incorporation of distance restraints. I extended the approach further
in \inchapter[chapter:inferring-interface-residues], and presented a method to
infer interface residues from distance restraints with the concept of the
average-interactions-per-complex (AIC). The AIC provides an objective
probability for a residue being at the interface in respect of the data.
Furthermore, I benchmarked the use of cross-link based distance restraints in
HADDOCK using four different approaches. It was shown that the unambiguous
distance restraints should either be complemented with center-of-mass
restraints or DisVis based ambiguous interactions restraints.


\Section{Challenges of integrative modeling}

The field of integrative modeling is still relatively young, and several
challenges are ahead that the whole structural biology community should face,
as integrative approaches are increasingly applied to solve the structure of
large macromolecular assemblies. Recently a task force was assigned by the
Worldwide PDB (wwPDB) to make recommendations for the field to follow in order
to consistently progress, resulting in the First wwPDB Hybrid/Integrative
Methods Task Force Workshop \cite[Sali2015]. The task force ultimately came up
with 5 recommendations for involving data-representation, model validation and
data-archives.  The 5 recommendations were that: 1) the experimental and
computational protocols in addition to the structural models should be
deposited; 2) multiple model representations should be allowed for multi-scale
and multi-temporal models; 3) new procedures should be developed to ascertain
model uncertainty and accuracy; 4) a federated system of data archives should
be created; and 5) publications standards need to be developed for integrative
models as is already the case for X-ray and NMR structures.

Thus point 1, 4 and 5 are mainly about the reproducibility of integrative
structural models, point 2 is about what data-structures and format standards
to use, so far all more practical matters reflecting the current immature
status of the field than real inherent scientific challenges, except for point
3 that explicitly deals with the precision and accuracy together with
validation of integrative models. 


\Subsection{Quid pro quo}


\Section{Future guidelines and additional fields of research}

\Subsection{Explorative modeling}

\Subsection{Formal structural biology}


\stopcomponent
