\startcomponent chapter7

\product thesis

\environment layout 


\Chapter{Summary and perspectives}

\Section{Summary}

The previous Chapters have mainly introduced and showcased novel approaches for
explorative and integrative modeling in the presence of cryo-EM data and
distance restraints.  In \inchapter[chapter:powerfit] I presented the PowerFit
software, a Python package for fast cross correlation based rigid body fitting
of high-resolution structures in low-resolution densities. PowerFit comes with
a new more sensitive scoring function, the core-weighted local cross
correlation, in addition to an optimized protocol for fast fitting. In
\inchapter[chapter:image-pyramids] I reported results of an extensive benchmark
of the PowerFit software using 379 subunits of 5 ribosome density maps. The
success rate of unambiguously fitting subunits bigger than 100 residues reached
approximately 90\% up to 12\Angstrom\ resolution, showing that objective
fitting methods have matured to aid in structural modeling. The limits of rigid
body fitting can be leveraged through the use of image pyramids to gain a
speedup of a factor of 30 on CPUs and 40 on GPUs, and it allows the
identification of possible over-interpreted regions of the density on an
objective basis. 

\inchapter[chapter:haddock-em] describes incorporation and benchmarking of
cryo-EM data into the data-driven docking program HADDOCK. The approach is
flexible and can be fully combined with other available sources of data in
HADDOCK. The approach was demonstrated on two ribosome systems, two
virus-antibody systems, and a symmetric pentamer. An update of the HADDOCK
webserver was presented in \inchapter[chapter:haddock2.2], and shows the
extensive usage of the software all over the world. 

\inchapter[chapter:disvis] dealt with explorative modeling using distance
restraints in general, and cross-link data specifically. I introduced the
concept of the accessible interaction space and presented a method to quantify
and visualize it. This directly indicates the information content of distance
restraints and shows whether all data is self-consistent and, if not, it gives
an indication of which restraint is a false-positive. The approach was
implemented in the DisVis software, another Python package, even though the
approach is general and can easily incorporated into FFT-based docking programs
for the incorporation of distance restraints. I extended the approach further
in \inchapter[chapter:inferring-interface-residues], and presented a method to
infer interface residues from distance restraints with the concept of the
average-interactions-per-complex (AIC). The AIC provides an objective
probability for a residue being at the interface in respect of the data.
Furthermore, I benchmarked the use of cross-link based distance restraints in
HADDOCK using four different approaches. It was shown that the unambiguous
distance restraints should either be complemented with center-of-mass
restraints or DisVis based ambiguous interactions restraints.


\Section{Challenges of integrative modeling}

The field of integrative modeling is still relatively young, and several
challenges are ahead that the whole structural biology community should face,
since integrative approaches are increasingly applied to solve the structure of
large macromolecular assemblies. Recently a task force was assigned by the
Worldwide PDB (wwPDB) to make recommendations for the field to follow in order
to consistently progress, resulting in the First wwPDB Hybrid/Integrative
Methods Task Force Workshop \cite[Sali2015]. The task force ultimately came up
with 5 recommendations for involving data-representation, model validation and
data-archives.  The 5 recommendations were that: 1) the experimental and
computational protocols in addition to the structural models should be
deposited; 2) multiple model representations should be allowed for multi-scale
and multi-temporal models; 3) new procedures should be developed to ascertain
model uncertainty and accuracy; 4) a federated system of data archives should
be created; and 5) publications standards need to be developed for integrative
models as is already the case for X-ray and NMR structures.

Thus point 1, 4 and 5 are mainly about the reproducibility of integrative
structural models, point 2 is about what data-structures and format standards
to use, so far all more practical matters reflecting the current immature
status of the field than real inherent scientific challenges, except for point
3 that explicitly deals with the precision and accuracy together with
validation of integrative models. Even though for several experimental methods
cross-validation (SAXS) and confidence intervals (cryo-EM) have been developed,
they have been sparsely used and thus far not been combined. For other methods
such as cross-links coupled with mass-spectrometry (CXMS) the statistical
propensities of derived distance restraints have only been sparsely studied
with small benchmark and sample sizes. To gain deeper insight into the
uncertainty of integrative models and current validation approaches, requires
new high-quality elaborate benchmarks on systems for which both high-resolution
structures are available for the bound state and the unbound state, of which
the protein-protein docking benchmark is a prime example. Thus, for the
integrative structural biology field to properly move forward a quid pro quo
mentality needs to be established of both experimental and computational
scientists. Especially for upcoming promising techniques as SAXS and CXMS
experimental data on multiple well-investigated systems is missing even for
binary protein interactions. Though there are data-bases for CXMS, they are
relatively small, e.g. the XLdb reports 62 intra-chain cross-links of which 34
are coming from a single RNA polymerase II system. The small sample size and
questionable reproducibility of the results are major inhibitors in the
development of robust validation and uncertainty protocols.


\Section{Future guidelines and additional fields of research}

\Subsection{Explorative modeling}

To adequately model the uncertainty of integrative models the emphasis should
be turned more towards the data itself by investigating the amount of
information the data carry by searching and quantifying the whole interaction
space, for which the methodology was presented in \inchapter[chapter:disvis].
The approach for appreciating the information content of distance restraints
can be further extended by using a statistical distance preference function,
i.e. a knowledge based potential, inferred from experimental data to better
investigate the probability distribution of the accessible interaction space.
Similar approaches can be developed for SAXS (though computationally more
expensive as the scattering curve needs to calculated millions to billions of
times), and other biochemical and biophyscial based potentials, such as surface
overlap/van der Waals interactions. Thus instead of saving the top X solutions
and heuristically optimizing the number acceptable models within it using a
linear combination of (pseudo-)energies, as is common in the docking field,
with a linear scoring function, the energy distributions can be analyzed to
give further indication of the reliability of each measure and from there on
define confidence intervals in models. 


\Subsection{Formal structural biology}




\stopcomponent
