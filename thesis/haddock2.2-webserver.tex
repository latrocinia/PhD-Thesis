\startcomponent haddock2.2-webserver

\product thesis

\environment layout
\setupexternalfigures[factor=broad, directory=haddock2.2-webserver/figures]
\input haddock2.2-webserver/figures

\Chapter[chapter:haddock2.2]{The HADDOCK2.2 webserver: User-friendly integrative modeling of
biomolecular complexes}

\emph{Based on: G.C.P. van Zundert, J.P.G.L.M. Rodrigues, M. Trellet, C.
Schmitz, P.L. Kastritis, E. Karaca, A.S.J. Melquiond, M. van Dijk, S.J. de
Vries and A.M.J.J. Bonvin. The HADDOCK2.2 webserver: User-friendly integrative
modeling of biomolecular complexes. J. Mol. Biol., submitted (2015).}

\Section{Introduction}

Cellular metabolism is a highly regulated and adaptive system where proteins,
the main participants, form a vast network of interactions collectively known
as the interactome. Knowledge of the three dimensional (3D) atomic structure of
protein-protein interactions is therefore critical for a fundamental
understanding of cellular and molecular biology, as well as for rational
drug-design. Unfortunately, solving such structures using classical
high-resolution methods (X-ray crystallography and NMR spectroscopy) is not
trivial, as each has its own limitations (e.g. protein flexibility, size,
strength of the interaction). Considering the magnitude of the interactome,
complementary high-throughput methods such as computational docking are
necessary if we aim to close the structure gap \cite[Petrey2014]. The goal of protein-protein
docking is to predict the structure of a complex starting from the individual
structures of its components \cite[Rodrigues2014], which can either be experimentally determined
or predicted \cite[Rodrigues2013].
 
Despite continuous advances in the field, the accuracy of ab initio docking –
without using any experimental restraints – remains generally low
\cite[Huang2015].  Data-driven approaches such as HADDOCK \cite[Dominguez2003,
deVries2007], which integrate information derived from biochemical, biophysical
or bioinformatics methods to enhance sampling, scoring, or both
\cite[Rodrigues2014], perform remarkably better. The information that can be
integrated is quite diverse: interface restraints from NMR, mutagenesis
experiments, or bioinformatics predictions \cite[deVries2011, Hopf2014]; shape
data from small-angle X-ray scattering \cite[Karaca2013] and cryo-electron
microscopy experiments (\inchapter[chapter:haddock-em]); and orientations of
the individual structures in the complex from NMR residual dipolar couplings
\cite[vanDijk2005], relaxation anisotropy \cite[vanDijk2006b] and pseudocontact
shifts experiments \cite[Schmitz2011]. The potential of data-driven docking is
reflected in the success of the HADDOCK server and software in recent CAPRI
experiments (Critical Assessment of Protein Interaction) \cite[Janin2005,
Lensink2013], as well as in the number of structures deposited (>120) in the
Protein Data Bank (PDB), which were calculated using our software.

Five years ago, we introduced the HADDOCK web server to provide a
user-friendly interface to the software and streamline its usage by
non-expert users in the structural biology field \cite[deVries2010]. Shortly after, it was
updated to handle multi-body docking \cite[Karaca2010]. The development of new and
improved protocols and the inclusion of additional sources of restraints
culminated in the recently released version 2.2 of the software, followed by
an update of the web server interfaces. Throughout the next section, we will
provide an overview of the newly updated HADDOCK web server, accessible at
http://haddock.science.uu.nl/services/HADDOCK2.2, which is freely accessible
to non-profit users upon registration, and discuss the most relevant
additions. We conclude by presenting usage statistics of the server to
demonstrate the usefulness and power of providing easy and free access to
scientific software.

\Section{Overview and advances}

The HADDOCK web server was created to facilitate the use of our docking
software, by removing the burden of its installation and setup, as well as
by providing validation routines for input data and options. In addition,
since HADDOCK runs are computationally demanding, the web server offers the
users access to sufficient resources - our local cluster(s) - to complete
their runs within a few hours. A grid-enabled version of the server can be
accessed via the WeNMR web site (www.wenmr.eu) \cite[Wassenaar2012], which uses resources
provided by the European Grid Initiative (EGI) and the associated National
Grid Initiatives (NGIs). This setup, which currently handles most
submissions, provides more than 110.000 CPU cores distributed over 41 sites
worldwide (see http://gstat.egi.eu/gstat/geo/openlayers\#/VO/enmr.eu).

The HADDOCK web server aggregates seven different interfaces, each associated
with a different level of control over the docking protocol reflected by the
number of parameters that can be changed. New users are granted access to the
Easy and the associated Prediction Interface only, but can request access to
the Expert and Guru levels, and their associated interfaces, if necessary.

The Easy interface provides the most basic level of control. It allows the user
to either upload two structures in PDB format or download them directly from
the RCSB PDB, and define sets of active and passive residues that represent the
(putative) interface. Unlike previous versions, HADDOCK 2.2 supports single
(protein, small molecule, RNA, or DNA) and mixed (protein-DNA, protein-RNA)
molecule types. This was implemented to handle the docking of proteins onto a
nucleosome complex -- a recent CAPRI target.

The Expert interface builds on the Easy interface and allows the user to
manually specify the protonation state of each histidine residue in the
proteins, which is otherwise determined automatically with MolProbity \cite[Chen2010].
Also, it offers control over which regions of the molecules are semi-flexible
and fully flexible segments, which has an impact during the refinement stage of
the docking. Lastly, the user is given the option to define the charge state of
the N- and C-terminus of the protein. The Expert interface also provides a
Distance Restraints section, where the user has the option to upload
user-defined ambiguous and unambiguous restraints files and/or use
center-of-mass restraints, useful for blind or ab initio docking when no other
information is available, but also to ensure compactness of the generated
models. The center-of-mass restraints are automatically generated by
calculating the dimensions of each molecule along the x, y and z-axis (\m{d_x,
d_y, d_z}) and summing the average of the two smallest components per molecule.
The resulting distance is used to define a restraint between the center of mass
of each subunit with an additional upper bound corrections of 1\Angstrom\ \cite[Karaca2013]. In
addition, the Expert interface gives control over the Sampling Parameters,
including namely the number of structures to generate at each stage and whether
or not to perform solvated docking \cite[vanDijk2006, Kastritis2013, vanDijk2013]. Finally, it exposes the
Clustering Parameters that define the clustering algorithm and cutoff. In
version 2.2, in addition to RMSD-based clustering, there is the option of using
the Fraction of Common Contacts (FCC) clustering algorithm \cite[Rodrigues2012], which is
significantly faster and especially useful for symmetric complexes.

The Guru interface gives full access and control to \textasciitilde500 parameters, nearly all
that are available in HADDOCK. The Distance Restraints section now offers a new
radius of gyration restraint, information that can be extracted, for example,
from SAXS experiments. Non-crystallographic Symmetry Restraints and Symmetry
Restraints are also available at this level and have been extended to handle
C4- and D2-symmetries in addition to the already available C2-, C3- and
C5-symmetries. There are additional sections for other types of NMR-based
restraints, such as Residual Dipolar Couplings \cite[vanDijk2005], Relaxation Anisotropy
\cite[vanDijk2006b], and the recently added Pseudo Contact Shifts \cite[Schmitz2011]. These latter require a
tensor distance restraints file and the definition of the rhombic and axial
components of the anisotropic tensor. Besides the restraints, all the energy
evaluations, scoring functions and analysis parameters can be tweaked; advanced
parameters for the sampling protocols are also available at this level,
offering a greater degree of control, for example, on the extent of each
refinement stage. There are also dedicated options to the solvated docking
protocol, which now uses by default propensities based on the Kyte-Doolittle
hydrophobicity scale, as these have been shown to improve the protocol \cite[Kastritis2013].
The original statistical-based propensities \cite[vanDijk2006], recently expanded to include
nucleotides \cite[vanDijk2013], can still be selected via a dropdown menu.

\placefigure[top][fig:haddock2.2-output]
{\getbuffer[cap:haddock2.2-output]}
{\externalfigure[fig:haddock2.2-output]}

The remaining four interfaces consist of: the Prediction Interface, which is
similar to the Easy interface, but with settings geared towards using
bioinformatics interface predictors such as CPORT \cite[deVries2011]; the Refinement Interface
(expert-level access), which runs only the water refinement stage on the
uploaded structures and can be used for scoring purposes; the Multi-body
Interface, based on the Guru interface, supporting upload of up to six
molecules that will be docked simultaneously \cite[Karaca2010] and also featuring the
Molecule Interaction Matrix section. This new addition displays a table with
scaling factors to adjust the interaction forces between different subunits,
allowing molecules to become invisible to each other during the docking, which
is useful in cases where multiple binding modes are required to satisfy the
experiment data (see for example \cite[Escobar-Cabrera2011]). The web server also offers a File
Upload Interface to allow the user to upload a run parameter file, created upon
successful validation and submission to the queue, and thus easily redo a
docking run or re-run it with slight changes in the parameters. Finally,
together with the creation of the Multi-body Interface, we setup an interface
called Gentbl to facilitate the creation of custom ambiguous interaction
restraint files between any number of molecules.

At submission time, once the input data have been properly validated, the
server offers the option to download a parameter file (also provided with the
results). Users are encouraged to save this file as it contains all required
input data and settings to repeat the docking, as recommended in the “Outcome
of the First wwPDB Hybrid/Integrative Methods Task Force Workshop”
(Recommendation 1) \cite[Sali2015]. After a successful docking run, the user
will receive an e-mail redirecting him/her to the results page (see
\infigure[fig:haddock2.2-output] for an excerpt of presented results). The page
indicates how many structures of the water-refined models could be clustered,
and lists the clusters in the order of their HADDOCK score. For each cluster,
detailed statistics are displayed, representing the average values calculated
over the top four best scoring structures within each cluster. Besides the
HADDOCK score and other standard energies (van der Waals, etc.), a z-score has
been added. The z-score represents how many standard deviations the HADDOCK
score of a given cluster is separated from the mean of all clusters, i.e. the
lower the z-score, the better. To visualize the results, plots are displayed at
the bottom of the results-page, showing for example the HADDOCK score of all
solutions against the interface-ligand-RMSD (i-l-RMSD) compared to the best
scoring structure, together with cluster averages and their spreads. 

\placelfigure[page,90][fig:haddock2.2-usage-statistics]
{\getbuffer[cap:haddock2.2-usage-statistics]}
{\externalfigure[fig:haddock2.2-usage-statistics]}

\Section{Usage statistics}

Since its opening in June 2008, the HADDOCK webserver has seen a sustained
increase in the number of registrations to reach over 5500 registered users to
date distributed all over the world
(\infigure[fig:haddock2.2-usage-statistics]). More than 103,000 runs have been
processed, 28\percent\ of which have run on EGI grid resources. This percentage
has increased to 75\percent\ for the HADDOCK2.2 server submission. An overview
of the number of runs processed per month with their distribution over local
and grid resources is shown in \infigure[fig:haddock2.2-usage-statistics].
Since the launch of the HADDOCK2.2 webserver in March 2015, an increased
fraction of runs are handled by the new 2.2 portal.  Statistics over the last
year (since January 2014) indicate that the portal is processing on average 75
docking runs per day. A majority of these runs are dealing with protein-protein
and protein-peptide docking (\textasciitilde61\percent), \textasciitilde19\percent\ correspond to
protein-nucleic acids systems and quite a significant fraction (\textasciitilde20\percent) is
dealing with protein-small molecule docking (both small ligand and
oligosaccharides). These numbers demonstrate the popularity and widespread
usage (both in terms of geographic distribution and type of systems being
studied) of our HADDOCK web server.

\stopcomponent
