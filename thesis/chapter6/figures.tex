\useexternalfigure[fig:precision-recall][precision-recall]
\startbuffer[cap:precision-recall]
\caption{Precision and recall rates.}
{The precision and recall rates are plotted against the
average-interactions-per-complex (AIC) cutoff. The results are shown averaged
over all 90 benchmarked complexes, and each difficulty category (58 Easy, 14
Medium, and 18 Difficult).}
\stopbuffer


\useexternalfigure[fig:benchmark-easy][benchmark-easy]
\startbuffer[cap:benchmark-easy]
\caption{Benchmark results on Easy complexes.}
{For each complex the best structure is plotted in terms, of the l-RMSD, for
the four procedures tested using (A) 3, (B) 5, and (C) 7 cross-links. The
dotted line is the ligand-RMSD cutoff for an acceptable model. Unambig:
unambiguous distance restraints; unambig + com: unambiguous restraints combined
with center-of-mass restraints; disvis: DisVis-based ambiguous interaction
restraints (AIRs); disvis + unambig: unambiguous restraints combined with
DisVis-based AIRs.}
\stopbuffer


\useexternalfigure[fig:benchmark-rosetta][benchmark-rosetta]
\startbuffer[cap:benchmark-rosetta]
\caption{Benchmark results on Medium and Difficult complexes.}
{For each complex the best structure is plotted in terms, of the l-RMSD, for
the four procedures tested using (A) 3, (B) 5, and (C) 7 cross-links. The
dotted line is the ligand-RMSD cutoff for an acceptable model. Unambig:
unambiguous distance restraints; unambig + com: unambiguous restraints combined
with center-of-mass restraints; disvis: DisVis-based ambiguous interaction
restraints (AIRs); disvis + unambig: unambiguous restraints combined with
DisVis-based AIRs. In (C) also the best structures of the Rosetta benchmark are
shown \cite[Kahraman2013].}
\stopbuffer
