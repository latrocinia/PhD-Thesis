\startcomponent summary-and-perspectives

\product thesis

\environment layout 

\startChapter[title={Summary and perspectives},
    reference=chapter:summary-and-perspective]

\Section{Summary}

The previous Chapters in this thesis have introduced and showcased novel
approaches for explorative and integrative modeling in the presence of cryo-EM
data and distance restraints.  In \inchapter[chapter:powerfit] I presented the
PowerFit software, a Python package for fast cross correlation based rigid body
fitting of high-resolution structures in low-resolution densities. PowerFit
comes with a new more sensitive scoring function, the core-weighted local cross
correlation, in addition to an optimized protocol for fast fitting. In
\inchapter[chapter:image-pyramids] I reported results of an extensive benchmark
of the PowerFit software using 379 subunits of 5 ribosome density maps. The
success rate of unambiguously fitting subunits larger than 100 residues reached
approximately 90\% up to 12Å resolution, showing that objective fitting methods
have matured to usable aids in structural modeling. The limits of rigid body
fitting can be leveraged through the use of image pyramids to gain a speedup of
a factor of 30 on CPUs and 40 on GPUs, and it allows the identification of
possible over-interpreted regions of the density on an objective basis. 

\inchapter[chapter:haddock-em] describes the incorporation and benchmarking of
cryo-EM data into the data-driven docking program HADDOCK. The approach is
flexible and can be fully combined with other available sources of data in
HADDOCK, making it a fully integrative modeling approach. It was demonstrated
on two ribosome systems, two virus-antibody systems, and a symmetric pentamer.
An update of the HADDOCK webserver was presented in
\inchapter[chapter:haddock2.2], together with extensive usage statistics of the
software all over the world. 

\inchapter[chapter:disvis] dealt with explorative modeling using distance
restraints in general, and cross-link data specifically. I introduced the
concept of the accessible interaction space and presented a method to quantify
and visualize it. This directly indicates the information content of distance
restraints and shows whether all data are self-consistent and, if not, it gives
an indication of which restraint is a false-positive. This was implemented in
another Python package, DisVis. The approach is general and can easily be
incorporated into FFT-based docking programs allowing the use of distance
restraints by combining the 'marriage made in heaven' of sampling and scoring
\cite[Vajda2013]. 

I extended this approach further in
\inchapter[chapter:inferring-interface-residues], presenting a method to infer
interface residues from distance restraints using the concept of the
average-interactions-per-complex (AIC) statistic. The AIC provides an objective
probability for a residue to be at the interface based on the available data.
Furthermore, I benchmarked the use of cross-link based distance restraints in
HADDOCK using four different approaches. My results show that using solely
unambiguous distance restraints is suboptimal; instead they should either be
complemented with center-of-mass restraints or DisVis-based ambiguous
interactions restraints.


\Section{Challenges of integrative modeling}

The field of integrative modeling is still relatively young, with several
challenges ahead that the structural biology community will have to face, since
integrative approaches are increasingly applied to solve the structure of large
macromolecular assemblies. Recently a task force was assigned by the Worldwide
PDB (wwPDB) to make recommendations for the field to follow in order to
consistently progress and allow a proper assessment of the quality of such
integrative models. The results of the First wwPDB Hybrid/Integrative Methods
Task Force Workshop were recently published \cite[Sali2015], with 5 main
recommendations about data-representation, model validation and data-archives.
These were that: 1) the experimental and computational protocols in addition to
the structural models should be deposited; 2) multiple model representations
should be allowed for multi-scale and multi-temporal models; 3) new procedures
should be developed to ascertain model uncertainty and accuracy; 4) a federated
system of data archives should be created; and 5) publications standards need
to be developed for integrative models as is already the case for X-ray and NMR
structures.

Thus point 1, 4 and 5 are mainly about the reproducibility of integrative
structural models, point 2 is about what data-structures and format standards
to use, so far all more practical matters reflecting the current immature
status of the field than real inherent scientific challenges. Point 3
highlights a current challenge in this field with respect to the precision and
accuracy together with the validation of integrative models. Even though for
several experimental techniques, cross-validation (SAXS \cite[Rambo2013]) and
confidence interval (cryo-EM \cite[Volkmann2009]) measures have been developed,
they have been infrequently used, except in X-ray crystallography where this
has been a standard since years (the concept of the free R-factor
\cite[Bruenger1992]), and thus far not been combined. For other methods such as
cross-links coupled with mass-spectrometry (CXMS) the statistical propensities
of derived distance restraints have only been sparsely studied for small
benchmark and sample sizes \cite[Kahraman2013, Leitner2014]. 

Gaining deeper insight into the uncertainty of integrative models and current
validation approaches, requires new high-quality and elaborate benchmarks on
systems for which high-resolution structures of both the bound and unbound
states are available, of which the protein-protein docking benchmark is a
prime example \cite[Vreven2015] (although not really representative of the
complexity of systems typically studies by integrative modeling approaches),
together with additional experimental data. Especially for upcoming promising
techniques as SAXS and CXMS, experimental data on multiple well-investigated
systems are missing even for binary protein interactions.  Although there are
databases for CXMS \cite[Zheng2013,  Kahraman2013], they are relatively limited
in size, e.g. the XLdb reports 62 intra-chain cross-links of which 34 are
coming from a single RNA polymerase II system \cite[Rappsilber2011,
Kahraman2013]. The small sample size and questionable reproducibility of the
results are major limitations in the development of robust validation and
uncertainty assessment tools.  Thus, for the integrative structural biology
field to properly move forward a \emph{quid pro quo} mentality needs to be
established between experimental and computational scientists: additional
experiments should be performed for the purpose of further understanding the
scope and limitations that the data are providing, so that, in turn, improved
computational models can be delivered to answer important biological questions.


\Section{Future guidelines and additional fields of research}

\Subsection{Explorative modeling}

To adequately model the uncertainty of integrative models more emphasis should
be put on the data themselves by investigating the amount of information the
data carry by searching and quantifying the whole interaction space. I
presented in this thesis a methodology in \inchapter[chapter:disvis] to assess
the information content of distance restraints, information that can be
obtained from a variety of techniques. Note, however, that the approach
presented is limited to binary complexes and fully characterizing the
interaction space of multi-component systems remains an open challenge. The
approach for appreciating the information content of distance restraints can be
further extended by using a statistical distance preference function, i.e. a
knowledge based potential, inferred from experimental data, to better
investigate the probability distribution of the accessible interaction space.
Similar approaches can be developed for SAXS (though computationally more
expensive as the scattering curve needs to be calculated millions to billions
of times, a more CPU-demanding process than a simple distance calculation), and
other biochemical and biophysical based potentials, such as surface overlap/van
der Waals interactions. Thus, instead of heuristically optimizing the number of
acceptable models within the top X best scoring structures using a linear
combination of (pseudo-)energies, as is common in the docking field
\cite[Vajda2009], the energy distributions can be analyzed to give further
indication of the reliability of each measure and from there to define
confidence intervals in models. Established probability distributions can
afterwards be used as Bayesian priors in an effort to move to Bayesian
statistical models.

Furthermore, current integrative modeling is often used to generate only a
handful or even, preferably, a single representative model of the data, even
though the original outset of the approach is to generate all data-consistent
models. This is unfortunate, since it hides many nuances and complexities of
the biological systems. For the integrative approach to live up to its
potential, requires a different mindset of the structural biologist in general:
a structural model should not be simply regarded as a single entity, but rather
as a whole set of conformations, as is already the case in NMR structure
ensembles. This insight is now also gaining ground in X-ray crystallography,
where methods are being developed that represent the electron density as a set
of conformers \cite[Fraser2011, vandenBedem2013] and ensembles
\cite[Burnley2012]. These representations are only the tip of the iceberg
within this mindset, as the ensemble space will be significantly bigger in the
presence of sparse data, such as CXMS data. Model representations should thus
become more diffuse with larger accessible interaction spaces, to accurately
present the ensembles consistent with the data. Again, explorative modeling
techniques can help here by quantifying the information content to provide
insight into the magnitude of the ensemble space, while concurrently easing the
transition from a single-structure mindset to an elaborate multi-ensemble
paradigm.  


\Subsection{Formal structural biology}

Further investigating the accuracy of individual experimental methods require
scientists that are trained in both computational and experimental techniques,
the \emph{hybrid scientist}. This allows the scientist to perform experiments
to further guide and validate the computational modeling, ultimately resulting
in a \emph{formal structural biology}, where instead of only advancing
biological insight, the emphasis is also put on investigating the accuracy and
precision of both models and experimental methods \emph{an sich} and the
interpretation of the generated results. In the semi-long run, this approach
will become a fertile and stabile foundation to build upon for in-depth
structural research in challenging and interesting biological systems and
networks. This will ultimately result in a more formal approach to structure
determination from multiple data sources.


\Section{Conclusion}

The (integrative) structural biology field is a fast moving and exciting field
of research, with many experimental and computational advances.  The most
recent dramatic example is of course the spectacular improvement of the cryo-EM
field due to direct electron detectors. Even though atomic resolution can now
be achieved for stable complexes, the bulk of the resulting densities need
additional data from diverse sources for a structural interpretation, requiring
high-end integrative methods. However, there are still significant challenges
to overcome for integrative modeling to become a standard tool in the toolbox
of structural biologists. These are mainly dealing with the reproducibility of
the results and the uncertainty of the models. By showcasing integrative
modeling approaches and introducing new methods for quantifying the information
content of experimental data this thesis has laid out some new building blocks
for the field to build upon and move forward.

\stopChapter

\stopcomponent
