Protein interactions underlie most of the complexities encountered in the cell.
They play a determining role in processes ranging from protein translation to
muscle contraction.  Numerous diseases are the result of mutations at the
interface of protein complexes \cite[Joerger2007, Lage2014].  For a thorough
and fundamental understanding of these processes and rational drug design,
knowledge of these interactions and interfaces at an atomic level is of
paramount importance \cite[Wells2007, Nero2014].  Unfortunately, the number of
available high-resolution structures of protein complexes determined by either
X-ray crystallography or NMR spectroscopy remains rather sparse compared to the
size of the interactome \cite[Mosca2013, Petrey2014].

Cryo-electron microscopy (cryo-EM) is a technique capable of imaging large
biomolecular complexes in their native hydrated state \cite[Orlova2011].  The
resolution is, however, usually limited to such extend that a direct atomic
view of the interface is out of the question.  In order to remedy this, cryo-EM
data are often combined with high-resolution atomic structures
\cite[Esquivel-Rodriguez2013].  The simplest and most common way of building
macromolecular assemblies into cryo-EM maps is by manual fitting of atomic
structures using dedicated graphics software \cite[Baker1996, Goddard2007].  A
more objective but less used method is full exhaustive search rigid body
fitting, for which a plethora of software has been developed as reviewed in
Esquivel-Rodríguez and Kihara (2013).  Still, as the resolution decreases,
placement of subunits becomes ambiguous, and more models need to be sampled
and/or additional data incorporated into the modeling to generate sensible
models. 

Protein-protein docking is in principle well suited for this task
\cite[Moreira2010, Huang2014], since it naturally samples a large number of
conformations and can take into account additional sources of information for
scoring and/or for driving the docking process \cite[Karaca2013,
Rodrigues2014].  Several docking programs have incorporated cryo-EM data into
their work flow.  MultiFit automatically segments the cryo-EM density using a
Gaussian mixture model to deduce anchors, subsequently docking the components
of the complex onto the anchors \cite[Lasker2010].  EMLZerD uses the cryo-EM
data to score the models using 3D Zernike descriptors
\cite[Esquivel-Rodriguez2012].  A recent approach has been implemented in
ATTRACT-EM \cite[deVries2012], which represents the cryo-EM data by a Gaussian
mixture model and fits the subunits into the map in a procedure reminiscent of
Kawabata's approach \cite[Kawabata2008]; the resulting models are then refined.
Most of these methods, however, separate the use of the cryo-EM data from the
use of other sources of information: They first fit the structures in the
density and only afterwards might take into account the physico-chemical
properties (energetics) of the interface.  Furthermore, they usually do not
actively use additional orthogonal information that may be available, such as
for example mutagenesis or mass-spectrometry cross-link data. 

Only few approaches have been published that can incorporate a variety of data
\cite[Alber2008], one of which is the Integrative Modeling Platform (IMP)
developed in the Sali group, which has the capability of integrating among
others cryo-EM data \cite[Topf2008, Schneidman-Duhovny2012,
Velazquez-Muriel2012].  Another approach is our in-house data-driven docking
software HADDOCK \cite[Dominguez2003, deVries2010a], which is already capable
of actively using information from various sources, such as mutagenesis, NMR
H/D exchange and cross-links data to name only a few.  In addition, it is able
to deal with multiple subunits \cite[Karaca2010], can handle proteins, peptides
\cite[Trellet2013], DNA \cite[vanDijk2006] and RNA complexes and any
combination thereof.  HADDOCK leveraged its unique ability to combine multiple
structural data into the modeling process to implement powerful strategies to
deal with large domain conformational changes \cite[Karaca2011].  Here we
describe how we have incorporated cryo-EM data into HADDOCK, such that the
density is actively used as an additional energy term during docking, scoring
and flexible refinement.  These cryo-EM restraints can be combined with all
other already available sources of information and restraints supported in
HADDOCK.  We first report on the optimization and benchmarking of our method on
17 complexes from the protein-protein docking benchmark 4.0 \cite[Hwang2010]
using simulated data of 10, 15 and 20Å, and a multi-component symmetrical
complex.  Then we demonstrate its applicability on five cases using available
experimental data for two ribosome complexes, based on 9.8Å \cite[Guo2011] and
13.5Å \cite[Boehringer2012] data; two virus-antibody complexes using 8.5Å
\cite[McCraw2012] and 21Å \cite[Wang2013] resolution data; and a symmetric
pentamer using 16Å negative stain data \cite[Dauden2013].  In several cases
additional interface information is included based on mutagenesis data and the
biology of the system.  The resulting models have high quality interfaces
without the clashes usually found in manually fitted models, revealing new
details of the interactions.
