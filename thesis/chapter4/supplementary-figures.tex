\useexternalfigure[fig:lcc-scores][lcc-scores][width=0.75\textwidth]
\startbuffer[cap:lcc-scores]
\caption{Local cross correlation scores for all 17 complexes using simulated cryo-EM data.}
{The local cross correlation score is plotted against the i-RMSD compared to
the native complex for all 17 complexes using simulated 20 (left), 15
(middle) and 10Å (right) resolution cryo-EM data.}
\stopbuffer

\useexternalfigure[fig:itw-refinement][itw-refinement]
\startbuffer[cap:itw-refinement]
\caption{Effect of the itw-flexible refinement stage with cryo-EM restraints on i-RMSD.}
{The i-RMSD improvement (i-RMSD it1 – i-RMSD itw) for all refined complexes
after itw when using 10 (A), 15 (B) and 20Å (C) resolution data. The dashed
vertical line in each figure represents the average i-RMSD improvement.}
\stopbuffer

\useexternalfigure[fig:2017-current][2017-current]
\startbuffer[cap:2017-current]
\caption{The deposited model of EMD-2017.}
{A ribbon representation of the 4ADV-model is shown in the middle. The
left and right figures are close ups of the interface. Atoms displayed as
yellow balls are clashes.}
\stopbuffer

\useexternalfigure[fig:KsgA-conservation][KsgA-conservation][width=0.75\textwidth]
\startbuffer[cap:KsgA-conservation]
\caption{Worm representation of KsgA showing the conservation score of each
residue.}
{The conservation score is higher for thicker and purple residues and lower for
thinner and blue residues. Conservation scores were determined using the
ConSurf web server \cite[Ashkenazy2010].}
\stopbuffer

\useexternalfigure[fig:simulated-densities][simulated-densities][width=0.9\textwidth]
\startbuffer[cap:simulated-densities]
\caption{Example of simulated cryo-EM data generated for benchmarking
HADDOCK-EM.}
{A surface representation of the 7CEI complex is shown on top. Under it, three
iso-surfaces are shown for simulated cryo-EM data at 10, 15 and 20Å\
resolution.}
\stopbuffer

\useexternalfigure[fig:KsgA-centroids][KsgA-centroids]
\startbuffer[cap:KsgA-centroids]
\caption{Determination of centroid position of KsgA in the cryo-EM density of
30S E.Coli ribosome.}
{Iso-contour of the 30S ribosome in gray and the iso-contour of local cross
correlation values at 0.5 (A, green) and 0.6 (B, red) as a result of the
full-exhaustive search. The centroid was positioned on the maximum correlation
value within the iso-contour shown in B.}
\stopbuffer

\useexternalfigure[fig:KsgA-initial-placement][KsgA-initial-placement]
\startbuffer[cap:KsgA-initial-placement]
\caption{Initial placement of the 16S rRNA and KsgA during it0, and active
residues of the 16S rRNA.}
{A front- (A) and side-view (B) of the 16S rRNA and
KsgA initial setup as was used during the rigid body docking stage. 
(C) A ribbon representation of the 16S rRNA with the active residues 768 – 773,
781, 782, 801 – 803, 899 – 902, 1512 – 1516 and 1523 shown in red.}
\stopbuffer

\useexternalfigure[fig:virus-centroids][virus-centroids][width=0.5\textwidth]
\startbuffer[cap:virus-centroids]
\caption{Determination of centroid positions for the Dengue-virus envelope
protein and antibody.}
{Iso-contour of a subunit part of the 21Å resolution cryo-EM data of
Dengue virus (grey), showing regions of high local cross correlation values
(0.35) for the envelope protein (red) and antibody (green).} 
\stopbuffer

\useexternalfigure[fig:2355-current][2355-current]
\startbuffer[cap:2355-current]
\caption{Current deposited model of the large terminase complex.}
{A ribbon representation of the current deposited terminase complex (4BIJ).
Multiple clashes (yellow ball-and-sticks) are observed when zooming in on the
interfaces of the subunits.}
\stopbuffer
