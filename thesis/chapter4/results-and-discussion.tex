\Section{Results and discussion}

\Subsection{Implementation of cryo-EM data into HADDOCK}

We first describe the implementation of cryo-EM restraints into the rigid body
docking stage of HADDOCK (HADDOCK-EM, \infigure[fig:it0-protocol]).  The
approximate position of the center of mass (COM) of each chain in the density
map is represented by a centroid.  The positions of these centroids can be
determined in multiple ways: subunits can first be placed manually in the
density to the correct position after which the COM can be calculated; a
full-exhaustive cross correlation search of the chains in the density using
rigid body fitting software can be used (e.g. \citeauthor{Hoang2013}) to
extract positions corresponding to high cross correlation values; several
automatic methods have been devised for simultaneous centroid placement
\cite[Birmanns2007, Lasker2010, Wriggers1998, Zhang2010]; a more elaborate
approach combines cross-link data with the cryo-EM map to infer the positions
of the subunits \cite[Murakami2013].

Once the centroids have been determined the docking can start. First the chains
are separated in space at an approximate minimal distance of 25\Angstrom\ from
each other and given a random orientation.  Distance restraints are defined
between the COM of each protein to either a specific centroid if one is able to
distinguish the two chains in the density, or ambiguously to all centroids if
the chains cannot be distinguished in the density.  The former can be
interpreted as unambiguous and the latter as ambiguous distance restraints. We
thus transform the density data into distance restraints for several reasons.
First and foremost, this increases the radius of convergence of pulling the
chains into the density towards specified positions compared to using a cross
correlation potential, making the approach more robust.  Indeed, when using
solely the cross correlation, we found that the chains often get stuck in local
minima before they can even interact with each other.  Secondly, the distance
restraints approach falls within the original philosophy of HADDOCK, making it
easier to combine cryo-EM data with other relevant information sources. Having
defined the cryo-EM derived distance restraints, we then dock the initial
complex by means of rigid body energy minimization, which effectively positions
it into the cryo-EM map to fit the centroids.  In the case of binary complexes,
the optimal orientation of the complex with respect to the density still needs
to be determined since the centroid-based docking allows for rotational
ambiguity.  Therefore, we perform a fine rotational search of the complex
around the axis formed by the line joining the centroids and score each
orientation using the cross correlation value between the model and the map.
The orientation corresponding to the highest cross correlation value is further
refined using a rigid body energy minimization where the energy consists of the
non-bonded interaction terms of classical force fields (intermolecular van der
Waals and electrostatic energies) and an added cross correlation potential.
Typically 10000 solutions are generated at the rigid-body docking stage.  All
calculations are performed with CNS (Crystallography and NMR System)
\cite[Brunger2007] (see Experimental Procedure section for details).

After the rigid body stage, the generated solutions are scored with the
HADDOCK-EM-it0 score, which correspond to the original HADDOCK score (see
\ineq[eq:haddock-it0]) complemented with a local cross correlation-based energy
(see Experimental Procedures, \ineq[eq:haddock-em-it0] --
\in[eq:haddock-em-itw]).  The 400 best scoring models are then refined using
the standard HADDOCK refinement protocol with an additional correlation-based
potential to further fit the chains into the density, while reckoning with the
energetics of the system. 

\placefigure[top][fig:it0-protocol]
{\getbuffer[cap:it0-protocol]}
{\externalfigure[fig:it0-protocol]}

\placefigure[top][fig:it0-results]
{\getbuffer[cap:it0-results]}
{\externalfigure[fig:it0-results]}


\Subsection{Impact of cryo-EM data in the rigid body docking stage}

Since the HADDOCK protocol consists of several stages (rigid body docking and
scoring (it0), and flexible refinement stages in vacuum (it1) and explicit
solvent (itw)), we will separately discuss the impact of incorporation of
cryo-EM data on each stage.  We investigated the use of 10, 15 and 20\Angstrom\ 
simulated cryo-EM data on a benchmark consisting of 17 complexes taken from the
protein-protein docking benchmark 4.0 \cite[Hwang2010].  These complexes
consist of 10 easy, 4 medium and 3 hard cases (based on the degree of
conformational changes taking place upon complex formation) and are listed in
\intable[tab:benchmark].  Even though the complexes in the benchmark are
significantly smaller than what can be imaged by cryo-EM, their use is still
justified to optimize our protocol and investigate the limits of using density
data during the docking.

\placetable[top][tab:benchmark]
{\getbuffer[cap:benchmark]}
{\getbuffer[tab:benchmark]}

As a reference to assess the performance of using cryo-EM data in the
it0-stage, we used the ab initio mode of HADDOCK (HADDOCK-CM), which uses
center of mass distance restraints between molecules to drive the docking
\cite[Karaca2013].  We investigated two different performance indices at this
stage, namely the interfacial quality of the best-generated solution, or
interface RMSD (i-RMSD) as defined by the CAPRI standards \cite[Janin2003],
and, secondly, the number of acceptable solutions at the rigid-body docking
stage among the 10000 models generated.  We define an acceptable solution as
having an i-RMSD ≤ 4.0\Angstrom\ from the native complex.

As can be seen in \infigure[fig:it0-results]{A}, HADDOCK-CM generates at the
rigid body stage at least one acceptable solution out of 10000 in 11 of the 17
cases, of which 9 come from easy and 2 from medium difficulty targets.  The
HADDOCK-EM protocol generates at least one acceptable solution in 13 out of 17
cases, independent of the resolution of the simulated density maps used for the
docking, of which all 10 easy targets, 2 medium and 1 hard target.  The quality
of the best-generated model improves for all complexes compared to HADDOCK-CM,
except for the smallest 2OOB complex when using 15 and 20\Angstrom\ resolution
data.  The average i-RMSD improvement is 1.2, 1.5 and 1.6\Angstrom\ when using
20, 15 and 10\Angstrom\ data, respectively.  Even for complexes for which no
acceptable solutions were generated, there is a considerable increase of
quality, e.g. the i-RMSD of the hard 1JMO complex decreases from 6.13 for
HADDOCK-CM to 4.66, 4.35 and 4.51\Angstrom\ when using 20, 15 and 10\Angstrom\
data, respectively. 

Moreover, not only the quality of the interface of the best model benefits from
the use of cryo-EM data, but the number of generated acceptable solutions also
increases significantly (\infigure[fig:it0-results]{B}). For HADDOCK-CM the
median number of generated acceptable solutions is 1, while for HADDOCK-EM it
raises to 8, 17 and 46 when using 20, 15 and 10\Angstrom\ data, respectively. The
only complex where HADDOCK-CM actually generates more acceptable solutions
compared to HADDOCK-EM is again the small globular 2OOB complex.

As our protocol is dependent on the input of centroid coordinates, we also
investigated its sensitivity to incorrect centroid placement.  To this end, we
repeated the docking for 5 cases where both centroids were separately displaced
by 3, 5 and 7\Angstrom\ in a random direction.  The total error in placement
was thus 14\Angstrom\ total in the latter case. The difference in the number of
acceptable solutions generated in the top 400 differed per case (see
\instable[tab:centroid-displacement]).  Only at 7\Angstrom\ displacement of
both centroids does the number of acceptable solutions in the top 400 decrease
consistently, but is still significantly larger compared to HADDOCK-CM.  Thus,
our approach is robust against centroid placement errors up to at least
7\Angstrom.


\Subsection{Impact of cryo-EM data on the scoring of rigid body docking solutions}

\placefigure[top][fig:impact-refinement]
{\getbuffer[cap:impact-refinement]}
{\externalfigure[fig:impact-refinement]}

To incorporate the cryo-EM data into the scoring function, we supplemented the
original HADDOCK score with a local cross correlation (LCC) energy term
(HADDOCK-EM score).  The efficiency of this combined score is shown in
\infigure[fig:it0-results]{B}.  The HADDOCK-CM models were scored with the
original HADDOCK score, which resulted in at least one acceptable solution in
the top 400 models for 7 out of the 11 successful cases, where at least one
acceptable solution was generated.  The HADDOCK-EM models were scored with the
HADDOCK-EM score, which resulted in at least one acceptable for all 13
successful cases, irrespective of resolution, with the exception of the 2OOB
complex using 20\Angstrom\ resolution data. 

To investigate the effect of the LCC term in the HADDOCK score, the total
number of acceptable solutions in the top 400 was calculated for the HADDOCK-EM
models using the regular HADDOCK and HADDOCK-EM score.  The influence of the
LCC term in the HADDOCK score is significant, as the median number of
acceptable solutions in the top 400 increases from 3 to 5, 4 to 13 and 13 to 38
when using 20, 15 and 10\Angstrom\ data respectively.  The HADDOCK-EM score is
able to rank 52\percent, 69\percent\ and 78\percent\ of the generated acceptable solutions in the
top 400 compared to 38\percent, 39\percent\ and 41\percent\ when using the regular HADDOCK score at
20, 15 and 10\Angstrom resolution data, respectively.

The discriminative ability of the LCC-term increases with the resolution, as
expected.  When plotting the LCC versus the i-RMSD (see
\insfigure[fig:lcc-scores]), we observe a funnel shape for most complexes, with
high LCC values found for complexes with low i-RMSD values.  This becomes even
more pronounced as the resolution of the data increases.  For higher i-RMSD the
correlation is lost and the LCC is no longer indicative of the quality of the
solutions as was observed before \cite[Shacham2007].  It should further be
noted that the absolute value of the LCC-term is not indicative of the quality
of the model.  For example, when using 20\Angstrom\ data correlation values of
> 0.9 are routinely found for non-native models.  As such, the correlation
> value only has meaning in a comparative setting, urging the need to sample
> and score multiple conformations. 


\Subsection{Effect of cryo-EM data on the flexible refinement stage}

\placefigure[top][fig:1884-results]
{\getbuffer[cap:1884-results]}
{\externalfigure[fig:1884-results]}

Next we investigated the impact of incorporating cryo-EM restraints on the
flexible refinement stage of HADDOCK.  We calculated the i-RMSD improvement of
the 400 best scoring it0-models after each refinement stage for all complexes.
A histogram of i-RMSD improvements after the it0 and itw stage is shown in
\infigure[fig:impact-refinement].  The average i-RMSD improvement after
refinement when using 20\Angstrom\ data is 0.20\Angstrom\ with a maximum of
2.49\Angstrom.  This increases to an average of 0.33 and 0.45\Angstrom\ and a
maximum of 3.34 and 4.37\Angstrom\ when using 15 and 10\Angstrom\ data,
respectively.  The average i-RMSD improvements between it1 and itw are modest:
0.04, 0.05 and 0.10\Angstrom\ when using 20, 15 and 10\Angstrom\ data with
maximums of 0.25, 0.34 and 0.43\Angstrom, see \insfigure[fig:itw-refinement].
So the bulk of the improvement is gained during the it1-stage, which was also
previously noted (see Figure 2 in \cite[deVries2007]).  The maximal improvement
observed with cryo-EM restraints is about two times larger than what was
previously observed in an analysis of our CAPRI predictions.  This substantial
improvement is also reflected in the increased number of acceptable solutions
after the refinement for each complex (\instable[tab:acceptable-solutions]).
The number of cases with at least one acceptable increases from 13 for
20\Angstrom resolution data to 15 for the 15 and 10\Angstrom\ resolution data
(\intable[tab:benchmark-quality]).  The resulting models are ultimately
re-scored using the itw-HADDOCK-EM score (\ineq[eq:haddock-em-itw]).  The
enrichment of models in the top 400, 10 and 1 compared to HADDOCK-CM are given
in \instable[tab:enrichment].

\placetable[top][tab:benchmark-quality]
{\getbuffer[cap:benchmark-quality]}
{\getbuffer[tab:benchmark-quality]}


\Subsection{Docking two ribosomal proteins using experimental 9.8\Angstrom\ cryo-EM data}

As a test case using experimental cryo-EM data, we docked the S7 and S19
proteins of the 30S E. coli ribosome using a 9.8\Angstrom\ cryo-EM map
(EMD-1884).  The map has a corresponding atomic structure (2YKR), which has
been modeled by manual fitting a crystal structure of the full ribosome in the
map as a rigid body.%TODO maybe add figure

We docked the two proteins using only the fraction of the cryo-EM density that
can be attributed to the two proteins (\infigure[fig:1884-results]{A}). The
centroids were determined by calculating the position of the COM of each
protein as they were currently placed in the density. Applying HADDOCK-EM
resulted in 15 clusters, with the best scoring cluster containing 105 of the
400 generated solutions of which the best scoring complex has an i-RMSD of
1.56\Angstrom\ compared to the crystal structure
(\infigure[fig:1884-results]{B}).


\Subsection{Integrative modeling of KsgA with rRNA using 13.5\Angstrom\ cryo-EM
data}

\placefigure[top][fig:2017-data]
{\getbuffer[cap:2017-data]}
{\externalfigure[fig:2017-data]}


As a more realistic example, we applied HADDOCK-EM to model the binding of
KsgA, a methyltransferase, to the 30S maturing E. coli ribosome.  Crystal
structures are available for the 30S ribosome and KsgA together with a 13.5Å
cryo-EM map of the complex (EMD-2017).  The rRNA can be unambiguously fitted in
the density because of the higher density of the phosphates in the backbone.
The cryo-EM data clearly show the density of KsgA, revealing that helices 24,
27 and 45 of the rRNA are involved in the interaction
(\infigure[fig:2017-data]{A}), which has been corroborated by hydroxyl
radical foot-printing data \cite[Xu2008].  Mutagenesis data show that the
positively charged residues R221, R222 and K223 of KsgA are important in the
interaction (\infigure[fig:2017-data]{B}) \cite[Boehringer2012].

\placefigure[page][fig:2017-haddock]
{\getbuffer[cap:2017-haddock]}
{\externalfigure[fig:2017-haddock]}

The 13.5\Angstrom\ cryo-EM map has a corresponding current PDB model (4ADV).
This model, however, contains a large number of clashes at the interface
(>100).  Furthermore, it reveals no favorable interactions and fails to give a
clear explanation for the importance of the arginine residues identified by
mutagenesis (\insfigure[fig:2017-current]).  This is a typical side effect
from manual rigid body fitting.  Running HADDOCK-EM using the radical
foot-printing, mutagenesis and cryo-EM data results in a single cluster
(\infigure[fig:2017-haddock]{A}) of which the best solution has an i-RMSD of
2.8\Angstrom\ compared to the 4ADV-model.  The placement and orientations of
the chains in the density are similar to the rigid body fitted model as defined
by the cryo-EM data. The HADDOCK-EM model is, however, of much better quality:
it contains no clashes and reveals favorable hydrogen bonds made by R221, R222
and K223 with the backbone of the rRNA.  Moreover, new potentially key residues
can be identified, such as R147 and R248 (\infigure[fig:2017-haddock]{B}).
Coincidentally, these newly identified residues are also highly conserved,
corroborating our docking results (see \insfigure[fig:KsgA-conservation]).


\Subsection{Modeling virus-antibody complexes using 8.5\Angstrom\ and
21\Angstrom\ cryo-EM data}

\placefigure[page][fig:virus-haddock]
{\getbuffer[cap:virus-haddock]}
{\externalfigure[fig:virus-haddock]}

To show the diverse range of systems that can be handled with HADDOCK, we
applied our protocol on the adeno-associated virus 2 and immature Dengue virus
complexed with antibodies for which 8.5\Angstrom\ and 21\Angstrom\ cryo-EM data
and deposited models (3J1S and 3J42) are available, respectively.

For both cases we performed a HADDOCK run, combining the cryo-EM data with
interface information. Since the binding regions on the antibody are known as
well as the virus capsid proteins, residues that were within 5\Angstrom\ of the
other chain in the deposited atomic models were used as active residues.  The
solutions of the adeno-associated virus 2 converge into 1 cluster with an
i-RMSD less than 1.5\Angstrom\ from the deposited model
(\infigure[fig:virus-haddock]{A}).  However, when zooming in on the interface
of the best scoring HADDOCK model, the interactions show an extensive hydrogen
bond network between the envelope protein and the antibody in contrast to the
deposited model (\infigure[fig:virus-haddock]{B}). 

\placefigure[page][fig:symm-protocol]
{\getbuffer[cap:symm-protocol]}
{\externalfigure[fig:symm-protocol]}

The HADDOCK solutions of the Dengue virus cluster into two groups, with an
approximate i-RMSD of 2.0\Angstrom\ and 4.5\Angstrom\ with respect to the
deposited model (\infigure[fig:virus-haddock]{C}).  Inspecting the interface of
the best scoring HADDOCK model again shows favorable interactions between the
prM protein of the Dengue virus with the antibody, while the 3J42-model lacks
side-chains and shows a backbone clash (\infigure[fig:virus-haddock]{D}). 

\Subsection{Symmetrical multibody docking with cryo-EM data}


HADDOCK is capable of using symmetry restraints to drive the docking of
symmetrical assemblies.  In order to combine symmetry and cryo-EM restraints
the rigid body docking protocol was slightly modified compared to non-symmetric
complexes, with as main difference the initial placement of the subunits
(\infigure[fig:symm-protocol]{A}, Experimental Procedures).  We tested
HADDOCK-EM with symmetry on the cyclic pentamer of the trypsin inhibitor
(1B0C).  The ab initio mode of HADDOCK with C5 symmetry restraints results in
two acceptable solutions after the refinement stage, with the best solution
having an i-RMSD of 3.2\Angstrom\ compared to the 1B0C-structure.  Adding
cryo-EM data results in an increased number of acceptable solutions of 54, 400
and 400 when using 20, 15 and 10\Angstrom\ resolution data, respectively, with
the best models having i-RMSDs of 1.6, 1.0 and 0.7\Angstrom\
(\infigure[fig:symm-protocol]{B}).  Using higher resolution data also results
in more compact clusters, i.e. the distribution of i-RMSD values is reduced.
When using 10\Angstrom\ data only a single near-native cluster is observed. At
20\Angstrom\ resolution, multiple clusters appear and require the HADDOCK-EM
score to discriminate the near-native cluster, which indeed has the best
(lowest) HADDOCK-EM score. 

\placefigure[top][fig:2355-results]
{\getbuffer[cap:2355-results]}
{\externalfigure[fig:2355-results]}

We applied the symmetrical HADDOCK-EM protocol to model the pentameric large
terminase complex of bacteriophage T7 using 16\Angstrom\ negative stain EM data
(EMD-2355, \cite[Dauden2013]).  As in the previous cases, the corresponding
deposited model (4BIJ) shows clashes at the interfaces
(\insfigure[fig:2355-current]).  The 400 generated HADDOCK models resulted in 33
clusters, with the best scoring cluster having an i-RMSD of 2.9\Angstrom\
compared to the 4BIJ-model (\infigure[fig:2355-results]).  Again, the interface of
the best scoring HADDOCK model alleviates the clashes and shows favorable
interactions, while agreeing with the general binding mode of the 4BIJ-model.
