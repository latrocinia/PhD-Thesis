\Section{Introduction}

Determining the architecture of large macromolecular complexes is of
considerable interest to understand their function and mechanisms.  Classical
high-resolution methods such as X-ray crystallography and NMR spectroscopy
might, however, struggle in doing that for large complexes that might be too
flexible to crystallize or too large for peak assignment because of spectral
overlap in NMR.  Cryo-electron microscopy (cryo-EM) is quickly becoming the
method of choice to gain structural insight into the nature of such large
macromolecular assemblies.  Especially with recent advances in detector
technology and improved software and algorithms, the resolution of cryo-EM
density maps is steadily increasing, occasionally at the point where models can
be build in the density ab initio \cite[Bai2015].  Still, for the bulk of the
determined structures the level of detail is too low to routinely allow this
and additional information is required to build an atomic representation of the
system \cite[Villa2014].

Typically, cryo-EM data are complemented with known high-resolution three
dimensional (3D) models determined either experimentally or via homology
modeling.  These represent the pieces of the density puzzle that should all be
fitted together in the map.  The first step in the high-resolution modeling
process is placing the subunits as rigid entities at the correct position in
the density.  This is often done manually using graphics software, most notably
UCSF Chimera using its fit-in-map function \cite[Pettersen2004].  This is
unfortunate as it is subjective and can lead to over-interpretation of the
density map, as there is no objective scoring function to give an indication of
the goodness-of-fit.  This is especially problematic if flexible fitting is
applied afterwards, since for the refinement to make sense the subunit should
be located in a local minimum, else it might drift away from its initial
position during the process.  To this purpose a plethora of automatic rigid
body fitting software has been developed \cite[Esquivel-Rodriguez2013].  A
major class among those is the cross-correlation based programs, which are
often combined with a full-exhaustive six dimensional (6D) grid search of the
three translational and three rotational degrees of freedom \cite[Volkmann1999,
Roseman2000, Chacon2002, Kovacs2003, Wu2003, Garzon2007, Hrabe2012, Hoang2013].
This leads to a thorough and objective analysis of all possible solutions to
locate the global cross-correlation minimum.

The first full-exhaustive cross-correlation based software was published by
\citeauthor{Volkmann1999}.  The approach was further developed by
\citeauthor{Chacon2002} using the Fast Fourier Transform (FFT) algorithm in
combination with the cross-correlation theorem, which decreases the
computational complexity of the search.  In addition, they applied a Laplace
pre-filter on the density and search object, significantly extending the
applicable resolution range.  \citeauthor{Roseman2000} introduced the more
sensitive local cross-correlation (LCC) score to fit subunits instead of whole
complexes in the density.  \citeauthor{Wu2003} acknowledged the problem of
overlapping densities of neighboring subunits at lower resolutions and
developed a core-weighted (CW) cross-correlation score to minimize this effect
by biasing the weight of density toward the core of the search object.
Recently, \citeauthor{Hoang2013} implemented a GPU-hardware-accelerated version
based on FFT techniques to calculate the LCC score, building on the earlier
work by \citeauthor{Roseman2003}.

Here we report on further developments in cross-correlation based rigid body
fitting.  In the Methods section, we first shortly describe the essence of
exhaustive cross-correlation based fitting and introduce a new
cross-correlation function that combines the core-weighted approach of
\citeauthor{Wu2003} with the LCC, demonstrating how it can be calculated
using FFTs.  Furthermore, to decrease the time required to perform a full
exhaustive search we use the optimal rotation sets developed by
\citeauthor{Karney2007} and decrease the size of the density by automatically
resampling the data, if possible, and trimming padded regions.  In the Results
section, we investigate the sensitivity of the newly developed scoring function
by automatically fitting the subunits of the 80S D. melanogaster ribosome
\cite[Anger2013].  Lastly, we present a performance comparison against other
fitting software using the GroEL/GroES system with experimental data
\cite[Ranson2001].

We implemented our approach in a Python software package called PowerFit, which
can run on multi-core CPU machines and can be GPU-accelerated using the OpenCL
framework.  PowerFit has been tested on Linux, MacOSX and Windows operating
systems and is Free Software.  The source code with detailed installation
instructions and application examples can be found at \from[url:powerfit].

