\Section{Methods}

\Subsection{State of the art of rigid body cross-correlation based fitting}

The goal of cross-correlation rigid body fitting is to determine the three
translational and three rotational degrees of freedom of the model that
optimize the cross-correlation score between the high-resolution model and the
density. To this end, the model is first blurred to the resolution of the
cryo-EM data to properly calculate the goodness-of-fit. It should be noted
that, although the notion of the exact resolution of a cryo-EM density is still
a matter of debate and can actually be anisotropic, the reported resolution of
the data is usually sufficient for fitting purposes. This blurred model is then
fitted by performing a systematic, full-exhaustive search of the 6D space and
saving locations corresponding to high cross-correlation values. Predictably,
the problems with this approach are sensitivity of the scoring function and
speed of the search.  The sensitivity of the global cross-correlation score as
originally used by \citeauthor{Volkmann1999} is often compromised as, typically,
subunits instead of the whole complex are fitted into the density. To make
things worse, at lower resolution the local densities of neighboring subunits
are overlapping, resulting in systemic noise mainly at the edges of the search
model. To overcome the first problem, \citeauthor{Roseman2000} introduced the local
cross-correlation function, which effectively is the cross-correlation
normalized under the running footprint of the shape of the model. This
localizes the score to only the region of interest, making the fitting of
subunits feasible. As for the effect of overlapping densities of neighboring
subunits, this can be minimized by biasing the density toward the core of the
search object. \citeauthor{Wu2003} incorporated this concept by calculating the
core-index of each voxel of the search object, where the core-index is a
measure for how far the voxel is from an edge. To further enhance the
sensitivity of the scoring function, a Laplace pre-filter can be applied to the
cryo-EM density and search object \cite[Chacon2002]. Originally combined with the global
cross-correlation, it was recently shown that combining it with the LCC further
extends the applicable resolution range \cite[Hoang2013]. 

To increase the efficiency of the search and minimize computational costs, the
main innovation was the use of the cross-correlation theorem in combination
with FFTs. By discretizing the model density on a grid with the same voxel
spacing and size as the cryo-EM grid, a translational scan can be performed
using the FFT-accelerated approach. This reduces the computational complexity
from \m{N^2} to \m{N \log\left(N\right)}, where \m{N} is the total number of voxels
of the cryo-EM data.  After each translational scan, the model density is
rotated and the process is repeated until a pre-set rotational sampling density
is achieved, meaning that the time required for a search depends linearly on
the number of rotations sampled. The rotation step can be accelerated by
directly rotating the density of the search object instead of repeatedly
rotating the high-resolution model and blurring it afterwards. The
GPU-architecture especially is suited for this task as tri-linear interpolation
can be done with high-efficiency \cite[Hoang2013].

\Subsection{Increasing the sensitivity by combining the LCC with the core-weighted approach}

Originally the core-weighted procedure was combined with the global
cross-correlation, which significantly extended the resolution range in which a
subunit could be successfully fitted into the density. The same procedure is
expected to also improve the sensitivity of the better performing LCC.
Combining the two approaches results in what we defined here as the
core-weighted LCC (CW-LCC) scoring function: 

\placeformula[eq:cw-lcc]
\startformula
\text{CW-LCC} = \frac{1}{N} 
\frac{\sum_i^N \left( w_i\rho_c - \overline{\rho_c^w} \right) \cdot \left( w_i\rho_o - \overline{\rho^w_o} \right)}
{\sigma_c^w \sigma_o^w}
\stopformula

where the summation is over all the \m{N} voxels that are within a distance of
half the resolution of any atom of the search object indexed by \m{i}; \m{w_i}
is the core-index of voxel \m{i}; \m{\rho_c} and \m{\rho_o} are the intensities
of the search object and the cryo-EM density at voxel\m{i}, respectively,
\m{\overline{\rho_c^w}} and \m{\overline{\rho_o^w}} are the core-weighted
density average for the search object and the local cryo-EM density,
respectively, given by \m{\overline{\rho_x^w} = 1/N \sum_i^N w_i \rho_x}.
\m{\sigma_c^w} and \m{\sigma_o^w} correspond to the core-weighted density
standard deviation given by \m{\sigma_x^w = \sqrt{\left( \overline{\rho_x^w}
\right)^2 - \overline{\left( \rho_x^w \right)^2}}} with \m{\overline{\left(
\rho_x^w \right)^2} = 1/N \sum_i^N\left( w_i \rho_x\right)^2}.  The CW-LCC
reduces to the regular LCC by setting \m{w_i = 1}. The Laplace pre-filtered
scoring function is defined by performing the mapping \m{\rho_x \to
\nabla^2\rho_x} in \ineq[eq:cw-lcc]. In order to calculate the CW-LCC we first
need to define the core-index of each voxel.  

\Subsubsection{Determining the core-index \m{w_i}}

The core-index is a measure for how close a voxel is to the core of the density
of the subunit that is being fitted. We calculate the core-index by
progressively eroding a binary mask of the search object and summing each
eroded mask together, see \infigure[fig:core-index_resampling-trimming]{A} for
a 2D example. This guarantees that voxels at the surface have a low core-index
value, while voxels deeply buried get a higher value, even for complex shapes.  

\placefigure[top][fig:core-index_resampling-trimming]
{\getbuffer[cap:core-index_resampling-trimming]}
{\externalfigure[fig:core-index_resampling-trimming]}

\Subsubsection{Using Fourier techniques to calculate the CW-LCC}

Starting from \ineq[eq:cw-lcc] and following in the spirit of
\citeauthor{Roseman2003}, we can normalize the core-weighted density \m{w_i
\rho_c} of the template by setting \m{\overline{\rho_c^w} = 0} and
\m{\sigma_c^w = 1}, which simplifies \ineq[eq:cw-lcc] to

\placeformula[eq:cw-lcc-simplified]
\startformula
\text{CW-LCC} = \frac{1}{N}
\frac{\sum_i^N \rho_c^n \cdot w_i\rho_o}
{\sqrt{\left( \overline{\rho_o^w} \right)^2 - \overline{\left( \rho_o^w \right)^2}}}
\stopformula

where \m{\rho_c^n} indicates the normalized core-weighted density. This leaves
three terms to be determined: the nominator, which we refer to as the
core-weighted global cross-correlation (CW-GCC); the square of the average
core-weighted density, \m{\left(\overline{\rho_o^w}\right)^2}, and the average
of the squared core-weighted density, \m{\overline{\left( \rho_o^w\right)^2}},
of the cryo-EM data. These can be calculated using FFTs as follows 

\placeformula[eq:cw-gcc]
\startformula
\text{CW-GCC} = \mathcal{F}^{-1} \left[ \mathcal{F} \left( w \rho_c^n \right)^* \times \mathcal{F}\left( \rho_o \right) \right]
\stopformula

\placeformula[eq:square-average]
\startformula
\left( \overline{\rho_o^w} \right)^2 = \mathcal{F}^{-1} \left[ 
\mathcal{F} \left( w \right)^* \times \mathcal{F}\left( \rho_o \right) \right]^2
\stopformula

\placeformula[eq:average-squared]
\startformula
\overline{\left(\rho_o^w \right)^2} = \mathcal{F}^{-1} \left[ 
\mathcal{F} \left( w^2 \right)^* \times \mathcal{F}\left( \rho_o^2 \right) \right]
\stopformula

where \m{\mathcal{F}} and \m{\mathcal{F}^{-1}} are the Fast Fourier transform and
its inverse, respectively, \m{^*} is the complex conjugate operator, \m{\times} is
the element wise multiplication operator, \m{w} is the core-weighted mask,
\m{\rho_c} and \m{\rho_o} are the calculated and experimental densities,
respectively. In \ineq[eq:cw-gcc] it is the search object that is multiplied
with the core-weighted mask, instead of the cryo-EM density. It is this trick
which allows the CW-GCC to be calculated using FFTs. Note that even though
there are 9 Fourier transforms required to calculate the CW-LCC, only 6 need to
be calculated for every orientation sampled, as the 3 Fourier transforms of the
cryo-EM data can be calculated just once before the search. So the
FFT-accelerated CW-LCC effectively costs only one Fourier transform more than
the regular LCC \cite[Hoang2013].

\Subsection{Speeding up the search}

\Subsubsection{Using optimized rotation sets to limit rotational degeneracy}

Since the computational complexity of the exhaustive search depends linearly on
the number of rotations sampled, optimizing and limiting rotational degeneracy
is important for an efficient search. However, sampling rotations or
orientations in a systematic and efficient manner is a non-trivial exercise. As
such, the number of orientations that are sampled to guarantee a certain
rotational sampling density can differ widely. For example, COLORES uses
proportional Euler angles \cite[Chacon2002], while gEMFitter performs an
icosahedral tessellation to generate rotations \cite[Hoang2013], resulting in
1264 and 900 orientations sampled for a coarse 24° search, and 119664 and 92160
for a fine 5° search, respectively. 

In our implementation, we make use of the optimal rotation sets determined by
\citeauthor{Karney2007}, originally developed for solid state NMR. These
sets were pre-calculated by enclosing the hypersphere of unit quaternions and
require only 648 orientations for a 20.83° search and 70728 orientations at a
4.71° sampling rate. This is an enhancement of the sampling efficiency of at
least a factor of 1.3 compared to gEMFitter, while offering a denser rotational
sampling interval.

\Subsubsection{Decreasing the map size by resampling and trimming the density}

In addition to the number of rotations sampled, the computational complexity of
the search scales with \m{N \log\left( N \right)} where \m{N} is the number of
voxels of the data. This is the major determinant for the computational
resources required. Limiting the density size is thus key to limiting the time
required for a search. Cryo-EM data are often oversampled with respect to their
resolution incurring a significant computational cost to perform an exhaustive
search.  Because neighboring voxel intensities will be highly correlated,
resampling the cryo-EM data will not affect the scoring sensitivity
significantly. However, as there is still signal after the resolution cutoff,
resampling the cryo-EM data to Nyquist rate will introduce aliasing effects and
image distortions.  Therefore, we choose to resample the cryo-EM map to a
default rate of 2 times Nyquist, i.e. the data are resampled such that the
voxel spacing is 1/4th of the resolution, allowing for a safe buffer to
minimize aliasing effects. 

In addition to that, cryo-EM data are usually generously padded with voxels
containing only noise. It is not uncommon for the padding to increase the
number of voxels in each direction by a factor of 2 or more. This comes at a
considerable cost when performing an exhaustive search as the number of voxels
grows by a factor of 8 or more. To eliminate the computational cost incurred by
this padding, we trim the padded voxels. The effect of resampling and trimming
is shown in \infigure[fig:core-index_resampling-trimming]{B} on a slice of
the GroEL/GroES complex (EMD-1046). 

\Subsection{Implementation and availability}

We implemented our methods in a Python package named PowerFit that comes with a
command line tool eponymously named \powerfit. A flowchart of the \powerfit\
algorithm is shown in \infigure[fig:powerfit-flowchart]. It requires as input
a PDB structure, a cryo-EM map and its resolution. Optional parameters are the
rotational sampling density (default=10.83°), whether to resample and/or trim
the density and use the Laplace pre-filter and/or core-weighted procedure, and
the number of PDBs that should be written to file after the search. In
addition, the number of CPU processors available to the search can be specified
or whether the computations should be off-loaded to the GPU. 

\placefigure[page][fig:powerfit-flowchart]
{\getbuffer[cap:powerfit-flowchart]}
{\externalfigure[fig:powerfit-flowchart]}

After invoking \powerfit, the software will first try to resample the cryo-EM
map to 2 times Nyquist rate and then trim it. A density of the search object
(the 3D structure) is constructed by a Gaussian convolution where the standard
deviation is a function of the resolution. Also, a binary mask is computed out
the structure, where voxels within half a resolution distance from any atom in
the model are set to 1 and otherwise 0. Both the search object density and mask
are discretized on grids of equal sizes and spacing as the cryo-EM density map
to allow for an FFT-accelerated search. The Laplace pre-filter is applied on
the cryo-EM and template densities, if requested. A core-weighted mask is
calculated from the initial binary mask using the procedure described above.
The data necessary for the search are offloaded to the GPU if requested. The
template and mask are rotated using tri-linear interpolation, where texture
memory acceleration as described by \citeauthor{Hoang2013} is used when possible. For
each rotation sampled, a translational correlation scan is performed using
FFTs. The rotational solution with the highest score is saved at every grid
position. This continues until the requested rotational sampling density is
achieved. At the end, the grid, which contains at each position the highest
found cross correlation score for all sampled rotations, is segmented using a
3D watershed algorithm \cite[Volkmann2002] in order to remove redundant
solutions. The location of each maximum together with its correlation score and
corresponding rotation are written to file as well as the corresponding PDB
coordinates of the top \m{N} solutions (where \m{N} is a user-defined parameter). 

PowerFit is written in the Python language (Python2.7) and requires the NumPy,
SciPy and Cython packages. The CPU version can be further accelerated by
installing the FFTW3.3 library together with pyFFTW. To offload the
computationally intensive search to the GPU, we used the OpenCL framework
together with the clFFT library, a high-performance FFT library for OpenCL.
Python bindings were available through the pyopencl and gpyfft packages.
PowerFit is licensed under the MIT license and can be downloaded from
\from[url:powerfit] together with instructions on how to install and use it. It
has been successfully tested on Linux, MacOSX and Windows systems and its
GPU-accelerated version can run on both AMD and NVIDIA GPUs, minimizing vendor
lock-in.

