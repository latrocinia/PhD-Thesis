\SubSection{Scoring sensitivity of the core-weighted LCC}

To test the scoring sensitivity of the CW-LCC, we used PowerFit to fit each
subunit of the 80S D. melanogaster ribosome \cite[Anger2013] independently in
the density at different resolutions. To this end, we simulated cryo-EM data
from a deposited model (4V6W) from 6Å to 30Å resolution in 1Å increments. The
cryo-EM data were created using a Python script based on the molmap function in
UCSF Chimera. Subsequently, we fitted each subunit using the LCC and CW-LCC
score and also together with the Laplace pre-filter (L-LCC and L-CW-LCC)
resulting in four different scoring functions. As there are 86 subunits in the
assembly, we performed 8600 exhaustive searches in total (86 subunits $\times$
25 resolutions $\times$ 4 different scores). The voxel spacing of the simulated
data was 1/4th of the resolution with a maximum of 4Å using a rotational
sampling density of 20.83° (648 rotations). We defined a fit as successful if
the positional RMSD of a solution in the top 10 was smaller than 8Å compared to
the reference structure (4V6W), which is a reasonable 2 voxel spacing away from
the correct solution at 16Å resolution and lower. Since we were testing the
sensitivity of the scoring function, the orientation of the correct model was
used during each search. The results of the scoring comparison are shown in
\infigure{B}[fig:ribosome-results].

\placefigure[top][fig:ribosome-results]
{\getbuffer[cap:ribosome-results]}
{\externalfigure[fig:ribosome-results]}

All four scoring functions can fit all subunits correctly in the density at 6Å
and 7Å resolution. However, the performance of the LCC begins to decrease after
8Å resolution and the number of successful cases drops markedly up to 18Å
resolution, to further only decrease. The CW-LCC score performs significantly
better, only starting to drop at 10Å resolution. After that, it follows a
similar pattern as the LCC with a quick drop first and a more stable region in
the end. The core-weighted approach extends the applicable resolution range of
the LCC by a respectable 3Å. The scoring functions combined with the Laplace
pre-filter are evidently performing better. The L-LCC score is almost 100\%
successful up to 12Å resolution. The success rate drops at lower resolutions,
though not as fast as the LCC and CW-LCC score and follows a rather linear
trend, which is in contrast with the other scoring methods. The best performing
score is the L-CW-LCC as expected. It is capable of fitting all subunits up to
a resolution of 12Å and is near-perfect up to 15Å resolution. Similar to the
L-LCC score, the success rate drops linearly up to 30Å resolution. 

This analysis demonstrates that including both the Laplace pre-filter and the
core-weighted approach results in the most sensitive scoring function. The
Laplace pre-filter seems to have the largest impact, changing the drop rate of
the curve to a linear one, while the inclusion of the core-weighted approach
results in a right shift of the curve.


\SubSection{Fitting performance of \powerfit}

\Subsubsection{Fitting subunits in the GroEL/GroES complex}

As an experimental test case for \powerfit, we used the GroEL/GroES complex
(EMD-1046, \infigure{}[fig:GroEL-GroES] 4) \cite[Ranson2001], which has been
used in the cryo-EM modeling challenge and makes comparison with other software
possible \cite[Pintilie2012, Hoang2013]. The crystal structure of GroEL/GroES
(1GRU) was used as a reference. We fitted a subunit of the trans, cis rings of
GroEL and the whole GroES ring (as with other software attempts, fitting
individual subunits of GroES was not successful \cite[Hoang2013]) independently
in the density, using the four different scoring functions, with and without
resampling. The rotational sampling density was set at 4.71°. For the cis and
trans rings we took the top 7 best scoring fits and calculated the average RMSD
to the 1GRU reference structure; for the GroES ring we took the best fit only.
The results are shown in \intable[tab:fitting-performance].

\placetable[top][tab:fitting-performance]
{\getbuffer[cap:fitting-performance]}
{\getbuffer[tab:fitting-performance]}

The LCC score was not capable of fitting any subunit properly as was noted
earlier \cite[Hoang2013]. In case of the GroES lid, it actually places it
upside-down in the density. The CW-LCC is more successful in this respect, and
properly fits the GroES ring at the top of the density with an RMSD of 7.4Å
using the full map and 4.4Å when using the resampled map. However, it still
fails to accurately fit the trans and cis subunits in the density. In general,
the Laplace pre-filter scoring functions are capable of fitting all subunits
successfully in the density, with no significant difference in accuracy
considering the resolution of the data. As expected, the accuracy lowers when
we resample the map to two times Nyquist, though the difference is less than
one voxel spacing; when refitting the top 7 solutions using one translational
scan in the fitted orientation with the regular voxel spacing, similar results
are obtained, but at a markedly lower computational cost (see next section).
The fitting results from \powerfit\ (RMSD of 3.4, 3.8 and 4.2Å) are competitive
compared to previous published ones: gEMfitter reported an RMSD of 2.8, 4.0 and
5.3Å for the trans, cis and GroES ring \cite[Hoang2013], respectively, and
Segger 3.1, 5.1 and 6.0Å \cite[Pintilie2010].


\placefigure[top][fig:GroEL-GroES]
{\getbuffer[cap:GroEL-GroES]}
{\externalfigure[fig:GroEL-GroES]}

\Subsubsection{Timing comparison of \powerfit}

We also investigated the effect of trimming and resampling the density on the
time required to perform a run. As the Laplace pre-filter only needs to be
applied once, the timings of the regular and Laplacian scores are similar. We
therefore only show times for the L-LCC and L-CW-LCC scores. The results of the
timing runs are shown in \intable[tab:time-required-lcc].


Running a coarse 20.81° rotational search can be done in a few minutes, even on
a single processor with a map size of $128 \times 128 \times 128$ voxels.
However, for a fine rotational sampling density of 4.71° an exhaustive search
already requires more than 6 hours. Using a GPU (NVIDIA Geforce GTX680) to
accelerate the search reduces the time required to approximately 30 minutes.
When trimming the density before the search, which in the GroEL/GroES case
reduces the map size to $72 \times 72 \times 90$, the time required for a fine
search drops to ~ 1.5 to 2 hours on a single processor and only 5 minutes on a
GPU. It should be emphasized that trimming the map does not have any impact on
the search accuracy and thus should always be applied for a faster search.
Further minimizing the map size by resampling the density results in $36 \times
36 \times 45$ voxels, and only requires 15 minutes on a single CPU and 1 to 2
minutes on a GPU. Thus, we advise to always use the trimming option and start a
search using the resampled option. The resulting solutions can than be refitted
using a single translational scan on the non-resampled map for an optimal speed
to accuracy trade-off.

\placetable[top][tab:time-required-lcc]
{\getbuffer[cap:time-required-lcc]}
{\getbuffer[tab:time-required-lcc]}

We compared the fitting times of \powerfit\ against another GPU-accelerated rigid
body fitting software gEMfitter \cite[Hoang2013]. The results are shown in
\intable[tab:timing-comparison]. Running gEMfitter using a 5° rotational sampling density (92160
orientations) with the L-LCC scoring functions, requires 5 hours and 48 minutes
against 6 hours and 23 minutes for \powerfit, without any of the simplifications
introduced here, on a single processor (Intel Core i7-3632QM). As the bulk of
the time is spent on computing FFTs, the difference in performance might be
found in the fact that the gEMfitter binary has been compiled with the
mkl-library and \powerfit\ with GCC. gEMfitter also has a resampling option,
which reduces the running time to 38 minutes. Only applying the resampling
option reduces the running time for \powerfit\ to 41 minutes, and combined with
trimming the running time drops further to ~13 minutes using the same L-LCC
scoring function. We could not properly compare the GPU-accelerated version of
gEMfitter against \powerfit\ as the provided gEMfitter binary runs only on Ubuntu
systems with NVIDIA GPUs and was not at the authors’ disposal. However, the
gEMfitter article reports 11 minutes running time using a NVIDIA C2075 GPU,
which is significantly shorter than \powerfit\ without trimming and resampling.
With the latter two options turned on, the \powerfit\ timings drop to close to 1
minute on a GTX680 GPU card.  Again, since the bulk of the time is spent on
computing FFTs, the difference probably arises in the efficiency of the FFT
implementation: the CUDA FFT implementation is specifically optimized for
NVIDIA GPUs while the clFFT implementation is mainly optimized for AMD
architecture, but runs on all OpenCL supported architectures. So there is a
choice between performance versus portability, although, with trimming and
resampling enabled, \powerfit\ is still faster. 

\placetable[top][tab:timing-comparison]
{\getbuffer[cap:timing-comparison]}
{\getbuffer[tab:timing-comparison]}


\Subsubsection{Additional complexes fitted with \powerfit}

To validate our approach further, we applied \powerfit\ on three additional cases
in the resolution range of 8.9 to 13.5Å (\intable[tab:additional-complexes],
\infigure{}[fig:additional-cases]). EMDB entry 2325 is another GroEL/GroES
complex, but at a considerably higher resolution of 8.9Å compared to the 1046
density \cite[Chen2013]. The increased level of detail allowed to fit each
GroES subunit independently in the map, irrespective of the scoring function
used, with the correct 7 fits found in the top 7. The other two cases are
ribosomes with a GTPase \cite[Guo2011] and methyltransferase
\cite[Boehringer2012] bound to it, subunits with comparable size. For entry
1884 with a reported resolution of 9.8Å, the RsgA GTPase was correctly fitted
in the density by all four scoring functions and was found within the top 2
best scoring solutions. The ribosome map 2017 with the bound KsgA
methyltransferase has a somewhat lower resolution of 13.5Å. In this case, the
LCC was incapable of correctly fitting the subunit in the density. The other
scoring functions placed the subunit properly in the map, with the correct fit
found within the top 3 best scoring solutions.

\placetable[top][tab:additional-complexes]
{\getbuffer[cap:additional-complexes]}
{\getbuffer[tab:additional-complexes]}

\placefigure[top][fig:additional-cases]
{\getbuffer[cap:additional-cases]}
{\externalfigure[fig:additional-cases]}

