\startcomponent summary
\environment layout

\startBackmatterHead[title=Summary]

The remarkable diversity and complexity of life in all its facets is a source
of constant wonder and amazement all through the history of humankind. Even
though every unique individual has his own experience and interpretation on how
to approach life, the scientific inquiry of modern man has resulted in a
paradigm that life is organized on the molecular level, where the typical
dimension is that of the ångstrom (10\high{-10}m). This insight has lead to
an intense interest in the molecules of life. With the discovery of the
double-helix structure of DNA, the biomolecule that holds the genetic code, as
a prime example, it is postulated that function follows structure, i.e.
knowledge of the precise three-dimensional structure of large biomolecules
gives an indication of their function. Maybe more important, having precise
knowledge of the biomolecular structure holds the promise of rational drug
design by developing biologically active molecules that specifically interact
with particular patches at the interface of a complex or in the active site of
an enzyme.

X-ray crystallography and NMR spectroscopy are the classical experimental
methods that are capable of elucidating the atomic arrangement of large
biomolecules, such as proteins. The importance of proteins in the cell cannot
be understated: they are the main actors in almost all cellular processes,
ranging from muscle contraction to the building of new other proteins through
the ribosome.  Currently, more than 100,000 structures have been solved and
deposited in the Protein DataBank (\from[url:pdb]). However, the vast majority
of the structures are single proteins, while proteins typically perform their
function through interacting with other biomolecules, resulting in large
biomolecular complexes. Since the difficulty of solving a structure depends on
several parameters, such as the complex’ size, binding strength and
environment, and the number of biomolecular complexes is estimated to be two
orders of magnitude bigger than the number of individual proteins,
complementary computational methods are required to close the structure
knowledge gap.

Integrative modeling is a particular approach to computationally predict or
model the structure of a biomolecular complex by combining all experimental
knowledge that is available for the system. It is assumed that this ultimately
results in a more accurate and precise model, than using a single kind of data.
The challenges within this approach are, predictably, sampling and scoring:
many different possible models need to be generated (sampling) and evaluated to
identify the correct ones (scoring). In this thesis, I mainly focus on
incorporating cryo-electron microscopy data (cryo-EM) and chemical cross-links
coupled with mass spectrometry (CXMS) in the modeling process. Cryo-EM is a
fast developing method that typically provides low-resolution density
information of large macromolecular complexes, though, with current advances,
near-atomic resolution can be achieved; in contrast, CXMS provides long-range distance
restraints between amino acids, each thus providing orthogonal information.

In \inchapter[chapter:introduction] I layout a more technical introduction to
integrative modeling and the experimental data used. I introduce major software
used today that offer a wide array of integrative methods for macromolecular
modeling, such as our in-house data-driven docking software HADDOCK.
In addition, I propose a new concept, that of explorative modeling, where the
emphasis is put on quantifying and preferably visualizing the information
content of the data available, rather than outputting specific data-consistent
models as is the case in integrative modeling.

\inchapter[chapter:powerfit] describes PowerFit, a high-performance software
package and program for automatic rigid-body fitting of high-resolution
biomolecular structures into low-resolution cryo-EM density maps. PowerFit
performs a systematic 6 dimensional search of the 3 translational and 3
rotational degrees of freedom to find local cross-correlation optima to
objectively place structures in cryo-EM densities. In addition, I introduce a
new and sensitive core-weighted local cross-correlation that further extends
the applicable resolution range for successful fitting. PowerFit is a first
step in this thesis into combining high-resolution structural data with
lower-resolution cryo-EM data.

Next, in \inchapter[chapter:image-pyramids] I quantify the resolution
requirements for successfully rigid-body fitting a biomolecular structure into
a cryo-EM map as a function of the biomolecule’s size. I furthermore
unambiguously show that the core-weighted Laplacian-enhanced local
cross-correlation function is the best performing score overall. Finally, since
the resolution limits for successfully fitting a subunit are often remarkably
lower than the resolution of current cryo-EM data, these limits can leveraged
by using the concept of multi-scale image pyramids, to significantly accelerate
the fitting performance and reduce computational time and resources.

\inchapter[chapter:haddock-em] discusses the incorporation of cryo-EM data in
HADDOCK, resulting in a truly integrative modeling approach, allowing their
combination with  all other data sources already supported in HADDOCK. A
central concept in the approach is the use of centroids, points that represent
the approximate center of mass of each subunit that is docked. HADDOCK also
allows the use here of ambiguous restraints if the location of each chain
cannot be differentiated in the density. The use of low-resolution cryo-EM data
notably increases both the number and quality of acceptable solutions generated
through our approach. The use of this powerful integrative method is
demonstrated on two ribosome and two virus systems, where additional details of
the interface are revealed, providing new insights to guide future experiments.

The HADDOCK2.2 web server is described in \inchapter[chapter:haddock2.2]. The
web server offers structural biologists a user-friendly interface to the
upgraded HADDOCK2.2 software. Notable features are the introduction of mixed
molecule types, e.g. protein-DNA complexes, and additional NMR-based
restraints, such as residual dipolar couplings and pseudocontact shifts,
opening up new venues for macromolecular docking. The web server can be used
free-of-charge for academic purposes at \from[url:haddock2.2] after simple
registration. 

Starting from \inchapter[chapter:disvis], I shift to the use of distance
restraints in general and CXMS data in particular. I introduce another software
package and program, named DisVis for Distance Visualization. DisVis quantifies
and visualizes the information content of distance restraints through the
concept of the accessible interaction space, the countable set of all
data-consistent solutions. It furthermore allows the identification of possible
false-positive restraints, and shows whether all restraints are consistent.
DisVis represents a first effort into the newly introduced concept of
explorative modeling.

The approach is further extended in
\inchapter[chapter:inferring-interface-residues], where explorative modeling is
used to infer interface residues in a model-free approach. It is shown that
interface residues can be predicted up to 90\% accurate for rigid binders in
the presence of 3 to 7 long-range distance restraints. The resulting possible
interface patches can subsequently be used in HADDOCK as active and passive
residues, to enhance the robustness of the CXMS data by combining it with the
standard unambiguous distance restraints, which often are not very accurate in
terms of distance ranges in the case of CXMS data. 

In the final \inchapter[chapter:summary-and-perspective] I summarize my
findings, present a personal perspective on the field of integrative modeling
and propose additional fields for future research. I plea for the education of
the \emph{hybrid scientist}, an expert at the interface of experimental and
computational sciences, to push forward integrative computational modeling. I
also argue for a reinterpretation of structural models, as they should not be
regarded as single well-defined structures, but instead as ensembles, as has
been proposed for NMR. Especially in integrative modeling where models are
typically generated using sparse data, the ensemble space is significantly more
diffuse. Explorative modeling and the concept of the countable accessible
interaction space can help in easing the transition to this ensemble
interpretation, while at the same time quantifying the information content of
the data.

Overall, this thesis provides new approaches and computational tools to help
structural biologists in interpreting data coming from increasingly diverse
sources, and introduces a new vantage point to approach modeling through
explorative modeling.

\stopBackmatterHead

\stopcomponent
