\startcomponent summary
\environment layout

\startBackmatterHead[title=Summary]

The remarkable diversity and complexity of life in all its facets is a source
of constant wonder and amazement all through the history of humankind. Even
though every unique individual has their own experience and interpretation on
how to appraoch life, the scientific inquiry of modern man has resulted in a
paradigm that life is organized on the molecular level, where the typical
dimension is that of the angstrom (\m{10^{-10}}m), resulting in an intense
interest in the molecules of life. With the discovery of the double-helix
structure of DNA, the polymer that holds the genetic code, as a prime example,
it is postulated that function follows structure, i.e. knowledge of the precise
three-dimensional (3D) structure of large biomolecules gives an indication of
their function. Maybe more important, having precise knowledge of the
biomolecular structure holds the promise of rational drug design by building
medicinal molecules that specifically interact with particular patches at the
interface.

X-ray crystallography and NMR spectroscopy are the classical experimental
methods that are capable of elucidating the atomic arrangement of large
biomolecules, such as proteins. The importance of proteins in the cell cannot
be understated: they are the main actors in almost all cellular processes,
ranging from muscle contraction to the propagation of nerve impulses.
Currently, more than 100,000 structures have been solved and deposited in the
Protein DataBank (\from[url:pdb]). However, the vast majority of the structures
are single proteins, while proteins typically perform their function through
interaction with other biomolecules, resulting in larger biomolecular
complexes. As the difficulty of solving a structure increases exponentially
with the size of the complex, and the number of biomolecular complexes is
estimated to be two orders of magnitude bigger than the number of inidividual
proteins, computational methods are required to close the structure knowledge
gap.

Integrative modeling is a particular approach to computationally predict or
model the structure of a biomolecular complex by combining all experimental
knowledge that is available for the system. It is assumed that this ultimately
results in a more accurate and precise model, than using a single kind of data.
The challenges within this approach are, predictably, sampling and scoring:
many different possible models need to be generated (sampling) and evaluated
(scoring). In this thesis, I will mainly focus on incorporating cryo-electron
microscopy (cryo-EM) and chemical cross-links coupled with mass spectrometry
(CXMS) data in the modeling. Cryo-EM is a fast developing method that typically
gives low-resolution shape information of large macromolecular complexes, while
CXMS provides long-range distance restraints between atoms, each thus providing
orthogonal information.

In \inchapter[chapter:introduction] I layout a more technical introduction into
integrative modeling and the experimental data used. I introduce the major
softwares today that offer a wide array of integrative methods for
macromolecular modeling building, such as our in-house data-driven docking
software HADDOCK. In addition, I propose a new direction, that of explorative
modeling, where the emphasis is put on quantifying and preferably visualizing
the information content of the data available.

\inchapter[chapter:powerfit] describes PowerFit, a high-performance software
package and program for automatic rigid-body fitting of high-resolution
biomolecular structures in low-resolution cryo-EM data. PowerFit performs a
systematic 6 dimensional search of the 3 translational and 3 rotational degrees
of freedom to find local cross-correlation optima. In addition, it introduces
the sensitive core-weighted local cross-correlation, to further extend the
applicable resolution range for successful fitting. PowerFit is a first step
into combining or integrating X-ray crystallography cryo-EM data.

Next, in \inchapter[chapter:image-pyramids] I quantify the resolution
requirements for successfully rigid-body fitting a biomolecular structure of a
particular size in a cryo-EM map. I furthermore unambiguously show that the
core-weighted laplacian-enhaced local cross-correlation function is the best
performing score. Finally, since the resolution limits for successful fitting
are often remarkably lower than the resolution of current cryo-EM data, these
limits can leveraged by using the concept of multi-scale image pyramids, to
significantly accelerate the fitting performance.

\inchapter[chapter:haddock-em] discusses the incorporation of cryo-EM data in
HADDOCK, resulting in a truly integrative modeling approach, allowing the
combination of all data sources already incorporated into HADDOCK. It is shown
that the use of low-resolution cryo-EM data notably increases both the number
and quality of acceptable solutions. The use of the powerful integrative method
is shown on two ribosome and two virus systems, where additional details of the
interface are revealed, guiding the direction of future experiments.

The HADDOCK2.2 web server is described in \inchapter[chapter:haddock2.2]. The
web server offers structural biologists a user-friendly interface to the
upgraded HADDOCK2.2 software. Notable features are the introduction of mixed
molecule types, e.g. protein-DNA complexes, and additional NMR-based
restraints, such as residual dipolar couplings and pseudocontact shifts,
opening up new venues for macromolecular docking. The web server can be used
free-of-charge for academic purposes at \from[url:haddock2.2] after
registering. 

Starting from \inchapter[chapter:disvis], I shift to the use of distance
restraints in general and CXMS data in particular. I introduce another software
package and program, named DisVis for Distance Visualization. DisVis quantifies
and visualizes the information content of distance restraints through the
concept of the accessable interaction space. It furthermore allows the
identification of possible false-positive restraints, and shows whether all
restraints are consistent. DisVis is also the effort into the newly defined
area of explorative modeling.

The approach is further extended in
\inchapter[chapter:inferring-interface-residues], where explorative modeling is
used to infer interface residues in a model-free approach. It is shown that
interface residues can be prediced upto 90\% accurate for rigid binders. The
resulting possible interface patches can subsequently used in HADDOCK again, to
enhance the robustness of the CXMS data. 

In the final \inchapter[chapter:summary-and-perspective] I summarize my
findings and present a personal perspective on the field of integrative
modeling and provide additional fields of research. I plea for the education of
the \emph{hybrid scientist}, an expert on the interface of experimental and
computational science, to further computational modeling with diverse data. I
also argue for a reinterpretation of structural models, as they should not be
regarded as a single well-defined structure, but instead as a ensemble as is
already the case in NMR. Especially in integrative modeling where models are
typically generated using sparse data, the ensemble space is signicicantly more
diffuse. Explorative modeling and the concept of the countable accessible
interaction space can help in easing the transition, while also quantifying the
information content of the data.

On the whole, this thesis provides new approaches and computational tools to
help the structural biologist in interpretating the data coming from ever more
diverse sources, and introduced a new vantage point to approch modeling through
explorative modeling.

\stopBackmatterHead

\stopcomponent
