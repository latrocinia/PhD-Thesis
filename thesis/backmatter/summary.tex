\startcomponent summary
\environment layout

\startBackmatterHead[title=Summary]

The remarkable diversity and complexity of life in all its facets is a source
of constant wonder and amazement all through the history of humankind. Even
though every unique individual has their own experience and interpretation on
how to appraoch life, the scientific inquiry of modern man has resulted in a
paradigm that life is organized on the molecular level, where the typical
dimension is that of angstrom (\m{10^{-10}m}), resulting in an intense
interest in the molecules of life. With the discovery of the double-helix
structure of DNA, the polymer that holds the genetic code, as a prime example,
it is postulated that function follows structure, i.e. knowledge of the precise
three-dimensional (3D) structure of large biomolecules gives an indication of their
function. Maybe more important, having precise knowledge of the biomolecular
structure holds the promise of rational drug design by building medicinal
molecules that specifically interact with particular patches at the interface.

X-ray crystallography and NMR spectroscopy are the classical experimental
methods that are capable of elucidating the atomic arrangement of large
biomolecules, such as proteins. The importance of proteins in the cell cannot
be understated: they are the main actors in almost all cellular processes,
ranging from muscle contraction to the propagation of nerve impulses.
Currently, more than 100,000 structures have been solved and deposited in the
Protein DataBank (\from[url:pdb]). However, the vast majority of the structures
are single proteins, while proteins typically perform their function through
interaction with other biomolecules, resulting in larger biomolecular
complexes.  As the difficulty of solving a structure increases exponentially
with the size of the complex, and the number of biomolecular complexes is
estimated to be two orders of magnitude bigger than the number of inidividual
proteins, computational methods are required to close the structure knowledge
gap.

Integrative modeling is a particular approach to computationally predict or
model the structure of a biomolecular complex by combining all experimental
knowledge that is available for the system. It is assumed that this ultimately
results in a more accurate and precise model, than using a single kind of data.
The challenges within this approach are sampling and scoring: many different
possible models need to be generated (sampling) and evaluated (scoring). In
this thesis, I will mainly focus on incorporating cryo-electron microscopy (cryo-EM) and
chemical cross-links coupled with mass spectrometry (CXMS) data in the
modeling. Cryo-EM is a fast developing method that typically gives
low-resolution shape information of large macromolecular complexes, while CXMS
provides long-range distance restraints between atoms.

In \inchapter[chapter:introduction] I layout a more technical introduction into
integrative modeling and the experimental data used. 

\stopBackmatterHead

\stopcomponent
