\startcomponent acknowledgements

\environment layout

\startBackmatterHead[title=Acknowledgements]

Now after 3 years 11 months and a couple days of hard work while finishing up
the PhD and thesis, it's an apt time to reminisce about how I got here and give
thanks where thanks is required and deserved. My scientific career starting
really during my Bachelor in the Theoretical Chemistry group of dr. Joop van
Lenthe, where I worked on some relativistic quantum chemistry in the GAMESS-UK
package written in Fortran77.  Even though the project was not really a success
for several reasons, mainly my lazy work ethics, it fired up my interest into
computational science, and taught me essential computer skills. Furthermore,
Joop's approach to science and approachable demeanor are highly valued by me
and I'm grateful for his supervision. Later in my Master, my work ethics picked
up, and I met high-potentials such as Freddy Rabouw (now dr. Freddy Rabouw),
Niek den Harder, Hinke Schokker and Marieanne van der Haar, together forming
the \emph{power trio}. Their nerdy approach to study and make assignments made
my life during the Master easier and filled with loads of fun. 

Thanks to Niek, I decided to do my Master internship at the FOM Institute for
Plasma Physics in Nieuwegein, where I eventually joined the MolDyn group under
the supervision of dr. Anouk Rijs, and Sander Jaeqx (pronounce: Sjaaks). I had
a blast for the year I was there, fully enjoying the work atmosphere, castle
garden, excellent canteen, borrels, and great colleagues. It even resulted in
my first (first-author) paper. So I'm very thankful to Anouk, who also gave an
amazing  speech during the Master ceremony. However, after my Masters I did
want to return to full computational science and so I had to move on. After
applying for a few positions that all were already filled and about to move on
to an ordinary job, I came across an open position in the Computational
Structural Biology group of prof. dr. Alexandre Bonvin and applied. Naturally,
I was hired, and so my HADDOCKing time started \ldots

One of my first memories when I started out in the HADDOCK group is the
presentation by Ezgi Karaca about SAXS scoring of protein models. The
presentation contents and level really blew me away, and I was just so
impressed of the whole group in general. Ezgi turned out to be a great role
model for me during my PhD for several reasons: first of all, I'm the follow up
"data-integrate-or"; furthermore, she publiced a Structure paper; she received
a Keystone scholarship; and she's just a great scientist and person in general.
So I'm very much obliged to Ezgi for setting out a path to walk during my PhD.
Another example was set by Panagiotis Kastritis. Panos' day typically started
around noon and ended somewhere around midnight, a schedule that I also
appreciate, and he showed incredible scientific hunger and interest, making him
an inspiring fellow PhD student, whose desk I am now occupying after cleaning
it up. On the same desk at the short end there sat Mikaël Trellet, a FIFA guru.
Mikaël is one of the few people that know that I'm an amazing cook as I once
made him my famous tagliatelli salmon-spinach-cream fresh dish, and he was,
according to an independent source, the only guy who was funnier than me in the
lab. The last person that initially sat in this office was Marc van Dijk, the
only other Dutch HADDOCK person that I met during my PhD. So I want to thank
him for being Dutch, for providing tips for creating an award winning poster,
and inititally helping me out with installing software on my Mac initially. 
Christophe Schmitz was another postdoc present in the beginning. Christophe has
had a way bigger impact on my PhD than he probably realizes, as he referred
within the HADDOCK CNS source code to the paper 'Quaternions in molecular
modeling'. This paper has been indispensable in the creation of mainly DisVis
and, to a lesser extend, PowerFit. So only for that I'm already very grateful. 



\stopBackmatterHead

\stopcomponent
