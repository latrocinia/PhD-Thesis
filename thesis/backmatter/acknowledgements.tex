\startcomponent acknowledgements

\environment layout

\startBackmatterHead[title=Acknowledgements]

Now after 3 years 11 months and a couple days of hard work while finishing up
the PhD and thesis, it's an apt time to reminisce about how I got here and give
thanks where thanks is required and deserved. My scientific career starting
really during my Bachelor in the Theoretical Chemistry group of \bold{dr. Joop
van Lenthe}, where I worked on some relativistic quantum chemistry in the
GAMESS-UK package written in Fortran77.  Even though the project was not really
a success for several reasons, it fired up my interest into computational
science, and taught me essential computer skills. Furthermore, Joop's approach
to science and approachable demeanor are highly valued by me and I'm grateful
for his supervision. Later in my Master I met high-potentials such as
\bold{Freddy Rabouw}, \bold{Niek den Harder}, \bold{Hinke Schokker} and
\bold{Marie Anne van de Haar}, together forming the \emph{power trio}. Their
approach to studying and making assignments was enlightening and made my life
during the Master easier and filled with loads of fun.

Thanks to Niek, I decided to do my Master internship at the FOM Institute for
Plasma Physics in Nieuwegein, where I eventually joined the MolDyn group under
the supervision of \bold{dr. Anouk Rijs}, and \bold{Sander Jaeqx} (pronounce:
Sjaaks). I had a blast for the year I was there, fully enjoying the work
atmosphere, castle garden, excellent canteen, borrels, and great colleagues. It
even resulted in my first (first-author) paper. So I'm very thankful to Anouk,
who also gave an amazing  speech during the Master ceremony. However, after my
Masters I did want to return to full computational science and so I had to move
on. After applying for a few positions that all were already filled and about
to move on to an ordinary job, I came across an open position in the
Computational Structural Biology group of \bold{prof. dr. Alexandre Bonvin} and
applied. Naturally, I was hired, and so my HADDOCKing time started \ldots

One of my first memories when I started out in the HADDOCK group is the
presentation by \bold{Ezgi Karaca} about SAXS scoring of protein models. The
presentation contents and level really blew me away, and I was just so
impressed by the whole group in general. Ezgi turned out to be a great role
model for me during my PhD for several reasons: first of all, I'm the follow up
"data-integrate-or"; furthermore, she published a Structure paper; she received
a Keystone scholarship; and she's just a great scientist and person in general.
So I'm very much obliged to Ezgi for setting out a path to walk during my PhD.
Another example was set by \bold{Panagiotis Kastritis}. Panos' day typically
started around noon and ended somewhere around midnight, a schedule that I also
appreciate, and he showed incredible scientific hunger and interest, making him
an inspiring fellow PhD student, whose desk I am now occupying after cleaning
it up. On the same desk at the short end there sat \bold{Mikaël Trellet}, a
FIFA guru.  Mikaël is one of the few people that know that I'm an amazing cook
as I once made him my famous tagliatelle salmon-spinach-cream-fresh dish, and
he was, according to an independent source, the only guy who was funnier than
me in the lab. The last person that initially sat in this office was \bold{Marc
van Dijk}, the only other Dutch HADDOCK person that I met during my PhD. So I
want to thank him for being Dutch, for providing tips for creating an award
winning poster, and initially helping me out with installing software on my
Mac.  \bold{Christophe Schmitz} was another postdoc present in the beginning.
Christophe has had a way bigger impact on my PhD than he probably realizes, as
he referred within the HADDOCK CNS source code to the paper 'Quaternions in
molecular modeling'. This paper has been indispensable in the creation of
mainly DisVis and, to a lesser extent, PowerFit. So only for that I'm already
very grateful. The senior postdoc of the lab, the one wielding the most power,
was \bold{Adrien Melquiond}, and I had the honour of him being my daily
supervisor and co-promotor during the first part of my PhD. So thanks to
Adrien for his supervision and helping me with brainstorming during the
HADDOCK-EM project. The last person of what I call the 1st HADDOCK generation
was \bold{João Rodrigues}, my fellow PhD colleague for almost my whole stay and
together with me the bridge to the 2nd HADDOCK generation. Known around the lab
as the "Mini-BOSS", João was an unstoppable Powerplayer during CAPRI and
organizing all kinds of stuff in and out the lab. He is also the one who came
up with the name 'PowerFit' for the fitting software. So a huge thanks to him
for making my stay pleasant and easier and even possibly delivering me a very
nice postdoc position. After his departure I was the senior PhD of the group,
and I felt that with great power there comes great responsibility.

The 2nd HADDOCK generation started with the introduction of some Italian blood
from \bold{Anna Vangone}. She once made a classical Italian dinner for the
whole group, which was amazing. \bold{Li Xue} was the next person to join the
group. I want to thank her for being the unscrambler of my amateurish Chinese
speaking and her wise lessons of Chinese culture. The latest PhD addition to
our group is \bold{Cunliang Geng}, my current office mate and first
PowerProtégé. I have seen Cunliang grow up since the start of his PhD to now,
both in the lab, and in the gym, now almost being capable of doing a full free
wide-grip chin-up. I am very proud of him and give him a big thanks for the
interesting dinners in the lab. Finally, the newest member of the lab is
\bold{Zeynep Kurkcuoglu} further enforcing the girl power in the group. She is
Docker-ifying PowerFit, a noble task, that I definitely appreciate.

This leaves of course the big cheese himself, the King of the Computational
Structural Biology group \bold{Alexandre Bonvin}. Now looking back at my
interview I am a bit puzzled why I was hired, but it doesn't matter as the
whole PhD experience has been an awesome ride for the last four years, and
which worked out very smoothly. Thanks to Alex' leadership and guiding of the
full group, working here just is a great experience with inspiring science. In
an analogy to chess, Alex once emailed me that even though the King stay the
King, he has limited moves and is depending on his army. All I can respond is
that this pawn was very happy keeping the King in charge and is now about to
promote. Overall I feel heavily indebted to him for providing me this
opportunity and it has been an honour and a pleasure.

Next to the CSB group members there are of course also other people that I
want to thank. For example people that I have shared an office with:
\bold{Klaartje Houben} and \bold{Maryam Faridounnia} thanks for keeping me
company the first years. Later on \bold{Alma Svatoš} joined me in my own
office.  We had some interesting Wikipedia crawls and she was just a very funny
office mate.  She was followed up by \bold{Siddarth Narasimhan}, who is now a
fellow PhD student in the ssNMR group. Sid was often there also in the weekend,
inhibiting me from playing The Police out loud. Still, it was fun to have him
sitting at the short end of my desk, asking many questions. 

I also want to thank the other people of the NMR lab in random order.
\bold{Mohammed Kaplan}, the man with many faces, who started and will graduate
around the same time as I, and the initiator of many philosophical
conversations. \bold{Mark Daniels} for his always constructive remarks and
creative impact during the writing of my papers and for his excellent keepers
training. \bold{Eline Koers} for providing intense discussions and interesting
opinions on diverse subjects. \bold{Elwin van der Cruijsen} for setting up
badminton and football teams and for keeping me off work. \bold{Mehdi Nellen}
who was a Master student and introduced me to the time management system of
Pomodoro. \bold{Ramon van den Bos} is another gym colleague with whom I also
engage in public tennis.  \bold{Prof. dr. Marc Baldus} for the fun Italy road
trip in the Fiat Cinquecento.  \bold{Markus Weingarth} for giving solid
political advice, providing GPU resources, and his understandable enthusiasm
for PowerFit.  \bold{Dr. Hans Wienk}, \bold{prof. dr. Rolf Boelens},
\bold{prof. emer. dr. Rob Kaptein}, \bold{ing. Johan van der Zwan} and
\bold{dr. Gert Folkers} for excellent high-quality coffee table discussions.
\bold{Barbara Hendricx} for taking care of important bureaucratic arrangements.
And the remaining PhD students \bold{Deni Mance}, \bold{Cecilia Pinto} and
\bold{Ivan Corbeski} for their company, and I wish them all the best in their
coming PhD-years.

\stopBackmatterHead

\stopcomponent
