\startcomponent samenvatting

\environment layout

\startBackmatterHead[title=Samenvatting]

De opmerkelijke diversiteit en complexiteit van het leven in al zijn vormen is
een bron van constante verwondering en verbazing gedurende de hele menselijke
geschiedenis. Ook al heeft ieder uniek individu zijn eigen ervaringen en kijk
op hoe het leven te benaderen, wetenschappelijk onderzoek van de moderne mens
heeft geresulteerd in een paradigma dat leven is georganiseerd op het
moleculare niveau, daar waar de typische afstand de ångstrom (10\high{-10}m) is.
Dit inzicht heeft geleid naar een intense interesse in de moleculen van het
leven. Met de ontdekking van de dubbele-helixstructuur van DNA, het biomolecuul
dat de genetische code vasthoudt, als een voornaam voorbeeld, wordt er
gepostuleerd dat vanuit structuur de functie volgt, dat wil zeggen dat kennis
van de precieze drie dimensionale structuur van grote biomoleculen een
indicatie geeft van hun functie. Wat misschien nog belangrijker is, de precieze
kennis van de biomoleculaire structuur draagt de belofte voor het rationeel
ontwerpen van medicijnen door biologisch actieve moleculen te ontwikkelen die
specifieke interacties aangaan met bepaalde delen van het oppervlak van een
biomolecuul of in de actieve locatie van een enzym.

Röntgendiffractie en kernspinresonantie (NMR) spectroscopie zijn de klassieke
experimentele methodes die de atomaire rangschikking kunnen onthullen van grote
biomoleculen, zoals eiwitten. De invloed van eiwitten in de cel kan niet
overgewaardeerd worden: zij zijn de voornaamste spelers in bijna ieder
cellulair proces, variërend van spiercontractie tot het bouwen van andere
eiwitten middels het ribosome. Op dit moment zijn meer dan 100,000 structuren
opgelost and geplaatst in de Eiwit DataBank (\from[url:pdb]). Echter, de grote
meerderheid van deze structuren zijn individuele eiwitten, terwijl eiwitten
meestal hun functie uitvoeren door het aangaan van interacties met andere
biomoleculen, wat resulteert in grote biomoleculaire complexen. Omdat de
ingewikkeldheid van het oplossen van een structuur afhangt van meerdere
variabelen, zoals de grote van het complex, de bindingssterkte en mogelijke
membraanomgeving, en het aantal biomoleculare complexen wordt geschat op
ongeveer 100 keer het aantal van individuele eiwitten, zijn complementaire
computationele methoden vereist om het gat in de structuurkennis te verkleinen.

Integratief modelleren is een benadering om de structuur van een biomoleculair
complex te verspellen/modelleren door het combineren van alle experimentele
kennis die beschikbaar is voor het systeem. Er wordt hierbij vanuit gegaan dat
dit uiteindelijk leidt tot een accurater and preciezer model, dan wanneer
iedere data op zich wordt gebruikt. De uitdagingen van deze benadering zijn,
voorspelbaar, bemonsteren en sco\-ren: vele verschillende mogelijke modellen
moeten gegenereerd (bemonstering) worden en geëvalueerd worden om de correcte
oplossingen eruit te vissen (scoren). In dit proefschrift, focus ik
voornamelijk op het incorporeren van cryo-electron microscopie data (cryo-EM)
en chemische kruisverbindingen gekoppeld met massaspectrometrie (CXMS) in het
modelleringsproces. Cryo-EM is een snel ontwikkelende methode dat meestal
lage-resolutie dichtheidsinformatie van grote macromoleculaire complexen
oplevert, hoewel, met de huidige vooruitgang bijna-atomaire resolutie kan worden
behaald voor specifieke systemen; CXMS daarentegen verschaft afstanden tussen
residuen binnen het complex. Beide methodes leveren dus elkaar-aanvullende informatie
op.

In \inhoofdstuk[chapter:introduction] geef ik een meer technische introductie
tot integratief modelleren en de verschillende experimentele data die gebruikt
worden. Ik introduceer de belangrijkste hedendaagse software paketten die een
uitgebreid pallet aan integratieve methodes bieden voor macromoleculair
modelleren, zoals het door data-aangedreven dockingsoftware HADDOCK, dat
ontwikkeld is in het lab waar ik werk. Naast dit stel ik ook een nieuw concept
voor, namelijk explorerend modelleren, waar de nadruk ligt op het kwantificeren
en, indien mogelijk, visualizeren van de informatiehoeveelheid van de
beschikbare data. De nadruk hier constrasteert met die van integratief
modelleren, dat voornamelijk specifieke data-consistente structuren verschaft.

\inhoofdstuk[chapter:powerfit] beschrijft PowerFit, een hoge-prestatie software
pakket en programma voor het automatisch plaatsen van hoge-resolutie
starre biomoleculaire structuren in lage-resolutie cryo-EM dichtheidsmappen.
PowerFit verricht een systematische zes-dimensionale zoektocht van de drie
translationele en rotationele vrijheidsgraden om locale kruiscorrelatie-optima
te bepalen voor het objectief plaatsen van structuren in cryo-EM dichtheden.
Daarnaast introduceer ik een nieuwe en gevoeligere correlatiescore, de
zogeheten kernverzwaarde locale kruiscorrelatie die de bruikbare
resolutierijkwijdte verder uitbreidt voor het succesvol plaatsen. PowerFit is
een eerste stap in dit proefschrift om hoge-resolutie structuurdata te
combineren met lage-resolutie cryo-EM data.

Vervolgens in \inhoofdstuk[chapter:image-pyramids] kwantificeer ik de
resolutievereiste voor het succesvol plaatsen van starre biomoleculaire
structuren in een cryo-EM map als functie van de grootte van het biomolecuul.
Ook laat ik unambigu zien dat de kernverzwaarde Laplace locale kruiscorrelatie
de best presterende score is. Als laatste, omdat de resolutievoorwaarde voor het succesvol
plaatsen van een subeenheid vaak aanzienlijk lager is dan de resolutie van
huidige cryo-EM data, kan dit uitgebuit worden door het multischaal
afbeelding-pyramide concept, om de zes-dimensionale zoektocht significant te
versnellen.

\inhoofdstuk[chapter:haddock-em] bespreekt de implementatie van cryo-EM data in
HADDOCK, wat resulteert in een waarlijke integratieve modelleringsbenadering,
hetgeen de combinatie van alle andere HADDOCK-ondersteunde data toestaat. Het
centrale concept hier is het gebruik van zwaartepunten, punten die het
zwaartepunt van alle gedockte subeenheden voorstellen. HADDOCK laat het de
gebruiker vrij om ambigue afstandsbeteugelingen te gebruiken, indien de
locatie van verschillende eenheden niet kan worden onderscheiden. Het gebruik
van de lage-resolutie cryo-EM data verhoogt aanzienlijk zowel het aantal alswel
de kwaliteit van acceptable oplossingen in onze methode. Verder wordt het
gebruik van deze krachtige integratieve methode gedemonstreerd op twee ribosoom
en twee virus-antilichaam systemen, wat verdere inzichten van het tussenvlak
oplevert en toekomstige experimenten kan leiden voor verder onderzoek.

De HADDOCK2.2 webserver wordt besproken in \inhoofdstuk[chapter:haddock2.2].
De webserver biedt structuurbiologen een gebruiksvriendelijke interface tot het
gebruik van onze geupgrade HADDOCK2.2 software. Noemenswaardige functies zijn
de introductie van samengestelde molecuultypes, zoals eiwit-DNA complexen, en
verdere op NMR-gebaseerde beteugelingen, zoals residuele dipolaire-koppelingen
and vals-contact verschuivingen, wat nieuwe wegen opent voor macromoleculair
docken. De webserver kan gratis gebruikt worden voor academische doeleinden via
\from[url:haddock2.2] na een simpele registratie.

Vanaf \inhoofdstuk[chapter:disvis] richt ik me op het gebruik van
afstandsbeteugelingen in het algemeen en het gebruik van CXMS data specifiek.
Ik introduceer nog een software pakket en programma, genaamd DisVis. DisVis
kwantificeert en visualizeert de informatiehoeveelheid van
afstandsbeteugelingen via het concept van de toegankelijke interactieruimte, de
telbare set van alle data-consistente oplossingen. Bovendien kan deze methode
vals-positieve afstandsbeteugelingen identificeren, en het toont de
zelf-consistentie van de data. DisVis is een eerste stap in het nieuwe concept
van explorerend modelleren. 

De methode wordt verder uitgebreid in
\inhoofdstuk[chapter:inferring-interface-residues], waar explorerend modelleren
is gebruikt om tussenvlakresiduen te infereren. Ik toon aan dat
tussenvlakresiduen kunnen worden voorspeld tot 90\% accuraatheid voor starre
bindingscomplexen als er 3 tot 7 langeafstandsbeteugelingen bekend zijn. De
resulterende tussenvlakresiduen kunnen vervolgens gebruikt worden in HADDOCK
als actieve en passive residuen om de robuustheid van de CXMS data te vergroten
door dit te combineren met de standaard non-ambigue afstandsbeteugelingen.

In het laatste \inhoofdstuk[chapter:summary-and-perspective] som ik mijn
bevindingen op, en ik geef een persoonlijke kijk op het veld van integratief
modelleren en stel verdere toekomstige onderzoeksvelden voor. Ik pleit voor het
opleiden van de \emph{hybride wetenschapper}, een expert op het tussenvlak van
experimentele en computationele wetenschappen om het integratief modelleren
verder te brengen. Verder bepleit ik een reïnterpretatie van structuurmodellen,
aangezien zij niet beschouwd moeten worden as een enkele goed-gedefinieerde
structuur, maar eerder als een ensemble van modellen, zoals al eerder is
voorgesteld in NMR spectroscopie. Vooral in integratief modelleren, waar de
modellen meestal zijn gegenereerd met schaarse data, is de ensembleruimte groot
en diffuus. Explorerend modelleren en het concept van de telbare toegankelijke
interactieruimte kan helpen in de transitie naar de ensemble-interpretatie, en
tegelijkertijd de informatiehoeveelheid van de data kwantificeren.

In het algemeen beschrijft dit proefschrift nieuwe benaderingen en
computationele hulpmiddelen voor structuurbiologen om de experimentele data te
interpreteren, dat van toenemende mate komt van diverse bronnen, en
introduceert het een frisse kijk op modelleren via het concept van explorerend
modelleren.

\stopBackmatterHead

\stopcomponent
